
% Lastly, as an example from
%   yet another different domain,
%   this code snippet~\cite{yosysxilinxpmgen}
%   comes from
%   the open source hardware synthesis tool 
%   Yosys's~\cite{wolf2013yosys}
%   \texttt{pmgen} framework:

\begin{figure}[H]
    \centering
\begin{minted}[baselinestretch=1]{c}
subpattern in_dffe
arg argQ clock
code
  dff = nullptr;
  if (argQ.empty())
    reject;
  for (const auto &c : argQ.chunks()) {
    if (!c.wire)
      reject;
    ...
\end{minted}
    \caption{
Snippet of code
  from Yosys's pmgen framework~\cite{yosysxilinxpmgen}
  attempting to map hardware designs
  to specific hardware primitives.
    }
    \label{fig:intro:yosys-pmgen}
\end{figure}

% This code
%   captures a model of a specific hardware platform's functionality;
%   specifically, it checks whether
%   there is a \textit{D flip-flop}
%   (a specific hardware primitive)
%   on the input of a hardware module,
%   and folds it in to a larger module
%   if so
%   (which is omitted).