\begin{figure}[H]
    \centering
% https://github.com/gcc-mirror/gcc/blob/29ac92436aa5c702e9e02c206e7590ebd806398e/gcc/config/i386/i386.md#L9840C1-L9854C1
\begin{minted}[baselinestretch=1]{racket}
(define_insn "*mul<mode>3_1"
  [(set (match_operand:SWIM248 0 "register_operand" "=r,r,r")
	(mult:SWIM248
	  (match_operand:SWIM248 1 "nonimmediate_operand" "%rm,rm,0")
	  (match_operand:SWIM248 2 "<general_operand>" "K,<i>,<m>r")))
   (clobber (reg:CC FLAGS_REG))]
   ...
\end{minted}
    \caption{
x86 machine description
  for the \texttt{imul}
  family of instructions~\cite{gccx86isa}.
}
    \label{fig:intro:gcc-imul}
\end{figure}
% \noindent
% This machine description
%   is simply an explicit model\footnote{
%   \hl{https://en.m.wikipedia.org/wiki/Greenspun's_tenth_rule}
%   as a fun note, thanks Steven
%   }
%   of the underlying hardware:
%   in this case, an explicit model
%   of the hardware's instruction set.
