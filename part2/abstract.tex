\chapter*{\Cref{part:lakeroad} Abstract}

In \cref{part:lakeroad} of this dissertation,
  I apply my thesis
  to a fully separate domain:
  \gls{hardwaresynthesis} tools.
\gls{fpga} \gls{technology-mapping} is the process of
  implementing a hardware design expressed in 
  high-level HDL (hardware design language) code
  using the low-level, architecture-specific \glspl{primitive} of 
  the target FPGA.
As FPGAs become increasingly heterogeneous, 
  achieving high performance
  requires hardware synthesis tools 
  that better support mapping to complex, 
  highly configurable primitives 
  like digital signal processors (DSPs).
Current tools
  support DSP mapping via handwritten special-case mapping rules,
  which are laborious to write
  (poor \cref{thesis:devtime}),
  error-prone
  (poor \cref{thesis:correctness}),
  and often overlook mapping opportunities
  (poor \cref{thesis:optimizations}).
In \cref{part:lakeroad} of this dissertation,
  we introduce \lr,
  a principled approach to technology mapping via
  sketch-guided \gls{program-synthesis}
  (\cref{thesis:algorithms}).
A primary insight of \lr
  is to utilize vendor-provided
  simulation models
  (\cref{thesis:models})
  to generate the semantics
  needed by program synthesis.
\lr provides more
  extensible (\cref{thesis:devtime})
  technology mapping 
  with stronger correctness guarantees
  (\cref{thesis:correctness})
  and higher coverage of 
  % micro-design 
  mapping opportunities
  (\cref{thesis:optimizations})
  than state-of-the-art tools.
Across representative microbenchmarks,
  \lr produces
  1.4--3.6$\times$ the number of optimal mappings
  compared to proprietary state-of-the-art tools
  and
  6--30$\times$ the number of optimal mappings
  compared to popular open-source tools,
  while also providing correctness guarantees
  not given by any other tool.