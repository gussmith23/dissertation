
% Bold the first occurrence of a glossary word
\DeclareRobustCommand{\glossfirstformat}[1]{\textit{#1}}
\renewcommand*{\glsdisplayfirst}[4]{\glossfirstformat{#1#4}}
\usepackage[normalem]{ulem}
\renewcommand*{\glstextformat}[1]{\color{lightgray}{\dotuline{\color{black} #1}}}


% \makeglossaries
\makenoidxglossaries

% NOTE: use \glstext and not \gls in this file.
% This prevents the glossary from thinking that uses of these words in the glossary are the first occurrence in text. We'd rather those first occurrences happen e.g. in the intro.

\newglossaryentry{compiler}
{
    name=compiler,
    description={
A tool which converts between representations.
    }
}


\newglossaryentry{target}
{
    name=target,
    description={
The platform which a \glstext{compiler} generates code for.
    }
}

\newglossaryentry{compilerbackend}
{
    name=compiler backend,
    description={
The portion of a \glstext{compiler} that concerns the \glstext{target},
  e.g., target-specific optimizations and code generation.
    }
}

\newglossaryentry{validation}
{
    name=validation,
    description={
The process of sanity-checking a program or hardware design using a
  limited, finite set
  of inputs, and judging the correctness 
  of the outputs.
Also referred to as \textit{testing.}
Validation should specifically be distinguished from
  \glstext{verification}.
Confusingly, 
  in the world of hardware design,
  validation is often referred to as
  verification,
  while verification (by our definition)
  is referred to as \textit{formal} verification.
    }
}

\newglossaryentry{verification}
{
    name=verification,
    description={
The process of mathematically proving correctness
  about a program or hardware design.
While the proofs of correctness themselves
  can have limitations,
  verification is generally more thorough
  than \glstext{validation} or \textit{testing.}
In the world of hardware design,
  ``verification''
  generally refers to what we call validation or testing,
  while ``formal verification''
  refers to our verification.
    }
}

\newglossaryentry{hardwaresynthesis}
{
    name={hardware synthesis},
    description={
Another term for hardware compilation.
The process of converting a hardware design
  captured in a high-level language
  to a low-level target implementation,
  for example,
  a netlist which can be programmed onto an FPGA
  or a geometry file to be made into an ASIC.
  \hl{define the terms in here}
    }
}