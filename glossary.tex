
\ifarxiv
% do nothing; bottom command is broken for arxiv
\else
% Bold the first occurrence of a glossary word
\DeclareRobustCommand{\glossfirstformat}[1]{\textit{#1}}
\renewcommand*{\glsdisplayfirst}[4]{\glossfirstformat{#1#4}}
\usepackage[normalem]{ulem}
\renewcommand*{\glstextformat}[1]{\color{lightgray}{\dotuline{\color{black} #1}}}
\fi


% \makeglossaries
\makenoidxglossaries

% NOTE: use \glstext and not \gls in this file.
% This prevents the glossary from thinking that uses of these words in the glossary are the first occurrence in text. We'd rather those first occurrences happen e.g. in the intro.

\newglossaryentry{compiler}
{
    name=compiler,
    description={
A tool which converts between representations}
}


\newglossaryentry{target}
{
    name=target,
    description={
The platform which a \glstext{compiler} generates code for}
}

\newglossaryentry{compilerbackend}
{
    name=compiler backend,
    description={
The portion of a \glstext{compiler} that concerns the \glstext{target},
  e.g., target-specific optimizations and code generation}
}

\newglossaryentry{validation}
{
    name=validation,
    description={
The process of sanity-checking a program or hardware design using a
  limited, finite set
  of inputs, and judging the correctness 
  of the outputs.
Also referred to as \textit{testing.}
Validation should specifically be distinguished from
  \glstext{verification}.
Confusingly, 
  in the world of hardware design,
  validation is often referred to as
  verification,
  while verification (by our definition)
  is referred to as \textit{formal} verification}
}

\newglossaryentry{verification}
{
    name=verification,
    description={
The process of mathematically proving correctness
  about a program or hardware design.
While the proofs of correctness themselves
  can have limitations,
  verification is generally more thorough
  than \glstext{validation} or \textit{testing.}
In the world of hardware design,
  ``verification''
  generally refers to what we call validation or testing,
  while ``formal verification''
  refers to our verification}
}

\newglossaryentry{hardwaresynthesis}
{
    name={hardware synthesis},
    description={
Another term for hardware compilation.
The process of converting a hardware design
  captured in a high-level language
  to a low-level target implementation,
  for example,
  a \gls{netlist} which can be programmed onto an FPGA
  or a geometry file to be made into an ASIC}
}

\newglossaryentry{netlist}{
    name={netlist},
    description={
Much like \glstext{rtl} or \glstext{hls}, netlists are
  a way to capture a hardware specification.
Netlists are lower-level than RTL or HLS, however.
As their name implies,
  netlists are simply lists of ``nets'' (or wires),
  plus, importantly,
  the modules they are connected to}
}

% Note: this is a nice hack from
% https://tex.stackexchange.com/questions/8946/how-to-combine-acronym-and-glossary
% which gives acronym-like behavior but
% still allows for a definition.
\newglossaryentry{dsl}
{
    name={DSL},
    first={domain-specific language (DSL)},
    description={
A programming language designed for a narrow domain, generally for a specific purpose.
DSLs are generally smaller
  than general-purpose programming languages,
  with fewer complicated features
  like general control flow.
This makes them easier to reason about
  and optimize}
}

\newglossaryentry{mlkernel}
{
    name={machine learning kernel},
    description={
A core subroutine used within many machine learning workloads---for example,
  matrix multiplication.
Machine learning kernels are often highly optimized,
  often by building custom hardware to implement
  them efficiently}
}

\newglossaryentry{instruction-selection}
{
    name={instruction selection},
    description={
A core compiler algorithm
  in which higher-level,
  hardware-independent operations
  are lowered to
  hardware-specific instructions~\cite{blindell2016instruction}}
}

\newglossaryentry{technology-mapping}
{
    name={technology mapping},
    description={
A core compiler algorithm
  in \glstext{hardwaresynthesis} tools
  in which
  hardware-independent components of a design
  (e.g.~a Verilog multiply operator \texttt{*})
  are lowered to
  hardware-specific primitive instantiations
  (e.g.~a Xilinx DSP48E2 DSP)}
}

\newglossaryentry{hls}
{
    name={HLS},
    first={High-Level Synthesis (HLS)},
    description={
A set of tools~\cite{cong2011high}
  for compiling high-level implementations of algorithms
  (often written in C)
  into hardware designs.
HLS is often contrasted against
  \glstext{rtl},
  written in languages like Verilog or VHDL,
  which is a comparatively lower level
  of abstraction}
}

\newglossaryentry{rtl}
{
    name={RTL},
    first={Register Transfer Level (RTL)},
    description={
A specific level of abstraction for a hardware design specification,
  in which registers (hardware modules which hold state)
  are explicit in the design,
  but computation is still modeled at a high level.
This is in contrast to a higher level representation
  like \glstext{hls}, which does not include timing,
  or a lower, gate-level or netlist representation
  which may use specific gates or other hardware primitives}
}

\newglossaryentry{equality-saturation}
{
    name={equality saturation},
    description={
A term rewriting algorithm~\cite{tate2009equality,tate2011equality,willsey2021egg}
  which uses a specific data structure---%
  the \egr---%
  to capture a large space of equivalent programs}
}

\newglossaryentry{tensorization}
{
    name={tensorization},
    description={
An algorithm within machine learning compilers
  which uncovers places to invoke primitives
  operating over tensors---%
  often in the form of accelerator 
  invocations~\cite{tvmtensorization}.
Tensorization is analogous to vectorization
  in standard compilers,
  in which groups of instructions are translated
  into a single vector instruction
  to improve performance}
}

\newglossaryentry{accelerator}
{
    name={accelerator},
    description={
Custom hardware, specialized to a specific task
  or set of tasks.
In this dissertation,
  we are most frequently referring to
  accelerators for
  \glstext{mlkernel}s}
}

\newglossaryentry{program-synthesis}
{
    name={program synthesis},
    description={
A broad term capturing a set of 
  techniques for generating programs,
  often using automated reasoning tools~\cite{gulwani2017program}.
In this dissertation,
  we are generally referring to solver-aided synthesis,
  which formulates program generation as a constraint solving problem
  and applies solvers (e.g.~\glstext{smt} solvers).
Specifically, we are generally referring to
  \textit{sketch-guided} program synthesis,
  in which the final compiled program
  is captured as a sketch, with holes to be filled in by the solver~\cite{solar2008program}}
}

\newglossaryentry{primitive}
{
    name={primitive},
    description={
The atomic units of a language or representation.
In the context of this dissertation, we often mean
  either \glstext{accelerator} primitives
  or hardware primitives.
Accelerator primitives are operations
  implemented by accelerators which are, from the 
  view of software, black box operations
  which cannot be split into smaller sets of instructions.
They represent the smallest unit
  a compiler can target.
Hardware primitives are the modules
  provided for use by a hardware platform---%
  e.g. the gates and devices available for use
  on an FPGA.
A \glstext{netlist}
  is composed of primitive instantiations}
}

\newglossaryentry{fpga}
{
    name={FPGA},
    description={
Field Programmable Gate Arrays are
  a reprogrammable hardware platform
  allowing users to design hardware
  without fabricating an \glstext{asic}.
FPGAs are pre-fabricated chips
  composed of programmable \glstext{primitive}s.
The FPGA can be reprogrammed
  to configure and connect the various primitives,
  producing different hardware designs}
}

\newglossaryentry{asic}
{
    name={ASIC},
    first={Application-Specific Integrated Circuit (ASIC)},
    description={
A hardware chip produced for a specific purpose.
In contrast to FPGAs, which are reprogrammable,
  ASICs are static.
As a consequence, the design of ASICs is an intensive process
  involving significant \glstext{validation} and \glstext{verification}
  to ensure hardware correctness}
}

\newglossaryentry{automated-reasoning}
{
    name={automated reasoning},
    description={
A term to capture algorithms such as \glstext{smt} or \glstext{equality-saturation}.
In general, automated reasoning algorithms
  are able to solve
  complex constraint or optimization problems
  by applying mathematical and logical rules.
Automated reasoning algorithms
  are generally applicable to many tasks in computer science,
  as long as the can be captured
  in a way the algorithm can understand}
}


\newglossaryentry{smt}
{
    name={SMT},
    first={SMT (SAT modulo theories)},
    description={
A class of constraint problem.
SMT is simply SAT---Boolean satisfiability---%
  with extra ``theories'' added in,
  somewhat like libraries.
For example, the theory of fixed-width bitvectors,
  which includes operations over groups of bits.
SMT solvers are \glstext{automated-reasoning} algorithms
  which solve SMT problems}
}