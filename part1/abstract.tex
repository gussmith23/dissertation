\chapter*{\Cref{part:glenside-and-3la} Abstract}

In \cref{part:glenside-and-3la},
  I describe an application of my underlying thesis
  to the generation of compilers
  for machine learning \glspl{accelerator}.
Specialized hardware, especially for high-performance fields
  such as machine learning,
  have only grown in importance
  over the last decade.
Despite this increased importance,
  it remains surprisingly difficult
  to build compilers for specialized hardware.
Existing approaches
  require significant
  developer effort (\cref{thesis:devtime}),
  and 
  often leave
  \cref{thesis:optimizations} on the table.
Because building a satisfactory compiler
  remains so challenging,
  many hardware designers choose not to.
Without a compiler,
  designers are unable to test
  their hardware on full workloads,
  leaving crucial bugs undiscovered
  (\cref{thesis:correctness}).
In the following chapters,
  I describe how,
  responding to the lack of testing
  in the accelerator design
  community,
  we applied
  my thesis
  to automatically generate compiler backends
  targeting machine learning accelerators.
Specifically, we utilized
  \gls{equality-saturation}
  (\cref{thesis:algorithms})
  driven by rewrites capturing
  the functional behavior of accelerators
  (\cref{thesis:models})
  to produce a backend
  which 
  requires little developer effort to use
  (\cref{thesis:devtime})
  but finds more mapping to accelerators
  than existing work
  (\cref{thesis:optimizations})
  and enables hardware designers to run
  crucial end-to-end testing
  (\cref{thesis:correctness}).
This is wrapped up inside a language and tool called
  \g;
  \g was then integrated into a larger compiler
  called 3LA.
\Cref{part:glenside-and-3la} draws from both the 
  \g paper,
  ``Pure Tensor Rewriting via Access Patterns''~\cite{smith2021pure},
  and the 
  3LA paper,
  ``Application-level Validation of Accelerator Designs Using a Formal Software/Hardware Interface''~\cite{huang2024application}.
% This compiler is an integral piece of 3LA,
%   a novel methodology
%   for building and testing hardware accelerators.
% \hl{the rest still TODO}
% In \cref{sec:part1-background}
%   we give related background.
% In \cref{sec:part1-motivation},
%   we introduce the
%   motivation for this part.
% In \cref{sec:part1-glenside},
%   we introduce \g,
%   \hl{a language for equality saturation.}
% In \cref{sec:part1-evaluation}
%   we present an evaluation.
