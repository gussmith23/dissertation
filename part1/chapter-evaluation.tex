\chapter{Evaluation}
\label{chapter:part1-evaluation}

Thus far, we have described
  how we have applied
  the thesis of this dissertation---%
  that compiler backends
  should be generated
  from formal models of hardware---%
  in the realm of deep learning accelerators.
We first described the difficulties
  in developing compilers
  for deep learning accelerators.
We then described how these difficulties
  motivated the creation of
  3LA: a mostly-automated, end-to-end
  methodology
  for accelerator development.
(Note again that 3LA itself
  is not a contribution of this dissertation.)
We identified a specific problem
  within this domain
  as a problem of interest:
  mapping applications to accelerators.
To address this problem,
  this dissertation
  introduced
  \g,
  a tensor language
  which enables powerful rewriting techniques.
We then integrated \g 
  into \TLA,
  to bring the power of equality saturation
  to bear on the task
  of mapping to accelerators.
  
This evaluation will first demonstrate the utility of
  \g
  via a number of case studies
  in \cref{sec:case-studies}.
Then, 
  in \cref{secion:eval-3la}
  we will evaluate the specific claims
  of our thesis
    (improved \cref{thesis:optimizations},
     \cref{thesis:correctness}, and
     \cref{thesis:devtime})
  by evaluating the components of \TLA 
  to which \g was essential.

  


\section{Case Studies}
\label{sec:case-studies}

% Unsure where this should go for now
%\begin{figure}
%\begin{lstlisting}[escapechar=!]
%matrix multiplication:
%!\colorbox{cyan}{(compute dotProd}!
% !\colorbox{cyan}{(cartProd}!
%  (access a 1)
%  (transpose (access b 1) (list 1 0))!\colorbox{cyan}{))}!
%    ...map to hardware...
%!\colorbox{CarnationPink}{(systolic-array ?rows ?cols}!
% (access a 1)
% (access b 1)!\colorbox{CarnationPink}{)}!
%  
%  
%conv2d:
%!\colorbox{cyan}{(compute dotProd}!
% !\colorbox{cyan}{(cartProd}!
%  (access weights 1)
%  (windows (access activations 4) ...)!\colorbox{cyan}{))}!
%      ...im2col...
%!\colorbox{cyan}{(compute dotProd}!
% !\colorbox{cyan}{(cartProd}!
%  (access weights 1)
%  (flatten 
%   (windows (access activations 4) ...))!\colorbox{cyan}{))}!
%    ...map to hardware...
%!\colorbox{CarnationPink}{(systolic-array ?rows ?cols}!
% (access weights 1)
% (flatten
%  (windows (access activations 4) ...))!\colorbox{CarnationPink}{)}!
%\end{lstlisting}
%
%    \caption{
%    \hl{TODO hideous. make fixed width, make smaller than text. unsure where this should go}
%    \g{}'s access pattern
%      data structure
%      reveals the similarity
%      between
%      dense matrix multiplication
%      and 2D convolution,
%      and elegantly leads
%      to the discovery
%      of the \tcd{im2col}
%      transform,
%      allowing both kernels
%      to be mapped
%      to the same hardware.}
%       %zach's case for figure 5: it's context free understandable. we need to go back to somethign more visual. make a version that someone can get with 5 seconds of viewing,and then maybe has more information for someone with a minute of viewing.
%    \label{fig:blah}
%\end{figure}

%\hl{1. represent kernels, 2. map to hardware w/ generic rewrite, 3. expand/make flexible that rewrite to other workloads}

To demonstrate \g's utility,
  we first show how it enables
  concise specifications of several
  critical ML kernels
  (Section~\ref{section:representing-kernels}).
We then show how
  \g's pure, binder-free
  % I don't think the shape-polymorphism is relevant here (actually, this rewrite is explicitly non-shape-polymorphic!
  %, shape-polymorphic
  representation enables mapping kernels
  to an example accelerator via
  direct application of generic rewrite rules
  (Section~\ref{sec:case-study-tensorization}).
Finally,
  we highlight how \g
  enables the
  flexible mapping of
  larger, more diverse kernels
  to our accelerator,
  utilizing the power
  of equality saturation
  to automatically discover
  a variety of program transformations.
Specifically,
  we show how \g can automatically
  map convolutions to matrix multiplications
  (Section~\ref{sec:discovering-im2col})
  and automatically
  map large matrix multiplications into a
  sequence of smaller matrix multiplications
  (Section~\ref{sec:case-study-blocking}).
This portion of the evaluation is drawn from
  Smith et al.~\cite{smith2021pure}.
  
  
% We then show how
%   \g's pure, binder-free representation
%   enables \textit{automatically discovering}
%   a variety of program transformations
%   for optimizing tensor programs and
%   mapping them to hardware accelerators
%   via term rewriting.
% First, Section~\ref{sec:case-study-tensorization}
%   shows how matrix multiplications can be
%   mapped to systolic arrays in hardware
%   via a single, generic \g rewrite.
% Next,   
%   Section~\ref{sec:discovering-im2col} and
%   Section~\ref{sec:case-study-blocking} show
%   using a database of similar rewrites along
%   with equality saturation to
%   map convolutions to matrix multiplications
%   and large matrix multiplications into a
%   sequence of smaller matrix multiplications.


  
% To show how this simple approach
%   can support larger workloads,
%   we also illustrate using rewrites in \g to
%   map convolutions to matrix multiplications
%   (Section~\ref{sec:discovering-im2col}),
%   and
%   maps large matrix multiplications into a
%   sequence of smaller matrix multiplications
%   (Section~\ref{sec:case-study-blocking}).
  
  
  
% To map convolutions to matrix ,
%   which ..., and
%   to map larger matrix multiplications into
%   smaller matrix multiplications

    
% In particular,
%     we show how \g can automatically map
%     various kernels to systolic arrays, including
%     matrix multiplications and
%     (Section \ref{sec:case-study-tensorization}),
%     2D convolutions
%     (Section \ref{sec:case-study-tensorization}).
    

    
%     map matrix multiplications to systolic arrays
%     (Section \ref{sec:case-study-tensorization}),
%     map 2D convolutions to matrix multiplication accelerators
%     (Section \ref{sec:case-study-tensorization}), and
%     ...

% We then use
%   equality saturation
%   (provided by the \texttt{egg} library~\cite{willsey2021egg})
%   to implement a term rewriting system
%   over \g, to
%   map matrix multiplications
%   to systolic arrays
%   (Section \ref{sec:case-study-tensorization}),
%   run 2D convolutions on matrix multiplication hardware
%   (Section \ref{sec:discovering-im2col}),
%   and discover
%   matrix multiplication
%   blocking schemes
%   (Section \ref{sec:case-study-blocking}).
  
  
%\g is implemented in ~25k lines of
%  Rust.
  
  
%\subsection{Representation of Common Machine Learning Kernels}
\subsection{Representation of Common ML Kernels}
\label{section:representing-kernels}

Figure~\ref{fig:all-kernels}
  lists the \g specifications
  of three common ML kernels:
  2D convolution,
  matrix multiplication,
  and max pooling.
Below, we discuss
  the specifications of 
  2D convolution
  and max pooling;
  see 
  Section~\ref{sec:glenside}
  for a description 
  of matrix multiplication.
  
\subsubsection*{2D Convolution}

2D convolution (\ctd{})
  is a core kernel
  in deep learning,
  defined element-by-element %(as in the TVM and TensorFlow documentation)
  over tensors storing
  activations $A$,
  strides $S$, and
  weights $W$ as: 
  % (from the Relay docs \cite{relay-conv2d}):
  % note: TF's tf.nn.conv2d gives almost an identical definition
\begin{equation*}%\label{eq:conv2d}
\begin{split}
\mbox{out}&[n, o, x, y] =\\
\sum_{dx, dy, c}&
    (A[n, c, S[0] \cdot x  + dx, S[1] \cdot y + dy] \
    \cdot W[o, c, dx, dy])
\end{split}
\end{equation*}
where
  $n$ indexes the output batch,
  $o$ indexes output channels,
  $x$/$y$ index spatial dimensions,
  $dx$/$dy$ index
    the convolutional window spatial dimensions,
  and $c$ indexes input channels.
%Written this way,
%  we can see that
%  each output location
%  $\mbox{out}[n,c,y,x]$,
%  is
%  the result of
%  a ``generalized'' dot product
%  of a three-dimensional window
%  of the ``data'' input
%  with one of the $c$ 
%  three-dimensional filters
%  in the ``weight'' input.
%Computationally,
%  a generalized dot product
%  is straightforward:
%  it's a pairwise multiplication
%  between two sets of numbers
%  followed by 
%  a reduction sum.
%Most of the complexity
%  of the \ctd{} kernel
%  is 
%  not in the computation itself,
%  but
%  in
%  how the data
%  is \textit{accessed:}
%  how we correctly order the values
%  to be multiplied and accumulated.
%\ctd{}
%  % TODO mention that padding also happens here?
%  must first
%  access the ``data'' input
%  as windows,
%  respecting the kernel height,
%  kernel width,
%  and stride parameters.
%It then
%  takes the
%  Cartesian product
%  of the windows
%  with
%  the list of filters in ``weight''.
%Additionally,
%  not shown in equation
%  \ref{eq:conv2d},
%  it's also common
%  to apply padding
%  to the input data,
%  which can be done
%  as the data is being accessed.
%It is only once
%  these accesses are set up
%  that we are able
%  to map
%  the generalized dot product
%  to calculate the result.
2D convolution
  slides each of the $o$
  filters
  of shape $(c, dx, dy)$
  through each possible
  $(c, dx, dy)$--shaped window
  of the input images.
At each of these locations,
  an elementwise multiplication
  and reduction sum
  is computed.

The \g specification
  of \ctd{}
  is shown in 
  Figure \ref{fig:conv2d}.
We access
  the \texttt{weights}
  as a vector of $O$ filters
  and the \texttt{activations}
  as a vector of $N$ images.
We leave the filters as they are,
  but form windows
  of shape
  $(C, K_h, K_w)$
  over the activations
  using the \texttt{windows}
  access pattern transformer
  (Table~\ref{tab:access-pattern-transformers}).
This produces an access pattern
  of shape
  \accesspatternshape
  {N, 1, H', W'}
  {C, K_h, K_w},
  i.e.,
  a batch of ``images''
  of new spatial shape
  $(H', W')$,
  where every location
  is a window of
  the original input.
Finally,
  we take the Cartesian product
  of the filters
  and the windows,
  compute their dot product,
  and \texttt{squeeze} and \texttt{transpose}
  the output
  into the correct layout.
  
%There is very little
%  computation
%  represented
%  in the \g{} code---%
%  only the outermost 
%  \texttt{compute dotProd}.
%Most of the complexity
%  is in expressing
%  exactly how the large,
%  multi-dimensional tensors
%  are accessed.
%Surprisingly,
%  the base structure
%  of \ctd{}
%  looks a lot like
%  our matrix--matrix multiplication
%  example:
%  we take the
%  Cartesian product
%  of two access patterns
%  and map a dot product
%  over the result.
%In matrix--matrix multiplication,
%  the two access patterns
%  access the rows of the first matrix
%  and the columns
%  of the second matrix,
%  respectively.
%What are
%  the two access patterns
%  in the case of
%  \ctd{}?
  

%In \g,
%  we construct
%  the convolutional window access pattern
%  with a nested series
%  of access pattern transformers.
%We begin
%  by converting the tensor
%  into an access pattern.
%We then pad
%  the access pattern
%  with \texttt{access-pad}.
%We add 1 length
%  of zero-padding
%  before and after
%  dimensions 2 and 3.
%Note that this
%  could also be done
%  as a pre-processing step.
%We access the padded tensor
%  at dimension 4,
%  which is simply to prepare it
%  to be input to
%  \texttt{access-windows}.
%At this point,
%  our access pattern
%  is of shape
%  \accesspatternshape
%  {1, 3, 34, 34}
%  {}.
%We then use
%  \texttt{access-windows},
%  which takes three arguments:
%  the access pattern,
%  the window shape,
%  and the strides.
%\texttt{access-windows}
%  expects an access pattern
%  of shape
%  \accesspatternshape{(a_0,\dots,a_n}{}.
%The window shape
%  $(w_0, \dots, w_n)$
%  and the strides
%  $(s_0, \dots, s_n)$
%  are the same length
%  as the access pattern shape.
%The resulting access pattern is of shape
%  \accesspatternshape{a'_0,\dots,a'_n}{s_0, \dots, s_n},
%  where $a'_i$
%  is determined by
%  the number
%  of
%  valid window locations,
%  given the input shape,
%  the window shape,
%  and the strides.
%Specifically, $a'_i$
%  is the number of locations
%  in which
%  we can lay down
%  the edge of the window
%  in dimension $i$,
%  given that the window
%  is length $w_i$ along dimension $i$,
%  and given that we are striding by
%  $s_i$ along dimension $i$.
%In our example,
%  our input access pattern
%  is of shape
%  \accesspatternshape{1, 3, 34, 34}{},
%  and our window shape and stride
%  are 
%  $(1, 3, 3, 3)$ and
%  $(1, 1, 1, 1)$,
%  respectively.
%Along the first (batch) dimension,
%  we can lay down a window
%  in exactly one location,
%  as the dimension is length 1,
%  and the window is of length 1.
%This is the same
%  for the second (channel) dimension,
%  where the dimension is length 3,
%  and the window
%  is of length 3.
%However, for the last two
%  (spatial, height and width)
%  dimensions,
%  we can lay down
%  a window edge
%  of length 3
%  in multiple places
%  along a dimension
%  of length 34:
%  specifically, 32 places.
%Thus,
%  our resulting access pattern
%  is
%  \accesspatternshape
%  {1, 1, 32, 32}
%  {1, 3, 3, 3}.
%
%To finish
%  preparing
%  the convolutional window
%  access pattern,
%  we ``squeeze''
%  (remove)
%  some of the dimensions
%  of length 1
%  produced by
%  \texttt{access-windows},
%  and re-access
%  the tensor
%  at dimension 3.
%The result
%  is an access pattern
%  with shape
%  \accesspatternshape
%  {1, 32, 32}{3, 3, 3}.
%  
%We now have
%  an access pattern
%  over the weights
%  of shape
%  \accesspatternshape
%  {8}{3, 3, 3}
%  and an access pattern
%  over the activations
%  of shape
%  \accesspatternshape
%  {1, 32, 32}
%  {3, 3, 3}
%Finally,
%  we take their Cartesian product---%
%  which has shape
%    \accesspatternshape
%    {8, 1, 32, 32}
%    {2, 3, 3, 3}---%
%  and map a dot product
%  over the result,
%  giving an access pattern
%  with shape
%    \accesspatternshape
%    {8, 1, 32, 32}
%    {}.
%Finally,
%  we transpose
%  the result
%  to give us
%  an access pattern
%  of shape
%  \accesspatternshape
%  {1, 8, 32, 32}{},
%  which is back in
%  \texttt{NCHW}
%  format.

\subsubsection*{Max Pooling}

Max pooling, commonly used in ML
  to condense intermediate activations (``act'' below),
  is defined as:
\begin{equation*}%\label{eq:maxpool}
\begin{split} 
\mbox{out}&[n, c, x, y] =\\
\max_{dx, dy}&
           (\mbox{act}[n, c,
                       \mbox{strides}[0] \cdot x  + dx,
                       \mbox{strides}[1] \cdot y + dy])
\end{split}
\end{equation*}

Max pooling 
  slides a window
  of shape $(dx, dy)$
  over all possible locations
  within the spatial (i.e.,~$x$ and $y$)
  dimensions.
At each window location,
  it reduces the window
  to a scalar
  with the $\max$ operator.
The \g specification merely applies
  \texttt{reduceMax} 
  over each two-dimensional window.
  
%To implement max pooling
  %in \g,
  %we first access 
  %the activation tensor
  %as a two-dimensional matrix
  %of two-dimensional matrices.
%We form windows
  %over each of these matrices,
  %and finally,
  %we compute over each window
  %with the \texttt{reduceMax}
  %operator.
  
\subsubsection*{Discussion}\label{section:kernel-implementation-discussion}

\g separates
  the \textit{computation}
  from the \textit{data access patterns}
  in these kernels while exposing
  the simplicity of their computation---%
  and the relative complexity
  of their data access.
In all three kernels,
  the computation can be described
  with a single operator;
  most of the specification
  entails
  setting up the data access pattern.

Furthermore,
  \g exposes similar structure
  between kernels;
  for example,
  both \ctd
  and matrix multiplication
  feature the expression
  \tcd{(compute dotProd (cartProd ...))}.
  
At their core, these kernels
  are performing the same computation,
  but with different patterns
  of data access.
In Section~\ref{sec:discovering-im2col},
  we exploit this similarity in structure
  when mapping kernels to hardware.
%As we'll see in 
  %Section~\ref{sec:discovering-im2col},
  %\g can do more
  %than just highlight
  %this similar structure---%
  %it can exploit it
  %when mapping to hardware.
  
These kernels highlight the expressive power
  of access patterns.
Consider the use of 
  \tcd{windows}
  in \ctd
   and max pooling.
Both kernels
  form windows
  differently:
  \ctd forms three-dimensional
  windows
  over the channels, height, and width
  dimensions,
  while max pooling forms two-dimensional windows
  over the height and width.
Rather than passing configuration parameters to \tcd{windows},
  \g attaches this information to the tensors themselves.
  
%With standard tensors,
  %we would have to express
  %these different intents
  %with a set of configuration parameters
  %passed in to \tcd{windows}.
%But with access patterns,
  %we can actually express this
  %via the extra information
  %attached to the tensor itself.
  
\begin{figure}
\begin{lstlisting}[escapechar=!]
(compute dotProd (cartProd ?a0 ?a1)) 
  !$\Longrightarrow$! (systolicArray ?rows ?cols ?a0 (access (transpose ?a1 (list 1 0)) 0))
!$\textrm{where \texttt{?a0} is of shape $((\mbox{\texttt{?batch}}), (\mbox{\texttt{?rows}}))$ and \texttt{?a1} is of shape $((\mbox{\texttt{?cols}}), (\mbox{\texttt{?rows}}))$}$!
\end{lstlisting}
\caption{Our rewrite rewriting matrix multiplication to a systolic array invocation.}
    \label{fig:systolic-array-rewrite}
\end{figure}
  
% Trying to get this on the right page.
\begin{figure}
\begin{lstlisting}[escapechar=!]
?a !$\Longrightarrow$! (reshape (flatten ?a) ?shape) 

(cartProd (reshape ?a0 ?shape0) (reshape ?a1 ?shape1))
  !$\Longrightarrow$! (reshape (cartProd ?a0 ?a1) ?newShape)
  
(compute dotProd (reshape ?a ?shape)) 
  !$\Longrightarrow$! (reshape (compute dotProd ?a) ?newShape)
\end{lstlisting}
\caption{Rewrites enabling the discovery of the \itc transformation.}
\label{fig:im2col-rewrites}
\end{figure}



%\subsection{Mapping Dense \tcd{matMul}Matrix Multiplication to Accelerators}
\subsection{Mapping \tcd{matMul} to Accelerators}
%\subsection{Mapping Matrix Multiplication to Accelerators}
\label{sec:case-study-tensorization}


\g can be used to uncover opportunities
  to invoke accelerator components---%
  indeed, this is exactly how \g
  is used
  within \TLA.
  %to discover places in our program
  %where we can invoke
  %accelerator components.
Consider a 
  weight-stationary systolic array,
  a common matrix multiplication
  architecture.
  %(commonly used for this purpose).
  %,
  %a common architecture
  %for implementing
  %matrix multiplication,
  %in which a weight matrix
  %is loaded
  %into the array
  %and remains stationary
  %while the activation matrix
  %is streamed through.
  %which loads and fixes a weight matrix in %the array
  %as the activation matrix is streamed %through.
A weight-stationary
  systolic array
  with $r$ rows
  and $c$ columns
  takes two lists
  of length-$r$ vectors
  (the activations
    and weights, respectively),
  pairing each vector
  from one list
  with each vector
  from the other,
  and computes a dot product
  over each pair.
The second list
  contains $c$ vectors,
  while the first
  can be of any length.

As discussed in \cref{sec:glenside-in-3la},
  \g's purity
  allows us to implement this hardware mapping task
  using a term rewriting system,
  in which we rewrite a matching program pattern
  to an invocation of our systolic array.
Our rewrite is shown in 
  Figure~\ref{fig:systolic-array-rewrite},
  mimicking
  \tcd{egg}'s rewrite syntax.
Tokens starting with a question mark
  (such as \texttt{?a0} in 
  Figure~\ref{fig:systolic-array-rewrite})
  are variables in the pattern,
  bound by the left-hand side (LHS),
  and then used on the right-hand side (RHS).
\tcd{egg} also allows for
  conditions on rewrites,
  which we print below our rewrites.

To design our rewrite,
  we first must design
  the LHS
  to match program patterns
  that resemble the data access pattern
  and compute pattern
  of our systolic array.
\g is eminently suitable for this task,
  as it can express
  %exposes
  %the perfect level
  %of abstraction
  %for this task,
  %allowing us to encode
  exactly the data access
  and computation pattern
  we described
  for the systolic array.
Pairing all vectors from one list
  with all vectors from another
  and computing the dot product
  of the pairs
  is represented as
  \tcd{(compute dotProd (cartProd ?a0 ?a1))},
  binding
  \tcd{?a0}
  and \tcd{?a1}
  to the input access patterns.
We encode
  the systolic array's
  constraints
  on the input shapes
  as a condition on the rewrite.
Patterns which match the LHS
  are mapped to the RHS;
  in this case, we introduce a new
  \tcd{systolicArray} construct
  to represent the functioning of our systolic array.
The shape of the systolic array 
  is given by the \tcd{?rows} and \tcd{?cols}
  parameters,
  and the inputs are given
  as access patterns.
Note how we also transform
  the second access pattern
  to more accurately convey
  how the actual systolic array
  hardware
  accesses the weight tensor:
  it reads it all at once
  (hence, \tcd{(access ... 0)}),
  and expects it to be laid out
  in transposed form
  in memory.
This added information---%
  enabled by \g's access patterns---%
  provides richer data layout information,
  potentially helping future rewrites
  or code generation steps.
 
\subsection{Flexible Mapping: Discovering \itc{}}
\label{sec:discovering-im2col}
%\begin{figure}
%   \centering
%\begin{lstlisting}[escapechar=!]
%?a !$\Longrightarrow$! (reshape (flatten ?a) ?shape)
%!$\textrm{where \texttt{?shape} is the shape of \texttt{?a}}$!
%\end{lstlisting}
%\caption{Exploratory flatten--reshape rewrite.}
%\label{fig:flatten-unflatten}
%\end{figure}

%\begin{figure}
%\begin{lstlisting}[escapechar=!]
%(cartProd
% (reshape ?a0 ?a0Shape)
% (reshape ?a1 ?a1Shape))
%     !$\Longrightarrow$! (reshape (cartProd ?a0 ?a1) ?newShape)
%!$\textrm{where \texttt{?newShape} is the shape of the LHS \texttt{cartProd}}$!
%
%(compute dotProd (reshape ?a ?shape))
%     !$\Longrightarrow$! (reshape (compute dotProd ?a) ?newShape)
%!$\textrm{where \texttt{?newShape} is the shape of the LHS \texttt{compute dotProd}}$!
%\end{lstlisting}
%\caption{Composition commutativity of \texttt{reshape} with \texttt{cartProd} and \texttt{compute dotProd}.}
%\label{fig:reshape-composition-commutative}
%\end{figure}

\begin{figure}
\begin{lstlisting}[escapechar=!]
(transpose                   
 (squeeze                    
  (reshape           !\color{gray}; \hspace{2mm}\accesspatternshape{N,1,H',W',O}{}!
   (compute dotProd  !\color{gray}; \hspace{2mm}\accesspatternshape{N \cdot 1 \cdot H' \cdot W',O}{}!          
    (cartProd                 
     (flatten        !\color{gray}; \hspace{2mm}\accesspatternshape{N \cdot 1 \cdot H' \cdot W'}{C \cdot K_h \cdot K_w}!
      (windows (access activations 1)  
               (shape C Kh Kw) (shape 1 Sh Sw)))
     (flatten        !\color{gray}; \hspace{2mm}\accesspatternshape{O}{C \cdot K_h \cdot K_w}!
      (access weights 1))))
   ?shape) 1) (list 0 3 1 2))
     \end{lstlisting}
     \vspace{-1em}
    \caption{An \tcd{im2col}-transformed 
      \ctd,
    after the application of the rewrites
    in Figure~\ref{fig:im2col-rewrites}
    and just before the application
    of the systolic array rewrite.}
    \label{fig:conv2d-im2col-rewritten}
\end{figure}
  
The \tcd{im2col} transformation
  is a data layout optimization
  which enables computing \ctd
  on matrix multiplication hardware.
The transformation
  involves instantiating
  the convolutional windows
  over the input activations
  directly in memory~\cite{im2col}.
This leads to data duplication,
  but the resulting speedup
  more than offsets that overhead.
%  but the time and space overhead of duplication
%  is often negligible
  %compared to the speedup that running on 
  %matrix multiplication hardware
  %provides.
In this case study,
  we show how a few
  general rewrites
  within \g
  lead to the 
  \textit{automatic rederivation}
  of the 
  \tcd{im2col} transformation.

%As  discussed
  %in section 
  %\ref{section:kernel-implementation-discussion},
  %\g exposes
  %striking similarities
  %in the structure
  %of \ctd
  %and matrix multiplication.
%We can also see this similarity
  %reflected in the fact that
  %\ctd
  %is similar
  %to the left-hand side
  %of our systolic array rewrite
  %in Figure \ref{fig:systolic-array-rewrite};
  %in fact, we can see 
  %the \tcd{(compute dotProd (cartProd ...))}
  %within \ctd.
\g's representation underscores
  the structural similarity
  between \ctd and matrix multiplication,
  reflected also by the shared
  \tcd{(compute dotProd (cartProd ...))}
  between \ctd 
  and the LHS of the systolic array rewrite
  in Figure~\ref{fig:systolic-array-rewrite}.
Using this rewrite 
  on \ctd would permit mapping it to the systolic array; 
  however, 
%But when we attempt
  %to apply this rewrite
  %to our current implementation
  %of \ctd,
  %we find that 
  the restrictions
  on the shape of
  \texttt{?a0}
  and \texttt{?a1}
  prevent its application.
The systolic array has 
  an activation access pattern
  of shape \accesspatternshape
  {a}{b}
  and a weight access pattern
  of shape \accesspatternshape
  {c}{d},
  while \ctd operates over
  access patterns
  of shape \accesspatternshape
  {N, 1,H',W'}{C, K_h, K_w}
  and
  of \accesspatternshape
  {O}{C, K_h, K_w},
  respectively.
%We described our systolic array
  %as reading lists of vectors
  %as inputs:
  %that is, an activations access pattern
  %of shape \accesspatternshape
  %{a}{b}
  %and a weight access pattern
  %of shape \accesspatternshape
  %{c}{d},
  %while the \tcd{cartProd}
  %in
  %\ctd is operating over
  %higher-dimensional access patterns
  %of shape \accesspatternshape
  %{N, 1,H',W'}{C, K_h, K_w}
  %and
  %of \accesspatternshape
  %{O}{C, K_h, K_w},
  %respectively.
Transforming the access pattern
  into a lower-dimensional form
  would enable the systolic array rewrite.
  
Figure~\ref{fig:im2col-rewrites}
  shows the rewrites
  which enable this transformation.
We call the first rewrite
  an 
  \textit{exploratory} rewrite
  as it
  optimistically matches
  any access pattern expression.
It flattens 
  an access pattern
  and immediately reshapes it
  back to its original shape, thus preserving equality
  (see Table~\ref{tab:access-pattern-transformers}
  for formal definitions).
This exploratory rewrite introduces the flattening
  necessary
  to convert the higher-dimensional access patterns
  of \ctd
  into the access patterns
  matched by the systolic array rewrite.
However, the \tcd{reshape} operator
  will still need to be moved 
  before we can fire 
  Figure~\ref{fig:systolic-array-rewrite}'s
  systolic array rewrite.
The second and third rewrites
  in Figure~\ref{fig:im2col-rewrites}
  take care of this;
  they implement \textit{composition commutativity}
  of \tcd{reshape}
  with \tcd{cartProd} and \tcd{compute dotProd},
  which ``bubble'' \tcd{reshape} operators
  up and out of expressions.
These rewrites
  express general properties of these operators
  and are not specific
  to this task.
  
These three rewrites work in concert 
  to map \ctd
  to a systolic array.
First,\footnote{
Since equality saturation 
  explores rewrites non-destructively, 
  the rewriting order here
  is purely for explanatory purposes.
}
  the exploratory rewrite
  flattens and reshapes
  all access pattern expressions.
This includes the inputs
  to \ctd's \tcd{cartProd}
  subexpression,
  which are flattened
  to shapes
  \accesspatternshape
  {N \cdot 1 \cdot H' \cdot W'}{C \cdot K_h \cdot K_w}
  and
  \accesspatternshape
  {O}{C \cdot K_h \cdot K_w}
  and reshaped
  back to their original shapes.
Next,
  the composition commutativity rewrites
  for \tcd{cartProd} 
  and
  \tcd{compute dotProd}
  fire in sequence,
  bubbling the \tcd{reshape} up
  through the 
  \tcd{cartProd} 
  and \tcd{dotProd} 
  expressions (shown in Figure~\ref{fig:conv2d-im2col-rewritten}).
Finally,
  the systolic array rewrite
  completes the \tcd{im2col} transform.
\g's equality saturation based rewrite engine
  discovers these rewrites
  as the exploratory rewrite 
  fires on every term
  and no rewrites are missed
  due to the absence of phase ordering.

This example highlights how,
  \textit{with straightforward,
  generally applicable rewrites
  defined over \g},
  equality saturation
  can emergently discover useful transformations
  that previously required
  expert insights to apply.
  %we can 
  %leverage the power of equality saturation
  %to produce interesting emergent transformations.

\subsection{Flexible Mapping: \tcd{matMul} Blocking}
\label{sec:case-study-blocking}

%\begin{figure}
%\begin{lstlisting}[escapechar=!]
%?a !$\Longrightarrow$! (concat (slice ?a ?dim ?b0 ?b1)
%               (slice ?a ?dim ?b1 ?b2) ?dim)
%\end{lstlisting}
%\caption{
%Exploratory slice--concatenate rewrite.
%}
%\label{fig:slice-concat}
%\end{figure}

%\begin{figure*}
%\begin{lstlisting}[escapechar=!]
%                                  ?a !$\Longrightarrow$! (concat (slice ?a ?dim ?b0 ?b1) (slice ?a ?dim ?b1 ?b2) ?dim)
% (cartProd (concat ?a0 ?a1 ?dim) ?b) !$\Longrightarrow$! (concat (cartProd ?a0 ?b)           
%                                                 (cartProd ?a1 ?b) ?newDim)        if ?dim is an access dimension
%
%\end{lstlisting}
%\caption{Rewrites used in the matrix multiplication blocking case study (\autoref{sec:case-study-blocking}).}
%\label{fig:blocking-rewrites}
%\end{figure*}

\begin{figure}
\begin{lstlisting}[escapechar=!]
?a !$\Longrightarrow$! (concat (slice ?a ?dim ?b0 ?b1) (slice ?a ?dim ?b1 ?b2) ?dim)
               
(cartProd ?a (concat ?b0 ?b1 ?dim)) 
  !$\Longrightarrow$! (concat (cartProd ?a ?b0) (cartProd ?a ?b1) ?newDim)
!$\textrm{if \texttt{?dim} is an access dimension}$!
 
(cartProd (concat ?a0 ?a1 ?dim0) (concat ?a2 ?a3 ?dim1)) 
  !$\Longrightarrow$! (concat (cartProd ?a0 ?a2) (cartProd ?a1 ?a3) ?newDim)
!$\textrm{if \texttt{?dim0} and \texttt{?dim1} are the same shape dimension}$!

(compute dotProd (concat ?a0 ?a1 ?dim)) 
  !$\Longrightarrow$! (concat (compute dotProd ?a0) (compute dotProd ?a1) ?dim)
!$\textrm{if \texttt{?dim} is an access dimension}$!

(compute dotProd (concat ?a0 ?a1 ?dim)) 
  !$\Longrightarrow$! (compute reduceSum (pair (compute dotProd ?a0) (compute dotProd ?a1)))
!$\textrm{if \texttt{?dim} is a shape dimension}$!
\end{lstlisting}
\vspace{-1em}
\caption{
Rewrites for blocking \tcd{matMul}.
%blocking case study (\autoref{sec:case-study-blocking}).
}
\label{fig:all-blocking-rewrites}
\end{figure}

\begin{figure}
%!\color{gray}; \hspace{2mm}\accesspatternshape{M, O}{}!
\begin{lstlisting}[escapechar=!]
(concat                          !\color{gray}; \hspace{2mm}\accesspatternshape{32,32}{}!
 (concat                         !\color{gray}; \hspace{2mm}\accesspatternshape{16,32}{}!
  (compute reduceSum             !\color{gray}; \hspace{2mm}\accesspatternshape{16,16}{}!
   (pair                         !\color{gray}; \hspace{2mm}\accesspatternshape{16,16}{2}!
    (compute dotProd             !\color{gray}; \hspace{2mm}\accesspatternshape{16,16}{}!
     (cartProd                   !\color{gray}; \hspace{2mm}\accesspatternshape{16,16}{2, 16}!
      (slice                     !\color{gray}; \hspace{2mm}\accesspatternshape{16}{16}!
       (slice (access activations 1) 0 0 16) 1 0 16)
      (transpose                 !\color{gray}; \hspace{2mm}\accesspatternshape{16}{16}!
       (slice 
        (slice (access weights 1) 0 0 16) 1 0 16)
       (list 1 0))))
    (compute dotProd             !\color{gray}; \hspace{2mm}\accesspatternshape{16,16}{}!
     (cartProd                   !\color{gray}; \hspace{2mm}\accesspatternshape{16,16}{2, 16}!
      (slice                     !\color{gray}; \hspace{2mm}\accesspatternshape{16}{16}!
       (slice (access activations 1) 0 16 32) 1 0 16)
      (transpose                 !\color{gray}; \hspace{2mm}\accesspatternshape{16}{16}!
       (slice                    !\color{gray}; \hspace{2mm}\accesspatternshape{16}{16}!
        (slice (access weights 1) 0 16 32) 1 0 16)
       (list 1 0)))))) ...
  \end{lstlisting}
  \vspace{-2em}
  \caption{A $32\times32$ \texttt{matMul}
  blocked into $16\times16$ \texttt{matMul}s.
  % via equality saturation per the rewrites in Figure
  % \ref{fig:all-blocking-rewrites}.
  % %\ref{fig:slice-concat} and %\ref{fig:concat-commutativity-rewrites}.
  Only two of the eight total multiplications are shown.
  }
  \label{fig:matmul-rewritten}
  \vspace{-1em}
\end{figure}

Equality saturation 
  can also be used with \g 
  to emergently discover a
  matrix multiplication
  blocking scheme.
%This case study will follow largely the same pattern
  %as \ref{sec:discovering-im2col}:
  %we will introduce an exploratory rewrite
  %and some associated ``cleanup'' rewrites,
  %and together they will produce the transformation
  %we are searching for.
Matrix multiplication blocking
  is the common strategy
  of breaking up a single, large
  matrix multiplication
%  (or a 2D convolution
    %converted to a matrix multiplication)
  into smaller multiplications,
  by multiplying subsets
  of the input matrices
  and assembling the results
  to form the output matrix.
This is essential in practice,
  as systolic arrays are small
  (often between $16\times16$ and $256\times256$)
  while matrices in ML and HPC applications
  can be much larger.
%Blocking is essential,
  %as, in practice,
  %systolic arrays are generally
  %somewhere between
  %$16\times16$ and $256\times256$ in shape,
  %while matrix multiplications
  %and 2D convolutions
  %may be larger. % any possible cite?
%Indeed, many accelerators 
  %specifically facilitate blocking
  %with accumulator registers
  %for storing intermediate results.
%Blocking is so common
  %that matrix multiplication hardware
  %is generally fitted
  %with \textit{accumulators}:
  %buffers built to store
  %the temporary partial results
  %produced in the middle of a blocked
  %matrix multiplication.

As in Section~\ref{sec:discovering-im2col},
  this transformation follows
  from an exploratory rewrite
  and associated ``cleanup'' rewrites.
The exploratory rewrite used for blocking
  is shown at the top of Figure~\ref{fig:all-blocking-rewrites}.
Given an access pattern,
  this rewrite slices the access pattern
  into two pieces
  along a dimension
  and then concatenates them back together.
The dimension
  as well as the division strategy
  are configurable.
For this example,
  we assume for simplicity
  that we run this rewrite
  on every available dimension,
  that we divide each dimension
  perfectly in half,
  and that all dimensions are powers of 2 in size.
Figure~\ref{fig:all-blocking-rewrites} gives rewrites
  for bubbling the introduced
  \texttt{concat}
  operators up through the expression,
  namely the compositional commutativity
  of \tcd{concat}
  with \tcd{cartProd}
  and \tcd{compute dotProd}.
Starting from the matrix multiplication
  in Figure \ref{fig:mat-mat-mult},
  assuming input shapes of $(32,32)$,
  the exploratory rewrite first slices and concatenates
  the access patterns
  at the input of \tcd{cartProd}.
Then, using the commutativity rewrites,
  the resulting \tcd{concat}s
  are bubbled up
  to produce the final expression
  in Figure~\ref{fig:matmul-rewritten}.
The effect of these rewrites
  is that the single 
  $32\times 32$ \tcd{matMul}
  becomes eight separate $16\times 16$ \tcd{matMul}s,
  which are
  summed
  and concatenated
  to form the full output matrix.
This case study demonstrates
  yet again
  that \g's expressiveness
  allows a small set
  of rewrites
  to produce interesting and useful
  emergent transformations.
  
% From here, we can apply
%   our systolic array rewrite (\autoref{fig:systolic-array-rewrite}).

    
% %\section{Future Work and Conclusion}
% \section{Conclusion}
% \label{sec:conclusion}

% In this paper,
%   we proposed \textit{access patterns} as an
%   abstraction to enable equality saturation
%   style term rewriting for low-level
%   tensor program mapping to hardware accelerators.
% Crucially, access patterns support
%   specifying and composing
%   higher-order, arbitrary dimension
%   tensor operators without the need
%   for binding structures like anonymous functions
%   or index notation.
% We demonstrated the potential utility of
%   access patterns in the \g IR through
%   case studies showing how rewrites in \g
%   can automatically uncover common
%   layout transformations like \tcd{im2col}
%   used for accelerator mapping.
% We are excited for the community
%   to join in further exploring the potential
%   applications of access patterns and to
%   build additional optimizations on
%   \g's foundations.



\input{part1/evaluation/eval-accelerator}
\input{part1/evaluation/eval-application}
\input{part1/evaluation/tab-compilation}
\input{part1/evaluation/eval-compilation}
\input{part1/evaluation/eval-validation-op}
\input{part1/evaluation/eval-validation-app}
\input{part1/evaluation/eval-fpga}


% \subsection{Extensibility}
% How hard to add a new accelerator support.
