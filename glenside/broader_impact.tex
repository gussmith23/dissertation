\section{Broader Impact Statement}

The ability to develop effective compiler support for
  specialized hardware accelerators in ML,
  and HPC more broadly,
  has generally been restricted to a handful of
  elite, well-resourced teams.
This restriction slows hardware development
  and creates barriers to entry for teams in
  less privileged environments to contribute to
  and help guide the development of the field.

We believe that the access pattern abstraction
  and \g's approach to term rewriting for
  improving compiler support for custom
  accelerators will help advance both
  near-term practical and longer-term principled
  approaches to building flexible compiler infrastructure.
In turn, we hope that this infrastructure will
  help contribute to a broader, more diverse, and
  more inclusive community of folks working
  together to build efficient technologies for social
  good.
  %to improve our communities.
  %by developing technologies for social good.
  
Of course, all technology is political and it can
  be difficult to anticipate how future
  researchers and practitioners may apply \g.
While the most obvious consequence of more
  efficient hardware utilization is better
  performance for users and lower environmental
  impact via decreased power consumption,
  it is also possible that access patterns and \g
  would enable the rapid obsoleting of current
  hardware platforms and therefore contribute
  to harmful electronic waste.
This work could also stimulate
  demand for hardware customization by
  removing compiler development--related overheads and
  ultimately lead to higher negative
  environmental impact similar to the
  situation with respect to custom ASICs
  for bitcoin mining~\cite{qin2020bitcoins}.

Also,
  any improvement to ML efficiency or applicability
  may contribute to economic and privacy concerns
  arising from increased technology company monopolization
  as discussed in Zuboff's
  \textit{The Age of Surveillance Capitalism}~\citep{surveillance}.
