\subsection{System Deployment and FPGA Emulation}
\label{sec.eval-fpga}

As an additional demonstration of {\TLA}, we explored its use in 
%To demonstrate that the {\TLA} flow also supports 
  compiling workloads
  to a real hardware platform.
  Specifically, 
  we used our prototype to compile workloads
  to an FPGA emulation of FlexASR.%
\footnote{We synthesized and placed-and-routed the FlexASR accelerator on a Xilinx Zynq ZCU102 FPGA, which consumed 86\% of the available LUT resources.
Due to the significant engineering overhead of FPGA emulation, FlexASR is the only accelerator we deployed on an FPGA.}
%We synthesized and placed-and-routed the FlexASR accelerator on a Xilinx Zynq ZCU102 FPGA, which consumed 86\% of the available LUT resources.%
%
%The approach was conceptually simple:
We configured our prototype to lower
 FlexASR ILA instructions
 to the corresponding MMIO commands for FlexASR, % (a trivial translation thanks to the one-to-one correspondence between the ILA instructions and MMIO commands),
 passing them to the FPGA using the Xilinx SDK~\cite{xsdk}.
%We utilized the Xilinx SDK~\cite{xsdk} to pass the accelerator instructions (MMIO commands) to the accelerator interface for invoking the supported operations.
%
%Specifically, 
Next, we compiled and executed synthetic workloads in which LSTM layers and linear layers were offloaded to the FlexASR accelerator.
The results matched those of the ILAng-generated simulator bit for bit, providing validation for %the software simulation of 
the custom numerics.
%\hl{AG: what is the punchline -- was this successful? how much effort?}
%With only light modifications to our previous FlexASR code generator, we were able to convert our simulated trials into executions on the FPGA.\mh{same as the previous comment: do we expand here on what to do specifically?}
%Although only a proof of concept, this case study demonstrates that the {\TLA} methodology can be potentially extended to a complete compilation flow. 
This is a proof of concept for utilizing the {\TLA} methodology for an actual deployment, above and beyond simulation-based testing.
  %and not just end-to-end testing.
  %provides a principled and extensible means
  %of compiling applications to new accelerators 
  %\hl{AG: seems meh}.
%This case study demonstrates the applicability of \TLA in actual system deployment on a commodity hardware platform.
%
%Further, it shows that in the absence of compilation-results validation enabled by the \TLA methodology, the need for a fully functional RTL model and the significant engineering overhead indeed limit early-stage software/hardware co-design.
