
\begin{lemma}
    \label{lemma:non-termination-of-rule-implies-recursive-call}
    If \textsc{Interp} fails to execute
    on node $n$, then this non-termination
    must be due to a recursive call to 
    \textsc{Interp}.
\end{lemma}
\begin{proof}
Suppose $\textsc{Interp}(p,e,t,n)$
    fails to terminate.
$n$ is one of five types of nodes:
    $\SynBv$,
    $\SynVar$,
    $\Op$, 
    \IRReg,
    and \IRPrim;
    \textsc{Interp} is not defined for holes
    $\blacksquare_{x_i}$.
    
Interpreting a $\SynBv$ node
    returns a bitvector constant,
    and this terminates.

Interpreting a $\SynVar$ node
    performs an environment
    lookup.
An environment is formed either by an
    explicit mapping of
    variables to concrete streams,
    or by the lambda abstraction
    in the rule for primitives.
Looking up a concrete mapping always
    terminates.
Executing the body of lambda abstraction
    formed in the primitives rule
    involves a single recursive call to \textsc{Interp},
    so if this abstraction fails to terminate
    it must be from a non-terminating recursive call.
    
Interpreting a \IRReg{} node when time $t=0$
    returns a bitvector constant, which terminates.
Interpreting a \IRReg{} node when time $t > 0$
    involves a single recursive call to \textsc{Interp};
    thus if execution fails to terminate then this
    recursive call must fail to terminate.

Interpreting an $\Op$ node performs a
    map of recursive calls to \textsc{Interp}
    to a finite list of operands,
    and then applies the semantics 
    of $\llbracket op \rrbracket$
    to the result.
Mapping function $f$ over a finite list $xs$ always
    terminates whenever $f$ is guaranteed to
    terminate, so for our map invocation to
    diverge, one of the recursive
    calls to \textsc{Interp}
    must fail to terminate.
Computing the operator semantics
    $\llbracket op \rrbracket$
    could fail to terminate in theory,
    but in practice these operators are
    simple bitvector operations
    and always terminate (e.g., bitvector addition).
    
Interpreting a \IRPrim{} node
    performs lambda abstraction and then
    passes the abstraction to a recursive
    call to \textsc{Interp}.
Lambda abstraction always terminates.
Thus, if interpreting a primitive node
    diverges this must be due to the
    recursive call failing to terminate.

This, if $\textsc{Interp}(p,e,t,n)$ fails to
    terminate, this must be due to a non-terminating recursive
    call to \textsc{Interp}, as was to be shown.
\end{proof}

We define a \textit{recursive call sequence}
    to be (possibly infinite) sequence of \textsc{Interp} executions 
    $\{\mathcal{I}_i\}$,
    where  $\mathcal{I}_i = \textsc{Interp}(p_i, e_i, t_i, n_i)$
    for some $p_i$, $e_i$, $t_i$ and $n_i$,
    such that $\mathcal{I}_i$ recursively invokes $\mathcal{I}_{i+1}$,
    such that the final execution $\mathcal{I}_f$ in the sequence,
    if it exists,
    does not recursively call \textsc{Interp}.
We say that a recursive call sequence $\{\mathcal{I}_i\}$ is
    \textit{divergent} if all $\mathcal{I}_i$ are divergent.
\begin{lemma}
    \label{lemma:non-termination-implies-infinite-recursion}
    A recursive call sequence of finite length cannot be divergent.
\end{lemma}
\begin{proof}
Let $\{\mathcal{I}_i\}$, $i = 1\ldots L$, be a finite length
    recursive call sequence.
By definition the final execution $\mathcal{I}_L$
    must not recursively call \textsc{Interp},
    and by Lemma~\ref{lemma:non-termination-of-rule-implies-recursive-call}
    $\mathcal{I}_L$ must terminate,
    and $\{\mathcal{I}_i\}$ is not divergent.
\end{proof}

\begin{lemma}
    If executing $\textsc{Interp}(p, e, t, n)$ diverges,
    then there must exist an infinite recursive call sequence
    $\{\mathcal{I}_i\}$ 
    starting with $\mathcal{I}_1 = \textsc{Interp}(p, e, t, n)$.
\end{lemma}
\begin{proof}
We prove by contradiction.
Suppose executing $\textsc{Interp}(p, e, t, n)$ diverges,
    and suppose that all recursive call sequences
    starting with $\mathcal{I}_1 = \textsc{Interp}(p, e, t, n)$ do
    not diverge.
An infinite length recursive
    call sequence corresponds to
    infinite recursion which is divergent,
    and so all recursive
    call sequences starting with $\mathcal{I}_1$
    must have finite length.

\end{proof}


\begin{lemma}
    \label{lemma:infinite-recursion-impossible}
    If program $p$ is well-formed, then
    $\textsc{Interp}\ p\ e\ t\ p.root$
    cannot recurse infinitely.
\end{lemma}
\begin{proof}
We prove by contradiction.
Suppose 
    $\textsc{Interp}(p, e, t, r)$ diverges.
Let $T$ be the execution tree
    defined by the divergent execution
    of $\textsc{Interp}(p, e, t, r)$,
    whose (1) tree nodes are the 
    individual executions of \textsc{Interp}
    on the program nodes of $p$,
    and (2) edges between
    tree nodes $N_i$ and $N_j$ are
    recursive invocations of \textsc{Interp}.
Since \textsc{Interp} can execute
    on a single program node $n_i$ multiple times,
    a program node $n_i$ can correspond to
    multiple tree nodes $N_{n_i, 1}, N_{n_i, 2}, \ldots$,
    one for each time \textsc{Interp}
    executes on $n_i$.
From the preceding arguments, the execution of each
    tree node is finite.
Since execution diverges, 
    $T$ must be \textit{infinite}
    and in particular have
    infinitely many tree nodes.

Each rule for \textsc{Interp}
    directly invokes \textsc{Interp} finitely
    many times, so each tree node has
    finitely many children.
This implies that for $T$
    to be infinite, it
    must have infinite depth,
    and it follows that
    there exists
    an infinite path
    $P$ through $T$.
Since our program $p$ has finitely many nodes,
    there must be program node $n$
    that is executed infinitely many
    times in path $P$ via nodes
    $N_{n,1}, N_{n,2}, \ldots$
This and well-formedness rule
    \textbf{W6} imply the existence
    of infinitely many tree nodes
    $N_{r,1}, N_{r,2}, \ldots$
    in $P$ corresponding to the
    execution of register program nodes.
There are two cases: $n$ itself is a
    register node, or
    $n$ is not a register node.
If $n$ is a register node, then
    our claim trivially holds:
    it exists within infinitely many tree
    nodes $N_{n,1}, N_{n,2}, \ldots$
    in $P$, and we are done.
Otherwise, $n$ is not a register node,
    and $n$ has a self-dependency.
By \textbf{W6}, program $p$ cannot have combinational
    cycles, meaning self-dependencies
    must be broken by a register
    node, and each subpath in $P$
    starting and ending with an execution of $n$
    must contain the execution of a register.

But a path through an execution tree
    cannot contain infinitely many register nodes:
    each execution of a register node decrements
    time, and no rules in \textsc{Interp} ever
    increment a time value.
Since time at the top of the path must
    be some natural number $t_P$, the path
    contains at most $t_P + 1$ register nodes.
This contradicts that $P$ has infinitely many
    register nodes: $P$ must be finite,
    $T$ must be finite,
    and $\textsc{Interp}(p, e, t, r)$
    must terminate when $p$ is free
    of combinational cycles.
\end{proof}


{\color{green}
THE FOLLOWING IS AN OLD PROOF THAT WE CAN RECYCLE, ARGUING THAT ROSETTE TERMINATES. WE DONT WANT TO MAKE IT A THEOREM, SO INSTEAD THIS WILL BE A PROSE ARGUMENT
\begin{proof}
Rosette searches through
  all possible ways of
  filling in a sketch's holes
  and uses \textsc{Interp} to
  check if the resulting program
  is equivalent to the behavioral
  specification.
There are finitely many holes,
  and each hole has
  finitely many possible completions.
Thus there are finitely many ways to
    fill the holes of a sketch,
    and as long as \textsc{Interp} always
    terminates, Rosette's synthesis query
    is guaranteed to terminate, and the
    entire search space can in theory
    be explored.

To show that \textsc{Interp} is guaranteed
    to terminate on well-formed programs,
    we show two intermediate results,
    proved below: 
    %
    (i) non-termination of \textsc{Interp}
    implies infinite recursion, 
    (Lemma \ref{lemma:non-termination-implies-infinite-recursion}),
    %
    and (ii) infinite recursion is impossible on
    a well-formed program
    (Lemma \ref{lemma:infinite-recursion-impossible}).
These two facts 
    combine to imply that \textsc{Interp}
    is guaranteed to terminate on well-formed
    programs.
\end{proof}
}