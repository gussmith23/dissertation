\section{Formalization}
\label{sec:formalization}

% These are necessary so that we can
% use listings to typeset Reg/Prim/etc
% as code within the semantics.
\newsavebox\boxlet
\newsavebox\boxassign
\newsavebox\boxin
\newsavebox\boxreg
\newsavebox\boxprim
\savebox{\boxlet}{\lstinline[language=thelang]!let!}
\savebox{\boxassign}{\lstinline[language=thelang]!:=!}
\savebox{\boxin}{\lstinline[language=thelang]!in!}
\savebox{\boxreg}{\lstinline[language=thelang]!Reg!}
\savebox{\boxprim}{\lstinline[language=thelang]!Prim!}

\newcommand{\Prim}[0]{\lstinline[language=thelang]{Prim}\xspace}

\newcommand{\Reg}[0]{\lstinline[language=thelang]{Reg}\xspace}

\newcommand{\Let}[0]{\lstinline[language=thelang]{let}\xspace}


\begin{figure*}

%%%%%%%%%%%%%%%%%%%% Lakeroad DSL Grammar %%%%%%%%%%%%%%%%%%%%
\begin{minipage}{.35\textwidth}
%%%%%%%%%%%%%%%%%%%%%%%%%%%%%%%%%% 1st Column %%%%%%%%%%%%%%%%%%%%%%%%%%%%%%%%%%
%\footnotesize{}
\begin{tabular}{l@{\hspace{.5em}}l}
$\SynProg$     & $\defn$ $\langle$ $\SynId$, $\seq{\SynId, \SynNode}*\rangle$\\[0.5em]
$\SynNode$       & $\defn \SynBv\ b$  | $\SynVar\ x$ \\
                 & \ \   | $\Op\ op\ \SynId$* \\
                 & \ \   | \IRReg{} $\SynId$ $(\SynBv~b)$ \\
                 & \ \   | \IRPrim{} $\mathsf{binds}$ $\SynProg$ \\
                 & \ \   | $\blacksquare_x$ \\[0.5em]
    
% $e\ \defn$       & $\seq{i, \hat{e}}$ \\[1em]
%
% $\hat{e}\ \defn$ & 
%   $b \vbar x \vbar$ 
%   \lstinline[language=thelang,mathescape]!let $x$ := $e$ in $e$! $\vbar$ 
% $o\ e^{*} \vbar$ \\
%   & \lstinline[language=thelang,mathescape]!Reg $e$ $e$ $b$! $\vbar$ 
%     \lstinline[language=thelang,mathescape]!Prim $x^{*}$ $e^{*}$ $e$! $\vbar$ 
% $\blacksquare_x$
\end{tabular}
\end{minipage}%
%%%%%%%%%%%%%%%%%%%%%%%%%%%%%%%%%% 2nd Column %%%%%%%%%%%%%%%%%%%%%%%%%%%%%%%%%%
\begin{minipage}{.25\textwidth}
\footnotesize{}
\begin{tabular}{l l}
$\SynId$  & $id  \in \mathbb{N}$ \\  
Bitvectors  & $b  \in \mathbb{BV}$ \\
Variables & $x  \in \text{LegalVarNames}$\\[0.5em]
Operators     & $op \in$ $\OpBv  \cup  \OpWire$\\[0.5em]
$\mathsf{binds}$ & $bs \in (\text{Variables} \rightharpoonup \SynId)$
\end{tabular}
\end{minipage}
%%%%%%%%%%%%%%%%%%%%%%%%%%%%%%%%%% 3rd Column %%%%%%%%%%%%%%%%%%%%%%%%%%%%%%%%%%
\begin{minipage}{.4\textwidth}
\footnotesize{}
\begin{tabular}{l l}
\;\;\; Wire op & $\OpWire =$ \lstinline[language=thelang,mathescape]!$\{$concat$,$extract$, \ldots \}$! \\
\;\;\; Non-wire op & $\OpBv =$ \lstinline[language=thelang,mathescape]!$\{+, -, \times, \ldots \}$! \\
\end{tabular}
\end{minipage}


%%%%%%%%%%%%%%%%%%%% END LR DSL GRAMMAR %%%%%%%%%%%%%%%%%%%%
\caption{Syntax of $\UberLang$. $\blacksquare_x$ is a syntactic hole, labeled with variable $x$. $A \rightharpoonup B$ denotes the set of partial functions from $A$ to $B$.}
\label{fig:syntax}
\end{figure*}

% \begin{figure*}
% \centering
$\textsc{BV} \inferrule {P[i] = v}
                            {\smallStep{P}{i}{E}{t}{v}}$ \quad 
%
$\textsc{Var} 
\inferrule {P[i] = x \quad E[x][t] = v } 
{ \smallStep{P}{i}{E}{t}{v} }$ \quad\\[1em]
%
$\textsc{Op} 
\inferrule { 
o \in \Op \quad
P[i] = o\ i_1\ \ldots\ i_n \quad
\smallStep{P}{i_1}{E}{t}{v_1} \quad
\ldots \quad
\smallStep{P}{i_k}{E}{t}{v_n} \quad    % INPUT n
\quad 
v = \llbracket o \rrbracket(v_1, \ldots, v_n)}
{ \smallStep{P}{i}{E}{t}{v} }$ \\[1em]

$\textsc{RegZero}
\inferrule{
  P[i] =\ $\IRReg$\ i_{clk}\ i_{data}\ v_{init}
}{\smallStep{P}{i}{E}{0}{v_{init}}}$\\[1em]

$\textsc{RegHold} 
\inferrule {
  t > 0 \quad
  P[i] =\ $\IRReg$\ i_{clk}\ i_{data}\ v_{init} \quad
  \smallStep{P}{i_{clk}}{E}{t}{v_{clk}} \quad
  \smallStep{P}{i_{clk}}{E}{t-1}{v_{oldClk}} \quad
  \smallStep{P}{i}{E}{t-1}{v_{oldQ}} \quad
  (v_{oldClk}, v_{clk}) \neq (0, 1)
  } 
{ \smallStep{P}{i}{E}{t}{v_{oldQ}} }$ \\[1em]

$\textsc{RegTick} 
\inferrule {
  t > 0 \quad
  P[i] =\ $\IRReg$\ i_{clk}\ i_{data}\ v_{init} \quad
  \smallStep{P}{i_{clk}}{E}{t}{v_{clk}} \quad
  \smallStep{P}{i_{clk}}{E}{t-1}{v_{oldClk}} \quad
  \smallStep{P}{i_{data}}{E}{t-1}{v_{oldData}} \quad
  (v_{oldClk}, v_{clk}) = (0, 1)
  } 
{ \smallStep{P}{i}{E}{t}{v_{oldData}} }$ \\[1em]

% 2023/10/21: Gus added a "hack" that might fix the prim problem of not putting streams in the environment, but we still may wanna do away with these inference rules altogether in favor of the interpreter pseudocode
$\textsc{Prim} 
\inferrule {
  P[i] = \SynPrim\ \langle x_1,\ldots, x_n\rangle\
                   \langle i_1, \ldots, i_n\rangle\
                   p \quad
  % \smallStep{P}{i_1}{E}{t}{v_1} \quad 
  % \ldots \quad 
  % \smallStep{P}{i_n}{E}{t}{v_n} \quad 
  1 \leq j \leq n\quad 
  \smallStep{P}{i_j}{E}{0}{v_j^0} \quad 
  \cdots \quad
  \smallStep{P}{i_j}{E}{t}{v_j^t} \quad 
  E^\prime = E\left[x_j \mapsto \left[v_j^0, \dots, v_j^t\right]\right] \quad
  \smallStep{p}{p.root}{E^\prime}{t}{v}
} 
{ \smallStep{P}{i}{E}{t}{v} }$ \\[1em]

$\textsc{Step}
\inferrule {
\smallStep{P}{P.root}{E}{t}{v}
}{\bigStep{P}{E}{t}{v}}
$

% \separatingline \vspace{1em}

% $\textsc{OneCycle} \inferrule { \evalsTo{e}{E}{S}{v}{S'} } { \multiEvalsTo{[E]}{e}{S}{v}{S'} }$ \quad
% $\textsc{MultiCycle} \inferrule { \evalsTo{e}{E_1}{S}{v_{ignore}}{S'} \quad 
% \multiEvalsTo{[E_2, \ldots, E_n]}{e}{S'}{v}{S''} } { \multiEvalsTo{[E_1, E_2, \ldots, E_n]}{e}{S}{v}{S''} }$ \\[1em]
% \caption{Semantics of Core \UberLang.
%    Note that we do not give semantics for $\blacksquare_x$
%    since these holes will be filled prior to interpretation.}

% \label{fig:semantics}
% \end{figure*}

%\sandy{Does the following mean that LR has "with" functions, as grammar suggests, or simply "functions"?}
We now formalize \lr{} 
  with functions $\lrfn$ and
  $\lrfnbmc$,
  and use these models
  to argue for the correctness
  and partial completeness of \lr{}. 
We first define  
  $\lrfn$ (\cref{subsec:the-lr-function}) and then 
motivate and define
  the language $\UberLang{}$,  
  specify its syntax and semantics,
  and define behavioral ($\SpecLang$), structural ($\ImplLang$), and sketch ($\SketchLang$) sublanguages (\cref{subsec:lrir-syntax-and-semantics}).
%
{% \color{blue}
We next explain the
  underlying queries
  \lr{} uses to
  synthesize hardware programs
  that meet the desired specification
} (\cref{sec:formalization-program-synthesis}).
  We demonstrate the correctness
  and partial completeness of $\lrfn$,
  enumerate our Trusted Computing Base
  (\cref{subsec:lr-correctness-and-completeness}) and 
extend $\lrfn$ to $\lrfnbmc$,
    which ensures the generated program's
    semantics matches the design over multiple
    timesteps (\cref{subsec:lrfn-bmc}).
%
Finally,
 we highlight potential future
 applications that could be
 built on this section's formalization
 (\cref{sec:sem-beyond}).
 

\subsection{The \lr Function $\lrfn$}
\label{subsec:the-lr-function}

We model the execution of \lr
  with the partial function
  \small
%
\[\lrfn: \Sketch \times \SpecLang \times \Time \rightharpoonup \ImplLang,\]
%
\normalsize
where $\lrfn(\Psi, d, t)$
  invokes Rosette
  to synthesize a $t$-cycle
  implementation
  of behavioral design $d$ 
  using sketch $\Psi$ to guide
  the search,
  where a $t$-cycle
  implementation of $d$ is a program
  that is equivalent to $d$ at clock cycle
  $t$.
By not requiring program equivalence before
    clock cycle $t$ we allow the
    synthesized program's semantics to
    differ from the design during 
    an initialization period
    (e.g., as the pipeline is being filled).
To get guarantees beyond a single
    point in time $t$, we generalize
    $\lrfn$ to $\lrfnbmc$,
    which synthesizes a program
    that is equivalent to the design
    from time $t$ to $t + n$.
We formalize a sketch $\Psi \in \Sketch$  as a tuple
  $(\psi, h)$,
  where $\psi$ is a program in $\SketchLang$ 
  and $h$ is a map
  from the holes in $\psi$
  to a finite set of valid hole-free
  nodes in $\ImplLang$
  that can be used to fill
  the mapped hole.
This mapping $h$ is handled implicitly by Rosette's
  \texttt{choose} and \texttt{hole} constructs and
  need not be explicitly specified by the
  sketch writer.
We write $\lrfn(\Psi, d, t) = p$ to 
    indicate that synthesis succeeded and produced
    \lr{} program $p$.
However, it is possible that sketch $\Psi$
    cannot implement $d$, in which case
    the synthesis fails
    (i.e., returns UNSAT)
    and
    $\lrfn$ does not return anything.
Design $d$ belongs to 
  $\UberLang$'s behavioral fragment,
  $\SpecLang$ (see
  \cref{subsec:lrir-syntax-and-semantics}).
%
When $t = 0$, $\lrfn$ 
  synthesizes a 
  \textit{combinational design}; when 
$t > 0$, $\lrfn$ 
  synthesizes a \textit{sequential design}
  over $t$ clock cycles.
The rest of this section 
  considers sequential design synthesis 
  since its combinational counterpart is 
  a special case covered
  by our general approach.
  
  
\subsection{Defining $\UberLang$}
\label{subsec:lrir-syntax-and-semantics}

\lr uses the 
  $\UberLang$ language to
  translate behavioral
  HDL programs
  to structural, 
  hardware-specific
  HDL programs.
To facilitate this
  translation, we
  designed $\UberLang$
  to satisfy the
  following properties:
\begin{enumerate}[label=\textbf{P\arabic*}.]
    \item \textit{Easy translation to/from HDLs:}
        we must be able
        to translate
        designs from 
        a behavioral HDL
        to $\UberLang$
        and translate
        synthesized implementations
        to a structural HDL.
        
    \item \textit{Support parallel stateful execution:}
        FPGA designs
        consist of
        potentially stateful elements
        executing in parallel.
        $\UberLang$ must
        allow unambiguous
        parallel execution.
        % (i.e., subexpression
        % evaluation order
        % does not matter).

    \item \textit{Support graph-like program structures:}
        An FPGA component's outputs
        can be wired to
        multiple other components,
        including back
        to itself.
        This means that 
        FPGA programs can
        form arbitrary
        graphs, and $\UberLang$ must
        be able to express this.
        % In practice,
        % cycles should
        % be \textit{sequential},
        % meaning every
        % cycle is
        % interrupted by a register.
        % Our language should
        % capture this restriction.
        
    \item \textit{Support for sequential designs:}
        $\UberLang$ must handle designs
        that run over multiple clock cycles.
        
    \item \textit{Support for different architectures:}
        $\UberLang$ must handle FPGA components
        from different architectures.
\end{enumerate}
%
We describe how $\UberLang$ satisfies
  P1-P5 when we 
  define its syntax and semantics
  in the following subsections. 


\subsubsection{$\UberLang$'s Syntax}
\label{subsubsec:syntax}

\Cref{fig:syntax} shows the $\UberLang$ syntax.
An $\UberLang$ program $\SynProg$
  consists of a root node ID
  and a graph of nodes,
  each of which is
  referred to by its ID.
A \textit{node} 
  can be 
  a constant bitvector,
  input variable,
  combinational (pure) operator,
  sequential (stateful) register,
  primitive,
  or hole.
Given a program 
  $p = (r, \langle id_1, {node}_1\rangle
           \ldots
           \langle id_n, {node}_n\rangle)$, 
  we use the notation 
  $p.root = r$, 
  $p.ids = \left\{id_1, \ldots, id_n\right\}$,
  and
  $p[id_i] = {node}_i$.
We define the free variables of a program $p.fv = \{x_i\}$ as the set of variable names occurring in $p$'s nodes of the form $(\SynVar\ x_i)$.\footnote{Note that this does not include variables of sub-programs occurring recursively inside of \IRPrim nodes.}  Finally, we use the notation $p.all\_ids$ for $p.ids$ together with $p'.all\_ids$ of any subprogram $p'$ of $p$ ($p'$ is a subprogram of $p$ if $\exists j, node_j = \textrm{\IRPrim}~bs~p'$).

%\hl{Given time, would be good to replace SynBv and SynVar with new IRBv and IRVar macros with appropriate typesetting}
Given a node $n$,
  we specify its inputs
  with the following function:
  \small
\begin{align*}
  %\hl{space per line}
  &\textsc{inputs}(\SynBv\ b) = \{\}, \\
  &\textsc{inputs}(\SynVar\ x) = \{\}, \\
  &\textsc{inputs}(\Op\ op\ i_1 \ldots i_n)
     = \{i_1,\, \ldots,\, i_n\} \\
  &\textsc{inputs}(\text{\IRReg}\ i\ b_{init})
     = \{i\} \\
  &\textsc{inputs}(\text{\IRPrim}\ 
                     bs\ p')
                = \{bs[x]\ |\ x \in \textrm{domain}(bs)\}
\end{align*}
\normalsize
Note that we use 
    $A \rightharpoonup B$ 
    to denote the set
    of partial functions
    from $A$ to $B$;
    given 
    $bs \in A \rightharpoonup B$, 
    we write $\textrm{domain}(bs)$
    to denote
    the set of 
    $x\in A$ s.t.
    $bs[x]$ is defined.

%\hl{Define free variables function and update W5 to use this.}

A program $p$ is well-formed
  if and only if
  all the following
  conditions hold:

% Well Formedness
\begin{enumerate}[label=\bfseries{W\arabic*}.]
  \item $p.root \in p.ids$;

  \item All ids are unique and distinct. (i.e. for any sub-program $p'$, $p.ids$ and $p'.all\_ids$ are disjoint, and for any two sub-programs $p'$ and $p''$, $p'.all\_ids$ is disjoint from $p''.all\_ids$.)
  
  \item The inputs of all nodes in $p$ are ids of other nodes in $p$:
     $\forall id \in p.ids$, 
     $\text{inputs}(p[id]) \subseteq p.ids$;

  \item All primitive nodes contain 
      well-formed programs;
  \item All primitive nodes bind exactly their free variables; i.e., for $\text{\IRPrim}\ bs\ p'$, $\textrm{domain}(bs) = p'.fv$; and
        
  \item Program $p$ is free of combinational loops (formalized below in \cref{property:free-of-combinational-loops}).
  % After expanding and inlining all primitive nodes
  %       in $p$, the resulting inlined program
  %       is free of combinational loops, meaning that 
  %       all \hl{self-dependencies} for any non-register node
  %       must be broken by a register.
        %
%         \footnote{
% Formally, a program $p$ is free of combinational
% loops if there exists a function
% $w : p.all\_ids \to \mathbb{N}$ 
% defined on the ids of $p$ and all 
% sub-programs that satisfies the following properties:
% (1) if $p[id] = \text{\IRReg{}}~\_~\_$, then $w(id) = 0$;
% (2) if $p[id] = \text{\IRPrim{}}~bs~p'$, 
% then $w(id) = w(p'.root)$ and $w(id') = w(bs[x])$ when $p'[id'] = Var~x$; 
% (3) otherwise, ($p[id] = OP~op~\_$) 
% $w(id) > w(id')$ for all $id' \in inputs(p[id])$.  
% The function $w$ acts as a witness to the 
% absence of combinational loops 
% because it is impossible to define a strictly monotonic function without acyclicity.
%         }
\end{enumerate}

\begin{property}[Free of Combinational Loops]
\label{property:free-of-combinational-loops}
Formally, a program $p$ is free of combinational
loops if there exists a function
$w : p.all\_ids \to \mathbb{N}$, that satisfies the following properties (collectively ``monotonicity''):
\begin{enumerate}
\item If $p[id] = \text{\IRReg{}}~\_~\_$, then $w(id) = 0$;
\item If $p[id] = \text{\IRPrim{}}~bs~p'$, then $w(id) > w(p'.root)$;
\item if $p[id] = \text{\IRPrim{}}~bs~p'$ and $p'[id'] = Var~x$, \\ then $w(id') > w(bs[x])$; and
\item Otherwise (e.g., $p[id] = \Op~op~ids^{*}$), \\
    if $id' \in \textsc{inputs}(p[id])$,
    then $w(id) > w(id')$. 
\end{enumerate}
\end{property}
\noindent The function $w$ acts as a witness to the 
absence of combinational loops because it is
impossible to define a strictly monotonic function without acyclicity.
We consider only well-formed
  $\UberLang$ programs.

% The following paragraph
% describes how each part
% of our syntax satisfies
% properties P1-P5.
% 

$\SynBv$, $\SynVar$, and $\Op$ nodes
  encode bitvectors, variables, and
  operators.

\Reg $i_{data}\ b_{init}$ nodes
  let \UberLang implement
  sequential designs (P4).
$i_{data}$ is the 
  register's data input,
  which updates the stored
  value at the positive edge
  of each clock cycle,
  and $b_{init}$ is the
  register's initialization value.

\IRPrim{} $bs\ p$ nodes
  let \UberLang programs
  use hardware-specific components
  from different architectures (P5).
The $bs$ component is a \textit{variable map},
  mapping $\SynVar$s to input $\SynId$s.
The $p$ component is an $\UberLang$ program
  that defines the semantics of the
  hardware primitive.
A \Prim node also carries some metadata 
  % describing the hardware primitive's name
  used during compilation
  to a structural
  HDL, which we omit 
  for clarity.\tighten

$\SpecLang$ is the concrete
  \textit{behavioral}
  fragment of $\UberLang$ used for
  writing specifications; it 
  is formed by
  excluding \Prim
  nodes and holes
  from $\UberLang$.
  
$\ImplLang$ is the concrete
  \textit{structural}
  fragment of $\UberLang$ used
  for lowering $\UberLang$
  to structural HDLs; it 
  is formed
  by excluding \Reg nodes, $\Op$ nodes,
  and holes from $\UberLang$,
  with the following
  exception:
  the $p$ term
  in $\usebox{\boxprim}\ bs\ p$ must
  always be from the $\SpecLang$ since it 
  is used to specify the semantics
  of the \Prim node
  to the synthesis engine.
The behavioral node $p$
  is not used during
  compilation to HDL,
  and this behavioral
  expression does not 
  propagate to the
  structural HDL output.

$\SketchLang$ is another sublanguage of $\UberLang$ that is $\ImplLang$ but also including holes. 
Let $s$ be a program in $\SketchLang$
  with holes 
  $\blacksquare_{x_1}, \ldots, \blacksquare_{x_k}$.
These holes can be \emph{filled}
  with nodes $n_1, \ldots, n_k$
  in $\ImplLang$ by replacing
  each hole $\blacksquare_{x_i}$ 
  with its 
  corresponding node $n_i$
  to obtain a complete
  $\ImplLang$ program,
  denoted by $s[\blacksquare_{x_1} \mapsto n_1, \ldots]$.

The simplicity of this syntax
  makes translating to and from
  HDLs straightforward (P1).
\cref{sec:implementation}
  describes how \lr{} implements the
  translations to and from HDLs.

\subsubsection{$\UberLang$'s Semantics}
\label{subsubsec:semantics}

\begin{figure}

% Vishal's version before gus's renamings
% let rec eval prog env rt t = 
%   match prog[t]:
%   | BV v -> v
%   | Var x -> env[x][t]
%   | Op (op, ids) ->
%      op (map (fun id -> eval prog env id t) ids)
%   | Reg (idC, idD, init) ->
%      if t = 0:
%       init
%      else:
%       let v_oldC = eval prog env idC (t - 1) in
%       let v_oldQ = eval prog env id (t - 1) in
%       let v_oldD = eval prog env idD (t - 1) in
%       let v_curC = eval prog env idC t in
%       if (v_oldC, v_curC) = (0, 1):
%         v_oldD
%       else:
%         v_oldQ
%   | Prim (varIdMap, primProg) -> 
%       let env' x t =
%         eval prog env varIdMap[x] t in
%       eval primProg env' primProg.rt t


% Types
% Prog: (id, id -> Node)
% Var = str
% Env: var -> Time -> Value
% Id = int
% Time = Nat
% Value = BV
% interp: Prog -> Env -> Id -> Time -> Value
% op argument of Op: [Value] -> Value
% var_map: Var -> Id
% body: Prog
\small
% \begin{minted}{ocaml}
% let rec interp prog env id t = 
%  (* Helper to recursively call interpreter. *)
%  let interp' = interp prog env in
%  let (_, nodes) = prog in
%  match nodes id with
%  | BV v -> v
%  | Var x -> env x t
%  | Op (op, ids) ->
%     op (map (fun id -> interp' id t) ids)
%  | Reg (id_d, init) -> 
%     if t = 0 then init else interp' id_d (t - 1)
%  | Prim (var_map, body) -> 
%     let env' x = interp' (var_map x) in
%     interp body env' body.root t
% \end{minted}

% \begin{align*}
%   &\text{interp env (BV v) t }&&\text{= v} \\
%   &\text{interp env (Var x) t }&&\text{= env x t} \\
%   &\text{interp env (Op op ids) t }&&\text{= op (map (interp env p) ids)} \\
%   &\text{interp env (Reg id_d init) 0 }&&\text{= init} \\
%   &\text{interp env (Reg id_d _) t }&&\text{= interp env p id_d (t - 1)} \\
%   &\text{interp env (Prim vmap body) t }&&\text{=}\\
%   &\text{interp ($\lambda$ x . interp env p (var_map x))} \\
%   &\text{body body.root}
% \end{align*}

\begin{flalign*}
  & \Time \hspace{0.2cm} t\in\mathbb{N} \hspace{1cm}  \textsf{Env} \hspace{0.2cm} e \in (\SynVar \rightharpoonup \Time \to \SynBv) \\
  &\vspace{0.5cm} \\
  &\textsc{Interp}\ :\ \SynProg \to \textsf{Env} \to \Time \to \SynNode \to \SynBv\\
  &\textsc{Interp}\ p\ e\ t\ (\SynBv\ b)\ = b&& \\
  &\textsc{Interp}\ p\ e\ t\ (\SynVar\ x)\ = e\ x\ t&& \\
  &\textsc{Interp}\ p\ e\ 0\ (\text{\IRReg}\ \_\ init)\ = init&& \\
  &\textsc{Interp}\ p\ e\ (t + 1)\ (\text{\IRReg}\ id\ \_)\ = \textsc{Interp}\ p\ e\ t\ p[id]&& \\
  &\textsc{Interp}\ p\ e\ t\ (\Op\ \mathsf{op}\ ids)\ = \llbracket op\rrbracket \ (\text{map}\ (\lambda id \ .\ \textsc{Interp}\ p\ e\ t\ p[id])\ ids)&& \\
  &\textsc{Interp}\ p\ e\ t\ (\text{\IRPrim}\ bs\ p')\ =&& \\
  %&\quad \text{let}\ \text{Interp}' = (\lambda\ i\ .\ \text{Interp}\ p\ e\ t\ p[i])\ \text{in}&&\\
  &\quad \text{let}\ e' = \lambda x, t'\ .\ \textsc{Interp}\ p\ e\ t' \left(p[bs\ x] \right) \text{in}\\
 % &\quad \text{let}\ e' = \text{fold}\ (\lambda\ (v, i),\  acc\ .\ acc[v \mapsto  \text{Interp}'\ i])\ \{\}\ vs\ \text{in}&& \\
  % &\quad \text{let}\ e' = \text{map} (\lambda \ .\ \text{Interp}\ p\ e\ t)\ ids\ \text{in}&& \\
  &\quad \textsc{Interp}\ p'\ e'\ t\ p'[p'.root]
\end{flalign*}


 %| Reg (id_clk, id_d, _) ->
 %   let clk = interp' id_clk t in
 %   let old_clk = interp' id_clk (t - 1) in
 %   if (old_clk, clk) = (0, 1)
 %    then interp' id_d (t - 1)
 %    else interp' id (t - 1)

% \begin{algorithmic}
% \end{algorithmic}

\caption{\lr's semantics as pseudocode.}
\label{fig:lr-interpreter-pseudocode}
\end{figure}

Before discussing the formal
  semantics of \UberLang,
  we present key 
  definitions.
We assume 
  a \textit{bitvector type}
  and, for simplicity,
  we elide bitvector
  widths.
We represent \textit{time} as a
  natural number.
A \textit{stream} is a function from $\Time$
  to bitvectors.
An \textit{environment} is
  a map from
  variable names
  to streams.


We give the semantics 
  for $\UberLang$ as an interpreter in 
  \cref{fig:lr-interpreter-pseudocode}.
We define the function \textsc{Interp} to interpret a program $p$ in environment $e$ at time $t$ and node $n$.
We do not define semantics for holes,
  as they are intended to be replaced
  by other constructs with well-defined semantics.

Most of the rules are straightforward.
A bitvector $\SynBv\ b$
  evaluates to
  its backing bitvector value $b$.
A variable node $\SynVar\ x$ 
  in an environment $e$
  at time $t$
  evaluates to
  the value returned 
  by the stream associated
  with $x$ in $e$
  at time $t$; 
using function notation, this is
  denoted by $e\ x\ t$.
A $k$-ary operator node
  $\Op{}\ op\ i_1\ldots i_k$ recursively
  interprets each operand in the current
  environment at the current time 
  and then applies $op$'s semantics,
  denoted $\llbracket op \rrbracket$,
  to the resulting values.
A register \IRReg $id\ b_{init}$ has two cases 
  depending on the current time: 
at time $t = 0$, a register evaluates
  to its initial bitvector value $b_{init}$; 
at nonzero times $t + 1$, a register evaluates
  to the value produced by the input $i$
  at the \textit{previous} timestep $t$.
A primitive \IRPrim{} $bs\ p'$
  in environment $e$ at time $t$
  is evaluated by interpreting
  the program $p'$ under
  the fresh environment $e'$
  formed by the binding map $bs$.
  
% \paragraph{Semantics of Registers}

% Registers capture and store
%   values for later use.
% A register has a \textit{clock input stream},
%   a \textit{data input stream},
%   and an \textit{initialization value}.
% We say that a register \textit{ticks}
%   at the \textit{positive edge} of the signal,
%   or simply that the register \textit{ticks},
%   when the clock signal transitions
%   from 0 to 1.
% Otherwise, we say
%   that the register \textit{holds}.
% When a register ticks, it updates its stored value
%   to whatever value is on the its data input stream.
%s foll
%Thus, a register will output a constant value
%   until it ticks, at which point it outputs
%   its previously stored value.\hl{TODO: This is not well written,
%   but info is here}

% Rather than store and pass around state
%   to track data from a previous time step,
%   \UberLang's semantics uses streams, indexed
%   by time, to access this data.
% In our stream semantics, 
%   we say that a register \textit{ticks} at time $t > 0$ when 
%   the clock stream is 0 at time $t - 1$ and 1 at time $t$.
% Otherwise the register holds at time $t$.

% Register semantics are broken into three rules:
% \textsc{RegZero},
% \textsc{RegHold}, and \textsc{RegTick}

% \textsc{RegZero} gives semantics for a register at time $t = 0$;
% it simply evaluates to its initialization value.
% \textsc{RegHold} gives semantics
%   for a register that holds,
%   at which point the register
%   outputs its previous
%   evaluation from time $t-1$.
% \textsc{RegTick} gives semantics
%   for a register that ticks rule describes the behavior
%   of a register when its clock ticks:
%   when the clock ticks, a register
%   evaluates to whatever data was passed into
%   the register at the previous time step $t-1$

% First, the subexpressions for computing the
%   clock and data inputs are evaluated.
% The state is threaded sequentially through
%   these recursive evaluations,
%   but recall that the order does not
%   matter because each rule only
%   accesses state for its expression's
%   unique identifier---%
%   thus the semantics faithfully models
%   the results of a parallel evaluation.
% Given the new inputs $v_{clk}$ and $v_{data}$
%   for clock and data respectively to this register,
%   the \textsc{RegHold} rule states that
%   the register continues to hold its old value
%   $v_{q'}$ but updates its state to
%   save the new clock and data input values,
%   i.e., to $(v_{clk}, v_{q'}, v_{data})$.
% The \textsc{RegTick} rule is similar,
%   but handles the case where the clock
%   is transitioning from low to high (0 $\rightarrow$ 1).
% In this case, the clock and data
%   subexpressions are recursively evaluated as before,
%   but the \textsc{RegTick} rule states that
%   the register updates its value to
%   $v_{data'}$ (the most recently seen data input)
%   and updates its state to
%   save the new clock and data input values,
%   i.e., to $(v_{clk}, v_{data'}, v_{data})$.

% \paragraph{Semantics of \textsc{Prim}}

% The \textsc{Prim} rule specifies how
%   an instance of a primitive
%   $\seq{i,\usebox{\boxprim}\ x_1\ x_2\ \ldots\ x_n\ e_1\ e_2\ \ldots\ e_n\ e_s}$
%   evaluates.
% The \textit{primitive input variables}
%   $x_1, x_2, \ldots, x_n$
%   model both the inputs to this primitive
%   (including the clock if this is a sequential primitive)
%   as well as parameter and port values
%   necessary to configure this primitive.
% The \textit{primitive argument expressions}
%   $e_1, e_2, \ldots, e_n$
%   represent the expressions that
%   flow into the input variables:
%   for configuration parameters,
%   they will be set to holes ($\blacksquare$)
%   in the sketch used for synthesis
%   which will then typically be
%   filled in by constants;
%   for inputs coming from other
%   components, these expressions
%   will be other parts of the design.
% Finally,
%   the expression $e_s \in \SpecLang$
%   encodes the semantics for this
%   primitive as extracted from
%   the vendor-provided HDL.
% The \textsc{Prim}  rule states that first
%   all the primitive argument expressions
%   are evaluated to get their updated outputs
%   $v_1, v_2, \ldots, v_n$ and the state map
%   resulting from their recursive evaluation.
% Again, although the formal semantics
%   sequentially threads the state
%   through these evaluations,
%   the actual order does not matter ---
%   unique identifiers for each subexpression
%   ensures that evaluating
%   subexpressions within a clock cycle
%   commutes.
% Given the results of evaluating all
%   the primitive argument expressions,
%   the rule next evaluates $e_s$,
%   i.e., the semantics for this primitive,
%   under the environment $E$
%   extended so that each
%   primitive input variable maps to
%   its corresponding argument,
%   $E[x_1 := v_1][x_2 := v_2]\ldots[x_n := v_n]$.
% Finally, $e_s$ is evaluated under this extended environment
%   and the state map $S_n$ resulting from evaluating all
%   the argument expressions to get a final
%   output $v_s$ for this primitive and
%   state map $S'$ which additionally
%   includes any updates to the internal
%   state of this primitive.
  
% \subsection{Interpreting Designs Over Inputs}
% \hl{This is a previous definition of Interp, and is now out of date. however, we use this version later in this section so I am preserving it so that we can be careful about how we update future things}

% We specify a function
%   $\textsc{interp} : \mathsf{Env}^{*} \to \UberLang \to \mathsf{BV}$
%   for interpreting a well-formed design $e$ 
%   (i.e., where all subexpression identifiers are unique)
%   on inputs $[E_1, E_2, \ldots, E_n]$
%   such that $\textsc{interp}$ respects $\rightarrow^{+}$.
  
% That is,
%   $\textsc{interp}([E_1, E_2, \ldots, E_n], e) = v$
%   if and only if
%   \multiEvalsTo{[E_1, E_2, \ldots, E_n]}{e}{S_{init}}{v}{S_{ignore}} and
%   $S_{init}[i] = (1, b, 0)$ 
%   for any 
%   $\seq{i, \usebox{\boxreg}\ e_{clk}\ e_{data}\ b}$
%   that occurs as a subexpression in $e$.
% In other words,
%   we set the initial state $S_{init}$ so that
%   the identifier for each
%   \lstinline[language=thelang]{Reg} subexpression
%   maps to its default value $b$.
% This ensures that all subsequent state lookups will succeed.

% We can interpret a program on an input with $\textsc{interp} : \mathsf{Env}^{*} \to \UberLang \to \mathsf{BV}$  where $\textsc{interp}([E_1, E_2, \ldots, E_n], e) = v$, given that \multiEvalsTo{[E_1, E_2, \ldots, E_n]}{e}{S_{init}}{v}{S_{ignore}} and $S_{init}[i] = (1, b, 0)$ whenever $\seq{i, \lstinline[language=thelang]{Reg}\ e_{clk}\ e_{data}\ b} \in e$.
% In other words, we initialize the initial state where the ID of each \lstinline[language=thelang]{Reg} subexpression maps to its default value.
% This ensures that all subsequent state lookups will succeed.

\subsection{Program Synthesis}
\label{sec:formalization-program-synthesis}

% $\lrfn$ performs sketch-based program
%   synthesis~\cite{solar2008program}.
% Recall that an invocation of $\lrfn$ 
%   requires three arguments:
%   a sketch $\Psi\in\Sketch$,
%   a design fragment $d\in\SpecLang$,
%   and a time value $t\in\Time$.
% Further, recall that
%   sketch $\Psi = (\psi, h)$ is 
%   the program $\psi \in \SketchLang$ whose
%   holes will be
%   filled by synthesis,
%   along with map
%   $h$ which maps each hole
%   in $\psi$ to a finite set
%   of valid hole-free nodes in $\ImplLang$
%   that can be used to fill the mapped hole.
% $\lrfn$ synthesizes a program
%   by filling each hole $\blacksquare_{x_i}$ in $\psi$
%   with a node $n_i \in h[\blacksquare_{x_i}]$,
%   such that the result
%   is equivalent to $d$
%   after running for
%   $t$ clock cycles.
%The holes represent the configuration
%  space of a sketch and
%  are filled during synthesis.
%The design specification $d \in \SpecLang$ 
%  provides the semantics that the
%  result of synthesis should have.
%Lastly, $t$ indicates the
%  number of clock cycles.
% The objective of 
%   the sketch-based 
%   program synthesis is 
%   to fill in all holes
%   of $\Psi$ with 
%   expressions from $\ImplLang$ such
%   that the semantics of the
%   resulting program is equivalent
%   to that of $d$ after $t$ clock cycles
%   have run.
% In \cref{subsec:lrfn-bmc} we
%   use bounded model checking to
%   extend $\lrfn$'s guarantees
%   beyond the single timestep
%   at clock cycle $t$.

%%%%%%%%%%%%% BEGIN EDIT
$\lrfn$ performs sketch-based program
  synthesis~\cite{solar2008program}.
%%%%%%%%%%%%%% END EDIT
Operationally, we implement
  the \textsc{Interp} function from 
  \Cref{fig:lr-interpreter-pseudocode}
  in Rosette, a solver-aided host
  language~\cite{torlak2014lightweight}.
Let sketch $\Psi = (\psi, h) \in \Sketch$, where
  $\psi \in \SketchLang$ has holes
  $\blacksquare_{x_{i}}$
  and $h$ maps $\psi$'s holes
  to the set of structural nodes
  that can legally fill the mapped hole.
Given a design $d$,
  we query Rosette if there
  are nodes 
  $n_1, n_2, \ldots n_k$ 
  such that 
  $n_i \in h[\blacksquare_{x_i}]$
  and
  $p = \Psi[\blacksquare_{x_1} \mapsto n_1, \ldots]$
   is well-formed and equivalent to $d$
   (i.e., we ask Rosette to fill
   each hole with a node associated with the node in $h$).
Program equivalence between well-formed
  programs $p$ and $d$ at time
  $t$, written $p\cong_t d$, is defined as
  \begin{align*}
    & p.fv = d.fv\ \wedge \\
    &\forall e\ s.t.\ \textrm{domain}(e) = p.fv, \\
    &\quad\textsc{Interp}\ p\ e\ t\ p.root =
   \textsc{Interp}\ d\ e\ t\ d.root.
  \end{align*}
% {\color{blue}
% \lr{} captures the quantification $\forall e$ by creating an initial environment
%     $e_0$
%     for a program $p$ in which all free variables
%     of $p$ are mapped to symbolic constant
%     streams:
%     $$e_0 = \lambda x,t.~bv_x,$$
%     where $bv_x$ is a unique
%     symbolic bitvector constant associated
%     with free variable $x$.
% }
In \cref{subsec:lrfn-bmc}, we
  use bounded model checking to
  extend $\lrfn$'s guarantees
  beyond the single timestep
  at clock cycle $t$.
  

\subsection{Correctness and Completeness of $\lrfn$}
\label{subsec:lr-correctness-and-completeness}

Recall that the synthesis
  function $\lrfn$ is partial.
We say that $\lrfn$
  is \emph{correct} if
  it 
  returns a program
  $\lrfn(\Psi,d,t) = p$  where
  $p$ is
  a well-formed
  completion of $\Psi = (\psi, h)$,
  meaning $p = \Psi[\blacksquare_{x_1} \mapsto n_1, \ldots]$
            such that $n_i \in h[\blacksquare_i]$ for all $i$
  and $p\cong_t d$.

% For a sequential design, $\lrfn$ is \emph{correct}, 
%   by which we mean
%   that for a sketch $\Psi$,
%   design $d\in \SpecLang$,
%   and $t \in \Time$,
%   if $\lrfn(\Psi, d, t)$ successfully
%   synthesizes an implementation
%   $p \in \ImplLang$,
%   then after $t$ clock cycles
%   $d$ and $p$ evaluate to the same
%   value under every possible
%   environment:
%   \begin{align*}
%       & \forall\ \Psi \in \Sketch,\, d \in \SpecLang, e \in \ImplLang, t \in \Time \\
%       & \lrfn(\Psi, d, n) = e \Rightarrow  e.fv = d.fv \wedge \\
%       &\forall\ env \in Env(d), \ 
%       \textsc{Interp}(d, env, n) = \textsc{Interp}(e, env, n)
%   \end{align*}
%   % \begin{align*}
%   % & \textsc{Interp}(d, env, n) = \textsc{Interp}(e, env, n)
%   % \end{align*}
  
% This follows directly from the program synthesis definition that Rosette implements\footnote{The actual Rosette synthesis query is stronger than the formula shown here in order to rule out trivial solutions~\cite{rosette:synthesis,rosette4}. We show the weaker variant here for simplicity.}.

% Let $d, e \in \UberLang$ without holes, we say that $e$ \emph{implements} $d$ after $t$ clock cycles, $\textsc{Impl}(d,e,t)$ when
% \[\begin{array}{rl}
% \forall env \in Env, &
% valid(env,d) \implies valid(env,e)\; \wedge \\
% &\textsc{Interp}(e_1, env, n) = \textsc{Interp}(e_2, env, n)
% \end{array}\]

% \hl{Define function $Env(d)$ for valid envs of $d$}

% Formally,

%   \begin{align*}
%   & \lrfn(\Psi, d, n) = e \implies \\
%   & \quad\exists e_1, e_2, \ldots, e_k \in \ImplLang,\\
%   & \quad\quad e = \Psi[\blacksquare_{x_i} \mapsto e_i]\ \wedge \\
%   & \quad\quad \forall\ env \in Env(d), \textsc{Interp}(d, env, n) = \textsc{Interp}(e, env, n)
%   \end{align*}

% Correctness for combinational designs
%   also follows, but is a special case
%   where the output results do not
%   depend on cycling the clock signal,
%   elided here for simplicity.

Furthermore, we say
  that $\lrfn$ is
  \emph{sketch-complete}
  if $\lrfn(\Psi,d,t)$
  is defined whenever
  there exists a
  well-formed completion
  $p$ of $\Psi$
  such that $p\cong_t d$.
That is, synthesis is
  correct if it never
  returns an erroneous result
  and sketch-complete
  if it returns a
  correct result
  whenever one exists.

{%\color{blue}
% From Gilbert
We have implemented $\lrfn$
    with Rosette 
    (see \cref{sec:formalization-program-synthesis}),
    which guarantees our system is correct and complete
    under the following assumptions:
\begin{enumerate}
    \item Correctness of Rosette and underlying SMT solvers;
    \item That our encoding of \lr{} is bug-free;
    \item That the lowering of \textsc{Interp}
    to SMT formulas by  Rosette always terminates.
    This is possible when partial evaluation of \textsc{Interp} on arguments $p$, $t$ and $n$ terminates (independently of the value of $e$).
    %   This is implemented in Rosette by
    %   partially evaluating \textsc{Interp} on
    %   inputs $p$, $t$, and $n$ without evaluating $e$,
    %   the environment.
    %   Below, we prove that this partial evaluation
    %   (which eliminates the recursion) terminates
    %   \emph{and} does not structurally depend on
    %   the value of $e$.
\end{enumerate}

        %unconditional of the input,
        %yielding a finite
        %unquantified SMT problem
        %(which is necessarily decidable).
}

\begin{lemma}
\label{lemma:interp-is-primitive-recursive}
Let $p$ be a well-formed program, 
    $e$ an environment,
    $t$ a \Time,
    and $n$ be a node belonging to $p$.
Then \textsc{Interp} is primitive recursive 
    (i.e. terminates) in the arguments $p$, $t$, and $n$.\tighten
    % p comes up in the penultimate case discussed, so I think it's appropriate
    %\hl{Should we remove $p$?}
\end{lemma}

\begin{proof}[Proof of Lemma~\ref{lemma:interp-is-primitive-recursive}]
Recall that a function $f(x,y,z)$ is
    primitive recursive in arguments $x$ and $y$
    (under a lexicographic ordering) 
    if in the definition of $f$ every
    recursive call $f(x',y',z')$ is made
    with values $(x',y')$ such that $x' < x$ 
    or $x' = x \wedge y' < y$.
    If $x$ and $y$
    are drawn from the natural numbers 
    (or another well-ordered set),
    then the recursion is guaranteed to terminate.

Under what order is 
  \textsc{Interp} primitive recursive?
Because our program
    is well-formed, it must be free
    of combinational loops (see \cref{property:free-of-combinational-loops}).
Formally, this means we have an acyclicity
    witness function $w : p.all\_ids \to \mathbb{N}$
    that monotonically increases in the direction of
    dataflow in our circuit.
Each node $n$ argument passed to \textsc{Interp}
    has an \textsf{Id} that is unique and distinct
    from the \textsf{Id}s used in $p$ or any of $p$'s
    subprograms (\textbf{W2});
    we denote this \textsf{Id} as $id_n$.
We can associate each $n$ argument
    to a recursive call of \textsc{Interp}
    with a number $w(id_n)$.
We claim that \textsc{Interp} is
    primitive recursive under
    the lexicographic ordering on $(t, w(id_n))$.
    
To prove this claim we need to demonstrate that
    if \textsc{Interp} with time and node
    arguments $t'$ and $n'$ makes a recursive
    call to  \textsc{Interp} with time and
    node arguments $t''$ and $n''$, then the following
    condition holds:
    \small
\begin{equation}
\label{eqn:primitive-recursion-condition}
    t'' < t' \vee \left(t'' = t' \wedge w(id_{n''}) < w(id_{n'})\right).
\end{equation}
\normalsize
To do this it suffices to examine each case of \textsc{Interp}'s definition.

When $n'$ is a $\SynBv$ constant,
    \textsc{Interp} makes no recursive calls,
    and the condition in \cref{eqn:primitive-recursion-condition}
    holds vacuously.
    
When $n'$ is a \IRReg{} node \textsc{Interp} either terminates
  (when $t'=0$) or makes a
  recursive call with time value $t'' = t' - 1$,
  maintaining the condition in \cref{eqn:primitive-recursion-condition}.
  
When $n'$ is an operator node, 
  \textsc{Interp} recursively interprets
  the operands with time arguments $t'' = t'$.
However, each operand's id $id''$ belongs to $\textsc{inputs}(n')$,
    and, by~\cref{property:free-of-combinational-loops},
    $w(id_{n'}) > w(id'')$,
    so our condition holds.
  
This leaves us with the less obvious
  cases in which $n'$ is either a \IRPrim or $\SynVar$,
  which work together in tandem.
When $n' = \text{\IRPrim{}}~bs~p'$,
    \textsc{Interp} makes a recursive
    call with node argument $p'.root$
    and time argument $t$.
By ~\cref{property:free-of-combinational-loops},
    $w(p'.root) < w(id_{n'})$,
    and the condition in \cref{eqn:primitive-recursion-condition} holds.
\textsc{Interp} also defines a new environment for execution
    of $p'$ via $\lambda$-abstraction, and this in turn
    will recursively invoke \textsc{Interp}.
These environments are only invoked by the rule for variables,
    which we handle presently.
    
When $n' = \SynVar~x$, the environment is 
    invoked on variable $x$.
Here, there are two possible cases.
First, we are interpreting the
    top-level program $p$. 
As this is the initial, top-level environment, there is no further recursion.
Second, we are
    interpreting a sub-program $p'$
    and $e'~x~t = \textsc{Interp}~p~e~t~(p[bs~x])$
    is actually a recursive call into the
    program $p$ one level up,
    with its environment $e$.
In this latter case,
    note that $w$ is defined such that
     $w(id_{p[bs~x]}) = w(bs~x) < w(id_{\SynVar~x})$
    (item 3 of \cref{property:free-of-combinational-loops}),
    satisfying our property.
All cases are complete.
\end{proof}

%%%%%%%%%%%%%%%%%%%%%%%%%% END PROOFS


From this, we conclude that
  all possible substitutions for $\Psi$
  are attempted, and $\lrfn$ is sketch-complete.
%Furthermore, $\lrfn{}$ is \emph{sketch-complete},
%  by which we mean that
%  if there are nodes $n_1, n_2, \ldots, n_k \in \ImplLang$
%  such that 
%  $\Psi[\blacksquare_{x_1} \mapsto n_1,
%  \blacksquare_{x_2} \mapsto n_2,
%  \ldots, \blacksquare_{x_k} \mapsto n_k]$
%  implements design $d$, then
%  $\lrfn(\Psi, d, t)$
%  will find some implementation
%  assignment of holes in $\Psi$
%  to implement $d$.
%This, again, follows directly from the definition
%  of program synthesis and the fact that
%  our queries to Rosette require only 
%  the theory of bitvectors,
%  which is decidable~\cite{Jonas2019thesis, BradleyDecidability}.
%Formaly, if 

%  \begin{align*}
%  & \exists e_1, e_2, \ldots, e_k \in \ImplLang,
%    \forall\ env \in Env, \\
%  & \quad \textsc{Interp}(d, env, n) = \textsc{Interp}(\Psi[\blacksquare_{x_i} \mapsto e_i], env, n)
%  \end{align*}
%then 
%\begin{align*}
%& \exists\ e' \in \ImplLang,\ \forall\ env \in Env, \\
%& \quad \lrfn(\Psi, d, n) = e'\ \wedge 
%  \textsc{Interp}(d, env, n) = \textsc{Interp}(e', env, n).
%\end{align*}


\paragraph{Trusted Computing Base.}

The \textit{trusted computing base} (TCB) of a
  system is the set of components
  it assumes to be correct~\cite{MacKenzieComputingTrust}.
A bug anywhere in the TCB
  could cause the guarantees
  made by that system to be violated.
\lr's  TCB includes:
  Rosette and the underlying
      SAT/SMT solvers that Rosette queries
      (Bitwuzla, cvc5, Yices2, and STP);
  the internal Yosys passes \lr
      uses to extract primitive semantics
      and translate design specifications
      from behavioral Verilog into
      $\SpecLang$;
  the semantics for $\UberLang$,
    which we assume conservatively
    models non-cyclic (DAG) designs;
  our code to translate from
      the $\ImplLang$ to
      structural Verilog; and
  the vendor-provided Verilog
    simulation models for FPGA primitives.
Each TCB component 
  has also been thoroughly tested,
  as described in \cref{sec:evaluation}.
Importantly,
  sketches and sketch generation
  are \textit{not} in \lr's TCB: %
  even if there were a
  bug in \lr's sketch-related components,
  it would not violate
  \lr's correctness guarantees.


\subsection{Multiple Clock Cycle Guarantees with $\lrfnbmc$}
\label{subsec:lrfn-bmc}


The preceding completeness and 
    correctness properties for
    $\lrfn$ 
    guarantee that
    running the
    synthesized program
    $p$ and the design $d$
    for $t$ clock cycles
    produces the same output.
To extend this guarantee, \lr supports
    a form of
    bounded model checking, 
    where
    synthesis ensures that
    $p$ is semantically equivalent
    to $d$ for $c$ additional clock cycles
    starting at time $t$.
We formalize this
    with the function $\lrfnbmc$,
    which takes a sketch $\Psi$,
    a behavioral design $d$,
    a number of clock cycles $t$,
    and a model checking
    time bound $c \geq 0$
    and returns an implementation
    $p \in \ImplLang$
    that is equivalent to $d$
    at time steps $t, t+1, \ldots, t + c$.
    

Our correctness and completeness guarantees are
    similar to those for $\lrfn$:

    \small
  \begin{align*}
    & p.fv = d.fv\ \wedge \\
    &\forall e\ s.t.\ \textrm{domain}(e) = p.fv, \\
    &\quad\bigwedge_{i=t}^{i=t+c}
      {\textsc{Interp}\ p\ e\ i\ p.root = \textsc{Interp}\ d\ e\ i\ d.root}.
  \end{align*}
  \normalsize

\subsection{Beyond \lr}
\label{sec:sem-beyond}

%This section formalized \lr's domain-specific language for designs and its overall synthesis process. However, 
$\UberLang$, its semantics,
  and the synthesis approach we describe here 
  are useful for applying program synthesis
  to other hardware design problems.
For example,
  the synthesis problem detailed above could be ``flipped''
  to decompile structural designs back 
  to higher-level behavorial designs,
  i.e., synthesizing from $\ImplLang$
  to an expression in $\SpecLang$.
Such decompilation has seen recent
  interest for
  recovering equivalent but faster-to-simulate
  models and for porting models across
  different architectures~\cite{sisco2023loop}.
As another example,
  the synthesis approach could be
  adapted to help port designs by
  synthesizing expressions in
  $\ImplLang$ that use one set of primitives
  on one architecture from 
  other designs in $\ImplLang$ that use
  a different set of primitives from
  a different architecture.
Thus, the formalization in this section
  transcends the particular challenges
  of FPGA technology and provides
  a reusable foundation for exploring
  a much broader range of hardware design challenges
  from a program synthesis perspective.
  