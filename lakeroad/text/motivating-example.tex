\section{Motivating Example: 8-bit Add}
\label{sec:motivating-example}


\begin{figure}
\begin{minted}{systemverilog}
module add8(input [7:0] a, b, output [7:0] y);
  assign y = a + b;
endmodule
\end{minted}
\caption{Behavioral SystemVerilog implementation of 8-bit add.}
\label{fig:8-bit-add-behavioral}
\end{figure}
  
\begin{table}
    \centering
    \begin{tabular}{c|ccc}
        & LUTs & CARRY8s & Datapath Delay (ns)  \\
        \hline
         first attempt& 11 & 0 & {\color{red}{0.799}}  \\
         second attempt& 8 & 1&  {\color{OliveGreen}{0.410}}  \\
    \end{tabular}
    \caption{Vector add statistics}
    \label{tab:vector-add-stats}
\end{table}
  
\begin{figure}
\begin{minted}{systemverilog}
module add8(input [7:0] a, b, output [7:0] y);
  impl _impl (a, b, y);
endmodule

module impl(input [7:0] a, b, output [7:0] y);
  wire [7:0] p;
  LUT2 #(.INIT(4'h6)) l0 (a[0], b[0], p[0]);
  // ...7 LUT2s omitted...
  CARRY8 carry0 (.CI(1'b0), .DI(a), .S(p), .O(y));
endmodule
\end{minted}
\caption{Structural Xilinx UltraScale+ SystemVerilog implementation of 8-bit add.}
\label{fig:8-bit-add-structural}
\end{figure}

To motivate the use
  of solver-aided programming
  to generate ISA instruction implementations,
  let's walk through an example
  of implementing an 8-bit add instruction
  for Xilinx UltraScale+ FPGAs.
As a first pass,
  we will attempt to 
  generate an UltraScale+ implementation
  of 8-bit add
  using the traditional flow.
In the traditional flow, we start with 
  a high-level,
  behavioral SystemVerilog implementation
  of 8-bit add,
  shown in \cref{fig:8-bit-add-behavioral},
  and use Xilinx'x Vivado toolchain
  to synthesize%
  \footnote{Note that we use the term ``synthesis'' here to specifically represent the step in traditional hardware compiler flows in which the design is compiled into logic gates, optimized, and then mapped to FPGA primitives.}
  , place, and route the design
  for an UltraScale+ board.

The result,
  recorded in the first row of
  \cref{tab:vector-add-stats},
  uses 11 LUTs,
  and has a datapath delay of 0.799ns.
At first glance,
  this may seem just fine.
However, those familiar with
  Xilinx FPGAs
  will notice that the implementation
  doesn't use a single CARRY8---%
  a specialized fast carry chain unit
  designed specifically for speeding up
  addition.
Noticing that,
  we attempt a more brute-force implementation,
  shown in
  \cref{fig:8-bit-add-structural}.
In this implementation,
  we manually implement the 8-bit addition
  using the UltraScale+ primitives
  \texttt{LUT2} and \texttt{CARRY8}.
This requires more intimate
  knowledge of the FPGA backend.
In the first place, we need to know
  the FPGA contains
  specialized carry logic.
Knowing that,
  we are then required to connect the wires by hand,
  and to determine the \texttt{INIT}
  values for the LUTs.
But our effort pays off:
  when we synthesize, place, and route
  our new implementation,
  the resulting design
  has a datapath delay of
  0.410ns---an improvement of nearly 2x!
  
This example clearly illustrates the problem:
  even on simple designs---indeed,
  just a single 8-bit add,
  traditional hardware compilers behave
  in unexpected ways,
  and users need backend-specific knowledge
  to extract maximum performance.
Note that we can also recover the performance
  of our behavioral implementation
  by explicitly clocking the result of
  \texttt{a + b}
  into a register, using an
  \texttt{always_ff}
  block,
  but an inexperienced user is unlikely
  to figure this out
  without first scouring the internet
  for help.
Furthermore,
  whether or not a design uses a register
  is irrelevant
  to the carry;
  this gives a clear example of the
  underlying heuristics
  producing unexpected results.


