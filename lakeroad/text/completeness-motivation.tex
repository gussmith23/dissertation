% Gus integrated this content into the intro.

% \section{Motivation: Techmapping Completeness}
% \label{sec:completeness}

% In this section,
%   we empirically motivate
%   the need for 
%   \lr.
% We first give a formal definition
%   for 

% \hl{
% I'm split on whether
%   we should introduce completeness
%   up front in its own section,
%   or split it up into existing sections.
% Specifically,
%   we have two components to be introduced:
%   the formal completeness defintion
%   and the empirical incompleteness evidence
%   of other tools

  
% }



\section{Motivation: Limitations of Current Approaches}
\label{sec:completeness}

Hand-written syntactic matching
  has several key limitations 
  when applied to programmable primitives.
\paragraph{Limitation 1: Mapping Completeness}
We refer to completeness as 
  a tool's ability to map a design
  to a primitive, if such a mapping exists.
Completeness is challenging
  to guarantee for programmable primitives,
  as there may be many designs that can
  map to a primitive---far too many to 
  capture via handwritten patterns.
This rigidity \hl{don't think this is the word} 
  is further exacerbated by the fact that a high-level operation can be represented with 
  several equivalent designs, meaning that each representation of a design would require
  a pattern.
As shown in \hl{andrew teaser table},
  even canonical representations of small workloads are not mapped 
  correctly \hl{finish off this sentence strongly}.

\paragraph{Limitation 2: Correctness Guarantees}
Handwritten patterns also do not provide formal
  correctness guarantees; a rewrite can
  match any pattern in the design, and 
  replace it with anything else.
Perhaps unsurprisingly, as a result, a
  significant number of bugs have been found
  across all major hardware synthesis tools \hl{sneak peek table?}
\hl{sentence about "if only formal semantics existed maybe? might be a bit tacky}
\paragraph{Limitation 3: Extensibility}
Even if a handful of patterns can be
  derived and implemented for a new primitive,
  adding full support requires writing patterns 
  to cover \textit{all possible ways} the primitive can be used, a daunting task.
This process then must be \textit{repeated}
  each time a new primitive needs to be supported.
