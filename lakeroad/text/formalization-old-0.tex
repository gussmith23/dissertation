\section{Formalization}
\label{sec:formalization}

% SpecIR example:
% (bvand a b) (a and b are (bitvector 8))
% LRIR:
% (hash-ref
%  (hw-module-instance 
%   "LUT2"
%   memory = 4'b1000,
%   i0 = a,
%   i1 = b)
%  'O)

% (rosette-synthesize SpecIR-program LRIR-program)
% -> assert SpecIR-program == LRIR-program
% -> solve

% I_S : SpecIR -> bitvector
%  --> "just" Rosette/Racket. (bvand a b) evals automatically.
%  we don't have to do (interpret (bvand a b))
% I_LR : LRIR -> bitvector
%  (lr:hash-ref ...)
%  (interpret (lr:hash-ref ...))

% exists M, 
%  forall inputs, I_S(P_S, inputs, M) 
%                   == I_LR(P_LR, inputs, M).
% synthesis: find M

% in the past/in the current version of the paper:
% I_S and I_LR -> bitvectors
%
% New thing:
% I_S and I_LR -> signals

% This will change the synthesis problem
% Old problem:
% exists M, 
%  forall inputs, I_S(P_S, inputs, M) 
%                   == I_LR(P_LR, inputs, M).
% synthesis: find M

% now we have multiple possible synthesis problems
% exists M, 
%  forall inputs, (signal-value I_S(P_S, inputs, M))
%                   == (signal-value I_LR(P_LR, 
%                                     inputs, M)).
% synthesis: find M

% exists M, 
%  forall inputs, 
%   (let o0 = I_S(P_S, inputs + clk=0, M);
%    let o1 = I_S(P_S, inputs + clk=1 + state(o0), M);
%    let o2 = I_S(P_S, inputs + clk=0 + state(o1), M);
%    ...
%    (signal-value oN))
%   == 
%   (let o0 = I_LR(P_LR, inputs + clk=0, M);
%    let o1 = I_LR(P_LR, inputs + clk=1 + state(o0), M);
%    let o2 = I_LR(P_LR, inputs + clk=0 + state(o1), M);
%    ...
%    (signal-value oN))
% synthesis: find M

In this section, we formalize \lr
  and provide a mathematical argument
  for its correctness.
We start by introducing and 
  formalizing the semantics used in \lr,
  beginning with introducing
  its base value type:
  the \signal datatype (\cref{sec:signal}).
Then, we define syntax and semantics
  of the imported Verilog (\cref{sec:verilog-semantics})
  and of the \lr IR, \lrir (\cref{subsec:spec-ir-and-lrir}).
Finally, we present
  a mathematical model
  of the \lr workflow
  (formalized via the \lrfn function)
  and prove its correctness
  (assuming the correctness of the underlying program synthesis tool and HDL compiler) in \cref{sec:correctness-argument}.
%Then we introduce the \lrfn function
  %to formalize the \lr workflow
  %introduced in \cref{sec:overview}
  %and argue for its correctness
  %via stating the \lr-Correct property
  %(\cref{sec:correctness-argument}).

  
%%%%%%%%%%%%%%%%%%%%%%%%%%%%%%%%%%%%%%%%%%%%%%%%%%%%%%%%%%%% 
% Subsection: Signals
%%%%%%%%%%%%%%%%%%%%%%%%%%%%%%%%%%%%%%%%%%%%%%%%%%%%%%%%%%%% 
\subsection{The \texttt{signal} Datatype}
\label{sec:signal}

To formalize
  expressions
  in both the
  imported Verilog 
  and \lrir,
  we first introduce their base value type:
  the \signal datatype.
\signal{}s are fundamental to \lr's ability
  to synthesize sequential designs.
While bitvectors alone are expressive enough
  to capture the semantics of 
  \textit{combinational} expressions,
  they are not expressive enough to capture
  \textit{sequential} expressions.
Thus, we introduce \signal{}s
  as a more expressive datatype
  for capturing hardware expressions.
  
A \signal represents
  the value on an $n$-bit
  wire and
  the current state
  of the stateful elements
  in the context where the \signal
  was produced.
Consequently, we represent a
  \signal{} as a pair $(v,m)$,
  where $v$ is a bitvector
  capturing the value on the wire
  and $m$ is a map
  capturing the current state.

% \begin{figure}
\begin{minted}[fontsize=\footnotesize]{python}
def merge(state0, ...):
  new_s = {}
  for s in [state0, ...], key in s:
    (new_val, new_vers) = s.get(key)
    (_, old_vers) = new_s.get_or_default(key, (_, -1))
    if new_vers > old_vers:
      new_s[key] = (new_val, new_vers)
  return new_s

def update(state, key, val):
  new_state = state.copy()
  (_, old_vers) = state.get_or_default(key, (_, -1))
  new_state[key] = (val, old_vers + 1)
  return new_state
\end{minted}
% In case we want to add examples
% >>> v0 = register(signal(0), signal(0xa)); v0
% signal(0, {"clk": 0, "d": 0xa, "q": 0 })
% >>> register(signal(1, v0.state), signal(0xa, v0.state))
% signal(0xa, {"clk": 1, "d": 0xa, "q": 0xa })
    \caption{
% Pseudocode implementation 
%   of a simple register 
%   using the \texttt{signal} data structure.
% The register takes two \texttt{signal}s as input:
%   \texttt{clk} and \texttt{d}.
% The states from the two inputs are first merged
%   into a single map,
%   which is used to look up old values of
%   \texttt{clk}, \texttt{d},
%   and the output \texttt{q}.
% Finally, we compute
%   the new value of \texttt{q}
%   based on whether or not
%   the clock ticked.
Pseudocode implementation
  of merging two state
  maps.
Versioning guarantees
  resolution in the case
  of state merge conflict.
}

    \label{fig:state-merge-and-update}
\end{figure}
\begin{figure} 
\begin{minted}[fontsize=\footnotesize]{python}
def merge(state0, ...):
  new_s = {}
  for s in [state0, ...], key in s:
    (new_val, new_vers) = s.get(key)
    (_, old_vers) = new_s.get_or_default(key, (_, -1))
    if new_vers > old_vers:
      new_s[key] = (new_val, new_vers)
  return new_s

def update(state, key, val):
  new_state = state.copy()
  (_, old_vers) = state.get_or_default(key, (_, -1))
  new_state[key] = (val, old_vers + 1)
  return new_state

def register(clk, d):
  # look up previous values
  s = merge(clk.state, d.state)
  (old_clk_val, _) = s.get_or_default("clk", (0, _))
  (old_d_val, _) = s.get_or_default("d", (0, _))
  (old_q_val, _) = s.get_or_default("q", (0, _))

  # compute the output
  clock_ticked = old_clk_val == 0 and clk.value == 1
  new_q_val = old_d_val if clock_ticked else old_q_value
  new_state = s.update("clk", clk.value)
               .update("d", d.value)
               .update("q", new_q_value)
  
  return signal(new_q_val, new_state)

>>> v0 = register(signal(0), signal(0xa)); v0
signal(0, {"clk": (0, 0), "d": (0xa, 0), "q": (0, 0) })
>>> register(signal(1, v0.state), signal(0xa, v0.state))
signal(0xa, {"clk": (1, 1), "d": (0xa, 1), "q": (0xa, 1) })
\end{minted}
    \caption{
Pseudocode implementation 
  of the core \signal operations
  (\texttt{merge} and \texttt{update}),
  and a demonstration of how \signal{}s
  can be used to implement the semantics
  of a simple register. 
The register takes two \texttt{signal}s as input:
  \texttt{clk} and \texttt{d}.
The states from the two inputs are first merged
  into a single map,
  which is used to look up old values of
  \texttt{clk}, \texttt{d},
  and the output \texttt{q}. 
Version numbers in the state maps
  help resolve potential merge conflicts.
Finally, we compute
  the new value of \texttt{q}
  based on whether or not
  the clock ticked. 
Note that this is just an example;
  \lrir does not include
  a register primitive.
Rather, registers are contained within
  the Verilog models
  we import, and their semantics
  are captured in 
  \cref{fig:verilog-semantics}
  (see \textbf{dff-hold} and \textbf{dff-tick}).
}
    \label{fig:signal-complete}
\end{figure}
The state map $m$
  maps keys (which uniquely identify
    an expression's internal state elements)
  to a tuple $(\mathit{sv}, \mathit{vers})$,
  where $\mathit{sv}$
  is the value of the state
  and $\mathit{vers}$ is
  its version number 
  (necessary for handling conflicts during merging).
Beyond indexing into a state map,
  we define two core operations of state maps:
  \textit{merging} and \textit{updating,}
  both defined via pseudocode
  in \cref{fig:signal-complete}.
Merging two state maps
  simply unions the two maps together,
  while handling conflicts by taking the value
  with the highest version number.
Updating a state map
  involves setting a key to a value
  while incrementing the version (or defaulting to 0).
  
  
%%%%%%%%%%%%%%%%%%%%%%%%%%%%%%%%%%%%%%%%%%%%%%%%%%%%%%%%%%%% 
% Subsection: Verilog Semantics
%%%%%%%%%%%%%%%%%%%%%%%%%%%%%%%%%%%%%%%%%%%%%%%%%%%%%%%%%%%% 
\subsection{Imported Verilog Semantics}
\label{sec:verilog-semantics}

\begin{figure}
% This line is causing a font error.
%\small
\setlength{\grammarparsep}{3pt plus 1pt minus 1pt} % increase separation between rules
\setlength{\grammarindent}{5em} % increase separation between LHS/RHS 
\small{}
\begin{grammar}

<expr> ::= signal-const | signal-var  
\alt <unop> <expr> 
        \alt <binop> <expr> <expr>
        \alt <ternop> <expr> <expr> <expr>
        
<unop> ::=  `not' | `neg' | `reduce_and' | `reduce_xor'
    
<binop> ::= `and'  | `or' | `xor'
    \alt  `eq' | `ne' | `logic_and' | `logic_not' | `logic_or' 
    \alt  `mul' | `add' | `sub' 
    \alt  `dff'
    \alt `concat'

<ternop> ::=  `mux'  | `extract'

\end{grammar}

\caption{
Syntax for imported Verilog.
This syntax is largely drawn from
  Yosys's~\cite{wolf2013yosys} internal syntax.
}
\label{fig:verilog-syntax}
\end{figure}

% Stateful
% $ff
% OR should we do $dff?
% Probably $dff. The difference is that one gets a clock and the other doesn't. That is, dff takes D and CLK inputs and has Q output, ff has just D and Q.

% Combinational:
% $add        
% $and        
% $eq         
% $eqx        
% $logic_and  
% $logic_not  
% $logic_or   
% $mul        
% $mux        
% $ne         
% $not        
% $or         
% $reduce_and 
% $reduce_bool
% $reduce_or  
% $reduce_xor 
% $sub        
% $xor        

\begin{figure*}

\mbox{
\inference{
e \Downarrow (v, s)
}{
\texttt{not}(e) \Downarrow (\sim{} v, s)
}[\textbf{not}]
}
% \mbox{
% \inference{
% e \Downarrow (v, s)
% }{
% \texttt{neg}(e) \Downarrow (-v, s)
% }[\textbf{neg}]
% }
\mbox{
\inference{
e \Downarrow \left(\langle b_0, \dots, b_n \rangle, s\right)
}{
\texttt{reduce\_or}(e) \Downarrow \left(\langle b_0 \:|\: \dots \:|\: b_n \rangle, s\right)
}[\textbf{redor}]
}
\quad
\mbox{
\inference{
e_0 \Downarrow (v_0, s_0)
\quad
e_1 \Downarrow (v_1, s_1)
& 
s = s_0 \cup s_1
}{
\texttt{and}(e_0, e_1) \Downarrow \left(v_0 \:\&\: v_1, s\right)
}[\textbf{and}]
}%

\bigbreak

\mbox{
\inference{
e_0 \Downarrow (v_0, s_0)
\quad
e_1 \Downarrow (v_1, s_1)
& 
s = s_0 \cup s_1
}{
\texttt{eq}(e_0, e_1) \Downarrow \left(
\langle 1 \rangle~\text{if}~v_0 = v_1~\text{else}~\langle 0 \rangle, s\right)
}[\textbf{eq}]
}%
\quad
\mbox{
\inference{
e_0 \Downarrow (v_0, s_0)
\quad
e_1 \Downarrow (v_1, s_1)
\quad
e_s \Downarrow (v_s, s_s)
& 
s = s_0 \cup s_1 \cup s_s
}{
\texttt{mux}(e_s, e_0, e_1) \Downarrow \left(
v_0~\text{if}~v_s = \langle 0 \rangle~\text{else}~v_1, s\right)
}[\textbf{mux}]
}%

% \mbox{
% \inference{
% e_0 \Downarrow (v_0, s_0)
% & 
% e_1 \Downarrow (v_1, s_1)
% & 
% s = s_0 \cup s_1
% }{
% \texttt{or}(e_0, e_1) \Downarrow \left(v_0 \:|\: v_1, s\right)
% }[\textbf{or}]
% }%
\quad

\bigbreak

% DFF without clock tick
\mbox{
\inference{
    e =
    \texttt{dff}(e_\mathit{clk}, e_d)
    &
    e_\mathit{clk} \Downarrow 
        \left(\mathit{clk},
        s_\mathit{clk}\right)
    &
    e_d \Downarrow 
        \left(d, s_d\right)
    \\
    s = s_\mathit{clk} \cup s_d
    % &
    % \left(q', d', \mathit{clk}'\right)
    %     =
    %     \left(
    %     s[e], s[e_d], 
    %         s[e_\mathit{clk}]
    %     \right)
    % \\
    &
    \left(s[e_\mathit{clk}],
            \mathit{clk} \right)
    \neq
    \left( 0, 1 \right)
}{
    e 
    \Downarrow
    \left(
        s[e],
        s[
            e := s[e],
            e_\mathit{clk} :=
            \mathit{clk},
            e_d := d]
    \right)
}
[\textbf{dff-hold}]
}%
\quad
% DFF with clock tick
\mbox{
\inference{
    e =
    \texttt{dff}(e_\mathit{clk}, e_d)
    &
    e_\mathit{clk} \Downarrow 
        \left(\mathit{clk},
        s_\mathit{clk}\right)
    &
    e_d \Downarrow 
        \left(d, s_d\right)
    \\
    s = s_\mathit{clk} \cup s_d
    &
    % \left(q', d', \mathit{clk}'\right)
    %     =
    %     \left(
    %     s[e], s[e_d], 
    %         s[e_\mathit{clk}]
    %     \right)
    % \\
    \left( s[e_\mathit{clk}],
            \mathit{clk} \right)
    =
    \left( 0, 1 \right)
}{
    e 
    \Downarrow
    \left(
       s[e_d],
        s[
            e :=s[e_d],
            e_\mathit{clk} :=
            \mathit{clk},
            e_d := d]
    \right)
}
[\textbf{dff-tick}]
}

\caption{
A representative set
  of the semantics rules
  for the subset of Verilog
  supported by \lr.
$s_0\cup s_1$ and $s[k:=v]$ 
  represent state map merging
  and updating, respectively.
}\label{fig:verilog-semantics}

\end{figure*}



\lr extracts semantics from Verilog,
  both from the user-provided Verilog file
  to be compiled,
  and the vendor-supplied FPGA primtive
  simulation models.
The process for extracting these semantics
  is described in detail in
  \cref{sec:implementation:importing-semantics}.
In this section, we formalize 
  the semantics
  which result from this extraction process.
While this formalization
  is a contribution of this paper,
  the underlying semantics
  of the subset of Verilog
  \lr supports
  are designed and implemented by
  the Yosys synthesis tool~\cite{wolf2013yosys}.

\cref{fig:verilog-syntax}
  lists the syntax for the subset of Verilog 
  supported for import by \lr,
  while \cref{fig:verilog-semantics}
  defines its semantics.
Imported Verilog expressions evaluate to \signal{}s.
In general, the imported Verilog expressions
  cover a wide swath
  of bitvector operations,
  including bitwise operations,
  arithmetic operations,
  comparisons,
  and muxes.

\texttt{dff} represents
  a D flip-flop: a state element
  with a clock and data input,
  which stores the value on the data input
  when the clock changes from low to high.
The semantics for this expression
  are given by \textbf{dff-hold}
  and \textbf{dff-tick}.
\textbf{dff-hold}
  describes 
  the case when the clock doesn't tick,
  i.e., there's no \textit{positive edge}
  going from 0 to 1.
In this case,
  the flip flop expression holds
  its previous value $q'$,
  which it gets from the state map.
\textbf{dff-tick}
  describes the case in which
  the clock ticks,
  and thus the expression
  takes on the new value $d'$.
Note, crucially,
  that the value becomes $d'$
  (the \textit{previous} value of $d$,
  read from the state map)
  and not the current value of $d$.\footnote{
  These semantics are defined by
  Yosys's \texttt{clk2fflogic} pass.
}


The \texttt{dff} expression is the 
  only expression
  in both Verilog and \lrir semantics
  that updates the state map.
All other expressions
  are \textit{combinational},
  meaning they do
  not alter state.
% This is significant as it allows us
%   to simplify our mental model of 
%   imported Verilog and \lrir semantics.
% Knowing that \texttt{dff} is the only
%   expression to update the state map,
%   we can interpret
%   all other expressions
%   in the imported Verilog
%   and \lrir
%   as being purely combinational.
% (Note that expressions generally \textit{merge}
%   state maps, seen in most rules in 
%   \cref{fig:verilog-semantics,fig:lrir-semantics},
%   but crucially, only \texttt{dff} \textit{updates}
%   values in the map).
  


%%%%%%%%%%%%%%%%%%%%%%%%%%%%%%%%%%%%%%%%%%%%%%%%%%%%%%%%%%%% 
% Subsection: Spec IR and LRIR
%%%%%%%%%%%%%%%%%%%%%%%%%%%%%%%%%%%%%%%%%%%%%%%%%%%%%%%%%%%% 
\subsection{\lrir}
\label{subsec:spec-ir-and-lrir}

% \lr defines two Intermediate Representations (IRs),
%   \Spec IR and \lrir;
%   syntax for both is defined
%   in \cref{fig:lakeroad-dsl-grammar}.
% \lr uses \Spec IR
%   to give specifications for
%   instructions to synthesize.
% Terms in \Spec IR are either
%   bitvector constants,
%   variables which represent
%   inputs to the instruction,
%   or operators
%   applied to smaller terms.
% \Spec IR is a subset of the SMT-LIB language~\cite{smtlib},
%   from which it derives its semantics.


%%%%%%%%%%%%%%%%%%%% Lakeroad DSL Grammar %%%%%%%%%%%%%%%%%%%%
\begin{figure}
% This line is causing a font error.
%\small
\setlength{\grammarparsep}{3pt plus 1pt minus 1pt} % increase separation between rules
\setlength{\grammarindent}{5em} % increase separation between LHS/RHS 
\small{}
\begin{grammar}

<\lowlrir> ::= `prim' name <list> <\lowlrir>
       \alt signal-const | signal-var  | <list> 
       \alt <signal-op> | <list-op>

<list>  ::= [<\lowlrir>\ldots]

<signal-op> ::= `extract' int int <\lowlrir>
         \alt `concat' <list>
%         \alt `zero-extend' int <lrir>

<list-op> ::= `list-ref' int <\lowlrir>
         % \alt `take' int <\lowlrir>
         % \alt `drop' int <\lowlrir>


% <routing> ::= `bitwise' | `bitwise-reverse'

% This isn't used
% <fpga-fn> ::= `ecp5-lut4' \ |\ \ldots

<sketch> ::= <\lowlrir> | <hole>

<hole> ::= `(' `bitvec' width `)' 
    \alt `(' `choose' <sketch> \ldots `)'
         
\end{grammar}

\caption{Syntax for \lrir.}
\label{fig:lakeroad-dsl-grammar}
\end{figure}
%%%%%%%%%%%%%%%%%%%% END LR DSL GRAMMAR %%%%%%%%%%%%%%%%%%%%

\begin{figure*}

\newcommand{\lrirSemanticsVspace}{5mm}

% \inference{\textrm{bv} ~ v}{v}[\lrir-bv]
% 
% \vspace{\lrirSemanticsVspace}

% \begin{subfigure}{0.32\textwidth}
%   \centering
  
%%%%%%%%%%%%%%%%%%%%%%%%%%%%%%%%%%
% LRIR-list
\mbox{
\inference{e_0 \Downarrow v_0 
   ~ \cdots ~ e_n \Downarrow v_n
}
          {     \left[ e_0,\, \ldots,\, e_n\right] 
                \Downarrow 
                \left[ v_0,\, \ldots,\, v_n\right] }
          [{\bf list}]}%
\quad
\mbox{\inference{l \Downarrow \lbrack v_0 \ldots v_i \ldots v_n\rbrack}
          {\texttt{list-ref}\ l\ i 
           \Downarrow 
           v_i}
          [{\bf list-ref}]
}
%%%%%%%%%%%%%%%%%%%%%%%%%%%%%%%%%%
% LRIR-prim
\quad
\mbox{
\inference%
  {e \Downarrow v 
  }
  {\texttt{prim}\ \rho\ \kappa\ e \Downarrow \Gamma(\rho)(\kappa)(v)}
  [{\bf prim}]
}

  \vspace{\lrirSemanticsVspace}

%%%%%%%%%%%%%%%%%%%%%%%%%%%%%%%%%%
% LRIR-extract
\mbox{
\inference{e \Downarrow \left(\langle b_0 \ldots b_i \ldots b_j \ldots b_n\rangle, s \right)}
          {\texttt{extract}\ i\ j\ e \Downarrow \left( \langle b_i \ldots b_j \rangle, s \right)}
          [{\bf extract}]
}%
\quad
%%%%%%%%%%%%%%%%%%%%%%%%%%%%%%%%%%
% LRIR-concat
\mbox{
\inference{   e_0 \Downarrow \left(\langle b^0_0\ldots b^0_{n_1}\rangle, s_0 \right)
              ~ \cdots ~
              e_k \Downarrow \left( \langle b^k_0\ldots b^k_{n_k} \rangle, s_n\right)
& s = \bigcup^i s_i
              }
          {\texttt{concat}\ e_0,\,\ldots,\, e_k
           \Downarrow 
           \left(\langle b^0_0 \ldots b^0_{n_1} 
           \ldots
           b^k_0 \ldots b^k_{n_k} \rangle, 
           s\right)}
          [{\bf concat}]
}


% \end{subfigure}%
% \hspace{5mm}%
% \begin{subfigure}{0.25\textwidth}


%%%%%%%%%%%%%%%%%%%%%%%%%%%%%%%%%%
% LRIR-take
% \inference{l \Downarrow \lbrack v_0 \ldots  
%                                 v_n\rbrack
%             & 0 \leq i \leq n+1}
%           {\texttt{take}\ l\ i 
%            \Downarrow 
%            \lbrack v_0,\, \ldots,\, v_{i-1} \rbrack}
%           [{\bf take}]
% \vspace{\lrirSemanticsVspace}

% \end{subfigure}\hspace{5mm}%
% \begin{subfigure}{0.32\textwidth}
%   \centering

          
%%%%%%%%%%%%%%%%%%%%%%%%%%%%%%%%%%
% LRIR-drop
% \inference{l \Downarrow \lbrack v_0 \ldots v_n\rbrack
%            & 0 \leq i \leq n+1 }
%           {\texttt{drop}\ l\ i 
%            \Downarrow 
%            \lbrack v_i,\, \ldots,\, v_{n} \rbrack}
%           [{\bf drop}]
% \vspace{\lrirSemanticsVspace}


          
% \end{subfigure}%


% \begin{subfigure}{.5\textwidth}
% \end{subfigure}%
% \hspace{5mm}%
% \begin{subfigure}{0.4\textwidth}
% \centering
%%%%%%%%%%%%%%%%%%%%%%%%%%%%%%%%%%
% LRIR-bitwise-phys-to-log
% \inference%
%   {l \Downarrow 
%          \lbrack 
%           \langle bv^0_0,\, \ldots,\,  bv^0_k\rangle,\,
%           \cdots,\,
%           \langle bv^n_0,\, \ldots,\,  bv^n_k\rangle
%          \rbrack}
%   {\texttt{wire}\ \ \texttt{bitwise}\ \ l
%    \Downarrow
%       \lbrack 
%        \langle bv^0_0,\, \ldots,\,  bv^n_0\rangle,\,
%        \cdots,\,
%        \langle bv^0_k,\, \ldots,\,  bv^n_k\rangle
%       \rbrack}
%   [{\bf bw-wire}]
% \vspace{\lrirSemanticsVspace}

% \end{subfigure}

\hspace{30mm}\begin{subfigure}{0.9\textwidth}

%%%%%%%%%%%%%%%%%%%%%%%%%%%%%%%%%%
% LRIR-bwrev-phys-to-log
% \inference%
%   {l \Downarrow 
%          \lbrack 
%           \langle bv^0_0,\, \ldots,\,  bv^0_k\rangle,\,
%           \cdots,\,
%           \langle bv^n_0,\, \ldots,\,  bv^n_k\rangle
%          \rbrack}
%   {\texttt{wire}\ \ \texttt{bitwise-reverse}\ \ l
%    \Downarrow
%       \lbrack 
%        \langle bv_k^0,\, \ldots,\,  bv_k^n\rangle,\,
%        \cdots,\,
%        \langle bv_0^0,\, \ldots,\,  bv_0^n\rangle
%       \rbrack}
%   [{\bf bwrev-wire}]
% \vspace{\lrirSemanticsVspace}

\end{subfigure}
\caption{\lrir semantics. Angled brackets indicate bitvector literals, while square brackets indicate list literals.}
\label{fig:lrir-semantics}
\end{figure*}



% %%%% TRY 2 %%%%%
% \begin{figure*}
% \newcommand{\lrirSemanticsVspace}{5mm}

% \begin{subfigure}{0.3\textwidth}
%   \centering
% %%%%%%%%%%%%%%%%%%%%%%%%%%%%%%%%%%
% % LRIR-list
% \inference{e_0 \Downarrow v_0 & \cdots & e_n \Downarrow v_n}
%           {     \left[ e_0,\, \ldots,\, e_n\right] 
%                 \Downarrow 
%                 \left[ v_0,\, \ldots,\, v_n\right]}
%           [{\bf list}]
% \vspace{\lrirSemanticsVspace}

% %%%%%%%%%%%%%%%%%%%%%%%%%%%%%%%%%%
% % LRIR-extract
% \inference{e \Downarrow \langle b_0 \ldots b_i \ldots b_j \ldots b_n\rangle}
%           {\texttt{extract}\ i\ j\ e \Downarrow \langle b_i \ldots b_j \rangle}
%           [{\bf extract}]
% \vspace{\lrirSemanticsVspace}
% \end{subfigure}
% \begin{subfigure}{0.3\textwidth}
%   \centering
% %%%%%%%%%%%%%%%%%%%%%%%%%%%%%%%%%%

% % LRIR-prim
% \inference%
%   {e \Downarrow v 
%   }
%   {\texttt{prim}\ \rho\ \kappa\ e \Downarrow \Gamma(\rho)(\kappa)(x)}
%   [{\bf prim}]
% \vspace{\lrirSemanticsVspace}

% %%%%%%%%%%%%%%%%%%%%%%%%%%%%%%%%%%
% % LRIR-list-ref

% \inference{l \Downarrow \lbrack v_0 \ldots v_i \ldots v_n\rbrack}
%           {\texttt{list-ref}\ l\ i 
%           \Downarrow 
%           v_i}
%           [{\bf list-ref}]
% \vspace{\lrirSemanticsVspace}
% \end{subfigure}
% \begin{subfigure}{0.3\textwidth}
% %%%%%%%%%%%%%%%%%%%%%%%%%%%%%%%%%%
% % LRIR-take
% \inference{l \Downarrow \lbrack v_0 \ldots  
%                                 v_n\rbrack
%             & 0 \leq i \leq n+1}
%           {\texttt{take}\ l\ i 
%           \Downarrow 
%           \lbrack v_0,\, \ldots,\, v_{i-1} \rbrack}
%           [{\bf take}]
% \vspace{\lrirSemanticsVspace}
          
% %%%%%%%%%%%%%%%%%%%%%%%%%%%%%%%%%%
% % LRIR-drop
% \inference{l \Downarrow \lbrack v_0 \ldots v_n\rbrack
%           & 0 \leq i \leq n+1 }
%           {\texttt{drop}\ l\ i 
%           \Downarrow 
%           \lbrack v_i,\, \ldots,\, v_{n} \rbrack}
%           [{\bf drop}]
% \vspace{\lrirSemanticsVspace}
% \end{subfigure}
% \begin{subfigure}{\textwidth}
% \centering
% %%%%%%%%%%%%%%%%%%%%%%%%%%%%%%%%%%
% % LRIR-concat
% \inference{   e_1 \Downarrow \langle b^1_0\ldots b^1_{n_1}\rangle
%               & \cdots &
%               e_k \Downarrow \langle b^k_0\ldots b^k_{n_k} \rangle}
%           {\texttt{concat}\ e_1,\,\ldots,\, e_k
%           \Downarrow 
%           \langle b^1_0 \ldots b^1_{n_1} 
%           \ldots
%           b^k_0 \ldots b^k_{n_k} \rangle }
%           [{\bf concat}]
% \vspace{\lrirSemanticsVspace}
% \end{subfigure}
% \begin{subfigure}{\textwidth}
% \centering
% \inference%
%   {l \Downarrow 
%          \lbrack 
%           \langle bv^0_0,\, \ldots,\,  bv^0_k\rangle,\,
%           \cdots,\,
%           \langle bv^n_0,\, \ldots,\,  bv^n_k\rangle
%          \rbrack}
%   {\texttt{wire}\ \ \texttt{bitwise}\ \ l
%   \Downarrow
%       \lbrack 
%       \langle bv^0_0,\, \ldots,\,  bv^n_0\rangle,\,
%       \cdots,\,
%       \langle bv^0_k,\, \ldots,\,  bv^n_k\rangle
%       \rbrack}
%   [{\bf bw-wire}]
% \vspace{\lrirSemanticsVspace}
% \end{subfigure}
% %\hspace{30mm}\begin{subfigure}{0.9\textwidth}
% %\centering
% %%%%%%%%%%%%%%%%%%%%%%%%%%%%%%%%%%
% % LRIR-bitwise-phys-to-log
% \begin{subfigure}{0.9\textwidth}
% \centering
% %%%%%%%%%%%%%%%%%%%%%%%%%%%%%%%%%%
% % LRIR-bwrev-phys-to-log
% \inference%
%   {l \Downarrow 
%          \lbrack 
%           \langle bv^0_0,\, \ldots,\,  bv^0_k\rangle,\,
%           \cdots,\,
%           \langle bv^n_0,\, \ldots,\,  bv^n_k\rangle
%          \rbrack}
%   {\texttt{wire}\ \ \texttt{bitwise-reverse}\ \ l
%   \Downarrow
%       \lbrack 
%       \langle bv_k^0,\, \ldots,\,  bv_k^n\rangle,\,
%       \cdots,\,
%       \langle bv_0^0,\, \ldots,\,  bv_0^n\rangle
%       \rbrack}
%   [{\bf bwrev-wire}]
% \vspace{\lrirSemanticsVspace}
% \end{subfigure}
% \caption{\lrir semantics.}
% \label{fig:lrir-semantics}
% \end{figure*}


% \inference{l \in [\textrm{BV}], \textrm{logical-to-physical}~\textrm{bitwise}~l}
% {
% [(\textrm{concat}~(\textrm{extract}~0~0~l[0])~%
% (\textrm{extract}~0~0~l[1])~\dots)~%
% (\textrm{concat}~(\textrm{extract}~1~1~l[0])~%
% (\textrm{extract}~1~1~l[1])~\dots)~\dots ]
% }
% [\lrir-ltop-bitwise]

% \vspace{\lrirSem\lrirSemanticsVspace}

% \inference{l \in [\textrm{BV}], \textrm{logical-to-physical}~\textrm{bitwise-reverse}~l}
% {
% \begin{array}{@{}c@{}}
% [(\textrm{concat}~(\textrm{extract}~n~n~l[0])~%
% (\textrm{extract}~n~n~l[1])~\dots)   \\
% (\textrm{concat}~(\textrm{extract}~n-1~n-1~l[0])~%
% (\textrm{extract}~n-1~n-1~l[1])~\dots)~\dots ]
% \end{array}
% }
% [\lrir-ltop-bitwise-reverse]

% \vspace{\lrirSem\lrirSemanticsVspace}

% \inference
% {\textrm{primitive}~n~e}
% {\textrm{look up primitive semantics for $n$, run on $e$}}

\lr uses the \lr Intermediate Representation (\lrir) to represent
  wire operations and computations on FPGAs.
The syntax for \lrir is shown in
  \cref{fig:lakeroad-dsl-grammar},
  while its semantics are in \cref{fig:lrir-semantics}.
In general, the \lrir syntax and semantics 
  are straightforward.
\lrir expressions evaluate to \signal{}s or
  lists of \signal{}s.
Fundamental operations over lists
  (construction, 
  indexing via \texttt{list-ref})
  and \signal{}s
  (\texttt{concat}, \texttt{extract}) 
  are supported.

The only \lrir expression
  which can perform actual computation
  or manipulate state
  (beyond the state merging in \texttt{concat})
  is \texttt{prim}.
\texttt{prim}
  represent instances
  of \textit{hardware primitives} (e.g., LUTs and DSPs).
A \texttt{prim} has a \textit{name}
  (e.g., ``LUT4''), a \textit{configuration},
  and an \textit{input}.

The semantics of a \texttt{prim} expression
  are the semantics extracted from the
  primitive's vendor-supplied Verilog model,
  as introduced in the previous section.
Evaluating \texttt{prim} expressions
  ``drops through'' to these imported
  semantics.
\lrir stores these semantics in
  a context 
  \[\Gamma: \left(\forall p: \textsc{Prim},
            \textsc{Config}\ p\right)
            \rightarrow [\texttt{signal}] 
            \rightarrow [\texttt{signal}]\]
  that takes a configuration
  for a primitive $p$
  and returns a function from
  \texttt{signal}s to \texttt{signal}s
  that captures the semantics
  of $p$ with that configuration.
A configuration of primitive $p$,
  \textsc{Config} $p$,
  is the list of parameters that
  an instantiation of $p$ expects
  (e.g., a Xilinx Ultrascale+ DSP48E2's
  \texttt{AREG}
  and \texttt{BREG}
  parameters, which configure pipelining).
  
%%%%%%%%%%%%%%%%%%%%%%%%%%%%%%%%%%%%%%%%%%%%%%%%%%%%%%%%%%%% 
% Subsection: Correctness of the LR Function
%%%%%%%%%%%%%%%%%%%%%%%%%%%%%%%%%%%%%%%%%%%%%%%%%%%%%%%%%%%% 
\subsection{Mathematical Model of \lr}
\label{sec:correctness-argument}

\newcommand{\SketchGenFn}{f_{\text{G}\xspace{}}}
\newcommand{\LRSynthFn}{f_{\text{S}\xspace{}}}
\newcommand{\HDLCompFn}{f_{\text{C}\xspace{}}}

Now that we have formalized
  \lr's fundamental components---%
  signals,
   imported Verilog semantics,
   and \lrir---%
  we now formalize the \lr workflow
  as a function \lrfn{}.
We define the function to be
the composition of two sub-functions, which we describe in the below sections:
% \[\lrfn: \Spec \times \Sketch \times \VendorSemantics \rightharpoonup \HDL.\]
$$\lrfn : \Spec \times \Sketch \times \VendorSemantics \times \textsc{Input} \rightharpoonup \HDL = \HDLCompFn \circ \LRSynthFn$$
where $\Spec$ represents hardware designs, $\Sketch$ represents program sketches, $\VendorSemantics$ represents vendor-provided semantics for FPGA components, and $\textsc{Input}$ represents partially symbolic inputs (discussed below).

\lrfn does not include sketch generation,
  instead taking the sketch as an input.
This is for the benefit of stating the correctness properties:
  There is no \textit{a priori} notion of the ``correctness'' of a sketch---%
  a sketch is ``correct'' 
  so long as program synthesis succeeds 
  in filling the holes in the sketch.
A sketch does, however, influence
  the search space for program synthesis, as we note in \hl{todo fill in forward reference}.
  
% This is because our correctness properties
%     state that any synthesized program is correct.
% If an incorrect sketch is generated then synthesis
%     will fail and the correctness
%     properties will hold vacuously.

% \hl{New user input needed: number of clock cycles, among other things?}

% \hl{should sems be gamma?}

%\hl{ben is pointing out that these functions don't actually compose in their current states b/c of their input/output types. we could do a few things: (1) change figure 2 so everything goes through sketch generator. (2) have the two differ but acknowledge the difference. or (3) go full formal and have the semantics "pass through"}

%\hl{see notes under figure. we can just drop fg, and mention that fg isn't necessary to correctness}

% %%%%%  Sketch Generation
%
% NOTE: I've removed this subsection, but we can include it
% later if we think we should. It's not part of the lr function
% so if we do include it, it should be put elsewhere!
%
% \subsubsection{\SketchGeneration}
% \lrfn begins by performing \SketchGeneration
%   on each available sketch 
%   template\footnote{While sketch
%   templates can in theory
%   be provided by users, the \lr tool supplies
%   a set of base templates and, for simplicity
%   of the formalization, we treat
%   them as being internal to the tool.}.
% We formalize \SketchGeneration with the function
% \hl{should also include the spec, b/c we need to know the input/output widths}
%   \[\SketchGenFn :\SketchTemplate \times \ArchDescr \rightharpoonup \Sketch,\]
%   where \(\rightharpoonup\)
%   indicates that this operation is partial.
% $\SketchGenFn$ is partial because a sketch template
%   may use a primitive interface that isn't supported
%   by a given architecture's description.
  
%%%%%  LRSynthesis
\subsubsection{\LRSynthesis: $\LRSynthFn$}
\label{sec:lr-synth-defs}
\lrfn begins by attempting to synthesize an \lrir{} program
  that implements a provided specification using
  a program sketch.
We formalize this component with 
  the $\LRSynthFn{}$ function:
  \[\LRSynthFn: \Sketch \times \Spec \times \VendorSemantics \times \textsc{Input} \rightharpoonup \lrir.\]
$\LRSynthFn$ is partial since
  synthesis may not find an
  implementation for a spec
  using the given sketch---%
  Rosette may conclude that no implementation exists
  or it may time out.
%\hl{Maybe add signal 
  %mention in this section 3.2.2.}
  
\newcommand{\lrinterp}[0]{\mathcal{I}_{Ax}}

We formalize synthesis
  in terms of an interpretation
  function 
  $\lrinterp$,
  which expects
  an \lrir program $p$,
  vendor-provided semantics $e$ 
  (used for interpreting the
  semantics of any \texttt{prim}
  constructs in $p$),
  a vector $\mathbf{i}$ of
  partially symbolic values,
  and a partially symbolic
  state map $m$
  as input values,
  and produces a partially
  symbolic signal as output.
A \textit{partially
  symbolic signal} is a pair $(v, m)$
  of a partially symbolic value $v$
  and a partially symbolic
  state map $m$.
A \textit{partially symbolic value} $v$
  is either a concrete bitvector
  or a symbolic bitvector.
A \textit{partially symbolic map} $m$
  is a map from concrete keys
  to partially symbolic values.
% For a symbolic bitvector $v$,
%   $f(v) = g(v)$ means that $\forall v^\prime, f(v^\prime) = g(v^\prime)$,
%   where $v^\prime$ is a concrete bitvector;
%   that is, we quantify 
%   over all possible instantiations 
%   of the symbolic values.

%\hl{we need to get $(v,m)$/$(v,s)$ consistent across the paper. $s=(v,m)$ is probably better. Needs to be propagated to first half of formalization. -gs}


We define $\lrinterp$ recursively:
\begin{align*}
    \lrinterp(p, e, [\mathbf{v_1}], m) &= \llbracket p \rrbracket (\mathbf{v_1}, m) \\
    \lrinterp(p, e, [\mathbf{v_1}, \mathbf{v_2}, \ldots, \mathbf{v_n}], m)  &= \lrinterp(p, e, \\
    & \hspace{15mm} [\mathbf{v_2}, \ldots, \mathbf{v_n}], \pi_S(\llbracket p \rrbracket (\mathbf{v_1}, m)))
    % a little nasty but it breaks margins otherwise
\end{align*}
% $$\lrinterp(p, e, [\mathbf{v_1}]) = \llbracket p \rrbracket (\mathbf{v_1})$$
% $$\lrinterp(p, e, [\mathbf{v_1}, \mathbf{v_2}, \ldots, \mathbf{v_n}]) = \lrinterp(p, e, [S_{\mathbf{v_1}}(\mathbf{v_2}), \ldots, S_{\mathbf{v_1}}(\mathbf{v_n})])$$
%\hl{i think we can make these prettier -- perhaps use aligned env?}
  where $\llbracket p \rrbracket(\textbf{v})$ evaluates $p$ 
  with respect to its semantics 
  (Figure~\ref{fig:lrir-semantics},
  substituting the elements of $\mathbf{v}$ for each \texttt{signal-var} in $p$), and $\pi_S$ projects
  the state map from a signal.
We similarly define $\mathcal{I}_V(d, \mathbf{i}, m)$,
  which behaves identically to $\lrinterp$,
  except that it takes as input a Verilog design $d$ 
  ($\llbracket d \rrbracket$
  evaluates the design
  according to the Verilog semantics
  in Figure~\ref{fig:verilog-semantics})
  and it does not use $e$,
  since there are no
  \texttt{prim} constructs in Verilog.

Given a hardware design $d \in \Spec$,
  vendor-provided semantics $e \in \VendorSemantics$,
  $s \in \Sketch$,
  and a input vector of partially symbolic values,
  $\LRSynthFn$ is defined as sending the following query to Rosette:
$$\exists p \in \lrir\ \text{s.t.}\ \lrinterp(p, e, \mathbf{i}) = \mathcal{I}_V(d, \mathbf{i}).$$
Synthesis succeeds if Rosette
  finds a program $p$ 
  (constructed by filling the holes in $s$)
  that satisfies this property.
How the input 
  $\mathbf{i}$
  is constructed 
  is an important design decision;
  we describe the specific decisions
  made by \lr
  in 
  \cref{sec:implementation:program-synthesis}.
(Note that our formalization has no need
  to ``privilege'' the clock---%
  it can be treated like any other signal.)

% [[d]]([t]) = [[d]]'(t)
% [[d]](s) = s
% Interp(e, i (vector)) each elem of i = tuple of signals
% Interp(e, [] ::' i) = [[e]](i)
% Interp(e, is ::' i) = let (_, s) = Interp(e, is) in
%   Interp(e, [i.update(s)]
% update: update((sig0, sig1, sig2), s) -> overwrites sig0, sig1, sig2 states with s
% vector i: list of concrete length
% element of vector i: tuple of concrete length
% element of that tuple: signal
% that signal: concrete or symbolic value, 
%   state map has concrete keys, concrete or symbolic values

% symbolic inputs are just variables, always bitvectors
% if Rosette succeeds, then it's quantified for all symbolic values


% The $\LRSynthFn{}$ takes
%   $(d, n) \in \Spec$ and attempts
%   to synthesize a program in \lrir
%   with the semantics of $d$.
% When $d$ is sequential, successful
%   synthesis yields a program that
%   implements $d$ after $n$ clock cycles.
% \hl{Ben: \@Gus I'll need you to give more detail on this}
% $\LRSynthFn{}$ does this by applying the
%   interpreter to the sketch multiple times
%   to simulate each cycle of the clock.
% We can treat a sketch $s$ as a function that
%   outputs values on the wire and 
%   updated state (e.g., incremented clock).
% Then Synthesis tries to find an instantiation
%   $S$ of $s$ such that $S^n(i)$
%   implements $d(i)$.
% \hl{I'm not saying the above correctly, we need
%   fix! But this is the basic idea}.
% Here we use the notation $S^n(i)$ to mean
%   the following:
%   \begin{itemize}
%       \item $S^1(i, \sigma_0)$ is defined to  be
%         the 2-tuple $(v, \sigma_1)$
%         of the output value $v$
%         and state $\sigma_1$
%         resulting from applying
%         $S$ to input $i$
%         with initial state $\sigma_0$.
%       \item $S^{n+1}(i, \sigma_0)$ is defined
%         to be the 2-tuple $(v, \sigma_{n+1})$ 
%         of the output value $v$ and the
%         state $\sigma_{n+1}$ resulting from
%         applying $S$ to input $0$ and state
%         $\sigma_n$.
%   \end{itemize}
%   \hl{Ben: This is not quite what we discussed, but I think it's correct for the formalism. We can go into the more detailed approach w/ multiple interpreter calls in implementation.

% I'm also happy to go into more detail here, but I want to start simple and see what this misses.}
  
% When $d$ is combinational,
%   successful synthesis yields a 
%   program that implements the
%   semantics of $d$ being run a
%   single time (no updated state or
%   clock cycles).


%%%%%  HDL Compilation
\subsubsection{\HDLCompilation}
Next, \lrfn compiles each synthesized design
  to an HDL. This is formalized with the function
  \[\HDLCompFn: \lrir \rightarrow \HDL.\]
This compilation is done as a very 
  straightforward syntactic mapping:
  each component
  of \lrir maps directly to the
  resulting Verilog implementation.

% \subsubsection{Combining Components into \lrfn}
%
% NOTE: We no longer need this section
%
% The \lrfn function is built by combining the three above components, as sketched in the following pseudocode:

% \lstset{style=pystyle}
% \begin{lstlisting}[basicstyle=\normalsize\ttfamily]
% def LR(spec, arch, sems):
%   hdl_impls = {}
%   for t in sketch_templates:
%     sketch = generate_sketch(t, arch)
%     lrir = synth_lrir(spec, sketch, sems)
%     if lrir is not None:
%         hdl_impls.add(compile2hdl(lrir))
%   return hdl_impls
% \end{lstlisting}
% \lstset{style=lispstyle}

  

%%%%%%%%%%%%%%%%%%%%%%%%%%%%%%%%%%%%%%%%%%%%%%%%%%%%%%%%%%%% 
% SubSection: Correctness of LRFN Function
%%%%%%%%%%%%%%%%%%%%%%%%%%%%%%%%%%%%%%%%%%%%%%%%%%%%%%%%%%%% 
\subsection{Properties of \lrfn{}}

From the above definitions,
  we can prove certain correctness properties
  about \lrfn{} and $\LRSynthFn$
  if we make certain assumptions about our some components of \lr (our trusted computing base):
\begin{enumerate}
\item \textbf{Rosette}: We assume that Rosette is correct (if it returns a program for a synthesis query, the program satisfies the specification) and sketch-complete modulo timeout (given a sketch, if there exists any solution using that sketch, Rosette will either find a solution or time out---it will not falsely conclude that no solution exists).
\item \textbf{Yosys}: We assume that Yosys's conversion of Verilog into bitvector expressions is correct.
\item \textbf{Vendor-supplied semantics}: We assume that these Verilog files correctly model the semantics of the FPGA components they model.\footnote{
  Note that this assumption may not always be true,
    as we will see in
    \cref{sec:completeness}.
}
\item \textbf{The implementation of $\lrinterp$}: In synthesis, we include an interpreter for the semantics of \lrir. We assume that our implementation correctly implements the claimed semantics in Figure~\ref{fig:lrir-semantics}.
\item $\mathbf{\HDLCompFn}$: We assume that our compilation from \lrir to Verilog is correct and complete. We do not prove this, though it could be done in principle via the semantics in Figures~\ref{fig:lrir-semantics} and~\ref{fig:verilog-semantics}; we rely instead on the simplicity of the implementation.
\end{enumerate}

\subsubsection{Correctness of Synthesis}

We can state this property as follows: 
If $\LRSynthFn(d, s, e, \mathbf{i}) = p$
  for design $d$, 
  sketch $s$,
  vendor-provided semantics $e$,
  and a partially symbolic input $\mathbf{i}$, 
  then for all inputs $\mathbf{i}^\prime$ 
  that are valid concrete instantiations of $\mathbf{i}$,
  $\mathcal{I}_V(d, \mathbf{i}^\prime) = \lrinterp(p, e, \mathbf{i}^\prime)$.
This follows directly from our assumptions about the correctness of Rosette and the correctness of Yosys (which is used to implement $\mathcal{I}_V$) and the definitions in \cref{sec:lr-synth-defs}.

\subsubsection{Sketch-Completeness of Synthesis Modulo Timeout}

Given design $d$,
  sketch $s$,
  vendor-provided semantics $e$,
  and a partially symbolic input $\mathbf{i}$,
  if there exists an instantiation of $s$ called $p$ for which $\mathcal{I}_V(d, \mathbf{i}^\prime) = \lrinterp(p, e, \mathbf{i}^\prime)$ 
  for all
  valid instantiations $\mathbf{i}^\prime$ of $\mathbf{i}$, 
  then $\LRSynthFn(d, s, e, \mathbf{i})$ 
  will either return some value or time out.
This follows from the sketch-completeness (modulo timeout) of Rosette.

\subsubsection{Correctness and Completeness of \lrfn{}}

Given the above properties of $\LRSynthFn$
  and our assumption of the correctness and completeness of $\HDLCompFn$,
  it follows that their composition will have analogous properties:
For design $d$,
  sketch $s$,
  vendor-provided semantics $e$,
  and a partially symbolic input $\mathbf{i}$,
  if $\lrfn(d, s, e, \mathbf{i}) = d^\prime$, then $\mathcal{I}_V(d, \mathbf{i}^\prime) = \mathcal{I}_V(d^\prime, \mathbf{i}^\prime)$ for all valid instantiations $\mathbf{i}^\prime$ of $\mathbf{i}$.
Additionally, with $d$, $s$, $e$, and $i$,
  if there exists some instantiation $p$ of $s$ for which
  $\mathcal{I}_V(d, \mathbf{i}^\prime) = \lrinterp(p, e, \mathbf{i}^\prime)$ 
  for all
  valid instantiations $\mathbf{i}^\prime$ of $\mathbf{i}$,
  then $\lrfn(d, s, e, \mathbf{i})$
  will either return a value or time out.

% \subsection{Correctness of the \lrfn{} Function}

% \hl{Gus:Enumerate the 
%  TCB,
%  if we really want to show
%  that we're serious
%  about correctness Solver, semantics importer (a number of components within Yosys + our $\sim$300 line btor to Racket compiler)}
%  \hl{Ben: The TCB is roughly enumerated after Property 3 is listed.}

% The \lr-Correct property says that
%   for instruction specification $s$, 
%   HDL program $v$,
%   architecture description $d$,
%   vendor-supplied semantics $e$,
%   and inputs $i$,
%   if synthesis and HDL compilation
%   with \lrfn on the instruction
%   specified by $s$
%   and architecture described by $d$
%   results in a HDL implementation $v$,
%   then simulating $v$ on inputs $i$
%   produces the same result as evaluating $s$
%   on inputs $i$. \hl{Should we talk here about how signals aren't relevant to the output comparison, just a way for interpretation to get to the output?}\\

% \newtheorem{property}{Property}
% \begin{property}[\lr-Correct]
% For all \(s \in \Spec\),
%         \(v \in \HDL\),
%         \(d \in \ArchDescr\),
%         \(e \in \VendorSemantics\)
%         and inputs \(\mathbf{i}\)
%   \[\lrfn(s,d,e) = v \implies 
%      \denote{s}_{\text{bv}}(i) 
%          = 
%      \denote{v}_{\text{HDL}}(i).\]
% \end{property}

% \hl{should we have
% (flr(...) = fc(fs(...)) = ...? or just one of those things? in which case, which one?}

% The commutative diagram in \cref{fig:correctness-comm-diagram}
%   shows how the \lr-Correct property
%   is composed from the
%   the Synth-Correct (bottom left triangle) and
%   Compile-Correct (bottom right triangle) properties,
%   which we describe below.
  
% %%%%%%%%%%%%%%%%%%%%%%%%%%%%%%%%%%%%%%%%%%%%%%%%%%%%%
% % COMMUTATIVE DIAGRAM FIGURE
% 
\begin{figure}
\small
\adjustbox{scale=0.85,center}{%
\begin{tikzcd}
    \Spec \times \Sketch \times \VendorSemantics
       % LR
       \arrow[rrrrr, rightharpoonup, "{\lrfn}" description, bend left=30]
       % Run Spec
       \arrow[rrddd, "{\denote{s}_{\text{bv}}(i)}" description] 
       % Run Synth
        \arrow[rr, "\LRSynthFn", rightharpoonup]
      & & \lrir \arrow[rrr, "\HDLCompFn"]
                 \arrow[ddd, "{\denote{p}_{\lrir}(i)}" description]
      & & & \HDL\arrow[lllddd, "{\denote{v}_{\HDL}(i)}" description] \\
    & & & & &  \\
    & & & & &  \\
    & &  \textsc{Val} & & & 
\end{tikzcd}}
\caption{
  \lr Correctness Properties:
  {\small 
  The Synth-Correct Property (left triangle)
    running an \lrir program
    synthesized from a spec $s$
    has the same result
    as running the spec program directly.
  The Compile-Correct Property (right triangle)
    says that
    running an \lrir program
    is the same as compiling it to an HDL
    and simulating the HDL program.
The \lr-Correct Property
  combines the Synth-Correct Property
  and the Compile-Correct Property
  to provide a correctness guarantee for \lr.}}
\label{fig:correctness-comm-diagram}
\hl{fg is not in the diagram---how do we add?}
\hl{above specxsketchxsems, instead we have spec and sems going into that existing node, and sketch being generated by fg}


\hl{Another solution here (maybe what we currently have...) is that we just don't consider sketch generation in correctness, because it doesn't have an effect. then we would need to get rid of sketch generation section and all mentions of fg. we need to make sure to indicate why we can ignore sketch generation}
\end{figure}

% %%%%%%%%%%%%%%%%%%%%%%%%%%%%%%%%%%%%%%%%%%%%%%%%%%%%%
% % Synth Correct Theorem Section
% The Synth-Correct property states
%   that when performing \LRSynthesis on a specification produces
%   a result, that result is correct.\\


% \hl{
% TODO:
% we have a big down arrow for what lakeroad program evals to over one timestep, we need a big down arrow that represents time stepping over multiple timesteps.}

% %%%%%%%%%%%%%%%%%%%%%%
% % Synth Correct Theorem
% \begin{property}[Synth-Correct]
%   For all \(k \in \Sketch\),
%   \(s \in \textsc{Spec}\),
%   \(c \in \ArchDescr\),
%   \(e \in \VendorSemantics\),
%   \(p \in \lrir\),
%   and inputs \(\mathbf{i}\),
%   \[\LRSynthFn(k, s, e) = p \implies \denote{s}_{bv}(i) = \denote{p}_{\lrir}(e)(i)\]
% \end{property}
% \label{thm:synth-correct}
% % End Synth Correct Theorem
% %%%%%%%%%%%%%%%%%%%%%%

% %%%%%%%%%%%%%%%%%%%%%%%%%%%%%%%%%%%%%%%%%%%%%%%%%%%%%
% % Compile Correct Theorem Section

% The Synth-Correct property asserts that 
%   the \HDLCompilation component preserves semantics.\\

% %%%%%%%%%%%%%%%%%%%%%%
% % Compile Correct Theorem
% \begin{property}[Compile-Correct]
%   For all \(p \in \textsc{Impl}\) and all inputs \(i\),
%   \[\HDLCompFn(p) = v \implies \denote{p}_\lrir(e)(i) = \denote{v}_\HDL(i).\]
% \end{property}
% \label{thm:compile-correct}
% % End Compile Correct Theorem
% %%%%%%%%%%%%%%%%%%%%%%

% The Synth-Correct property depends on
%   the correctness of the synthesis tool (in our case Rosette~\cite{torlak2014lightweight, torlak2013growing}),
%   the correctness of vendor-supplied HDL FPGA primitive semantics,
%   and the correctness of our \lrir interpreter.
% All three of these pieces are real software and are thus subject
%   to bugs in principle,
%   but all are heavily tested and thus reasonable
%   inclusions in our trusted computing base.
% Rosette in particular is a widely-used
%   by the Programming Languages and Program Synthesis communities.
% \lr's core \lrir interpreter is straightforward and
%   lifts bitvector and list manipulation
%   functions from Rosette and Racket.
% As such its correctness depends primarily on that of the host languages, though we  have also extensively tested it.
% All hardware-specific portions of the \lrir interpreter are imported
%   automatically from vendor-supplied HDL primitive specifications.
% We assume that these specs are correct as they are provided as ground truth
%   by vendors.
% Again, we have also validated this 
%   importing process through
%   extensive testing.
% Finally, the Compile-Correct property relies
%   on the simplicity of \lrir: each component
%   of \lrir maps directly to the
%   resulting Verilog implementation,
%   and compilation is a simple
%   syntactic transformation.


  
%%%%%%%%%%%%%%%%%%%%%%%%%%%%%%%%%%%
%%% GRAVEYARD
%%%%%%%%%%%%%%%%%%%%%%%%%%%%%%%%%%%

% \begin{figure}
% \hl{ben: ltop and ptol should also have some shift functionality}
% \begin{verbatim}
% spec ::= <bv-const> | <bv-var>
%       | <binop> <spec> <spec>
%       | <unop> <spec> <spec>
%       | extract <nat> <nat> <spec>
%       | concat <spec> ...
%       | zero-extend <spec> <nat>
%       | list <spec> ...
%       | list-ref <spec> <nat>

% binop ::= bvand | bvor | bvxor 
%         | bvadd | bvsub | bvmul
        
% unop  ::= bvnot

% impl  ::= bv <bv-const-or-var>
%         | extract <nat> <nat> <impl>
%         | logical-to-physical <ltop> <impl> ...
%         | physical-to-logical <ptol> <impl> ...
%         | primitive <name> <spec> ...

% sketch ::= impl with holes
        
% ltop  ::= bitwise
%         | bitwise-reverse

% ptol  ::= bitwise
%         | bitwise-reverse
% \end{verbatim}
% \caption{Caption}
% \label{fig:syntax}
% \end{figure}
