\chapter{Future Work: Churchroad}
\label{chapter:churchroad}

I wanted to briefly mention
  an ongoing extension to \lr.
A fundamental limitation
  of solver-aided approaches
  like \lr
  is problem size:
  the larger a constraint problem,
  the more likely solvers are to time out
  while trying to find a solution.
In \lr, this means that
  compiling larger hardware designs
  will present issues;
  we already saw in \cref{sec:evaluation}
  that \lr's underlying solvers
  time out
  on some benchmarks.
Furthermore, even some small tasks
  are 
  notably difficult for solvers---%
  namely, reasoning about multiplication~\cite{rath2024polysat,brain2021further}.
Scaling to large designs
  and compiling multipliers
  are core requirements
  for a hardware compiler,
  however.
So how do we ensure that
  we can use \lr
  on large designs?

In response,
  we have been developing Churchroad~\cite[section 4]{smith2024there},
  which combines many of the techniques from this dissertation.
Namely, we employ
  \gls{equality-saturation}---%
  our algorithm of choice in \cref{part:glenside-and-3la}.
Using equality saturation,
  we can capture an entire design
  in an \egr.
Once we have captured a full design, we can
  use equational reasoning via rewrites
  to achieve tasks that
  would otherwise be challenging
  for SMT solvers---%
  for example, splitting a wide
  multiplier
  into multiple multipliers and adders.
Similarly, we can use rewrites
  to discover places in the design
  where we might want to invoke \lr.
Consider the benchmarks
  we generated to evaluate DSP mapping in
  in \cref{sec:evaluation}---%
  we could similarly generate patterns
  to identify potential places where DSPs could be used
  in larger designs.
Once these locations are identified within the design,
  we can call \lr as a subroutine,
  and if \lr finds a mapping,
  we can then import it back into the \egr.

Churchroad has already shown promise as a method
  of scaling solver-aided hardware compilation.
Beyond that, however,
  Churchroad also presents new strategies
  for leveraging the strengths of
  of SMT solvers and equality saturation.