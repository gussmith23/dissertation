\section{Conclusion}
\label{sec:lakeroad-conclusion}

This paper presents \lr, 
  a novel approach to FPGA technology mapping
  that leverages program synthesis techniques
  to provide stronger correctness and completeness guarantees 
  than state-of-the-art tools.
Because program synthesis tools
  can efficiently explore large search spaces, 
  \lr 
  can find mappings
  of hardware designs
  to FPGA DSPs
  in more cases
  than state-of-the-art tools,
  often finding more efficient implementations
  in the process.
With our techniques
  of semantics extraction
  from HDL
  and architecture-independent sketch templates,
  users must expend little manual effort 
  to apply \lr to
  a given FPGA architecture
  and extend it to handle further primitives.
% \lr showcases 
%   the incompleteness of 
%   existing tools for DSPs,
%   and showcases the advantages
%   of complete semantics for technology mapping.
Moreover, our formalization of \lr
  fosters greater confidence
  in its correctness.
\lr hence enables the 
  extensible, efficient, and correct 
  lowering of hardware designs to FPGAs,
  highlighting the effectiveness
  of program synthesis
  for FPGA technology mapping.


% In this paper we showed how two key insights,
%   scoping technology mapping to design fragments
%   (like instructions)
%   and
%   automatically extracting primitive semantics from HDL,
%   enabled building \lr,
%   a new technology mapper
%   based on state-of-the-art program synthesis techniques.
% \lr's prototype
%   automatically imports semantics
%   for all primitives from the
%   Xilinx UltraScale+,
%   Lattice ECP5,
%   and SOFA~\cite{sofa} FPGA architectures, and
%   only requires the user to provide
%   a short architecture description.
% \lr showcases 
%   the incompleteness of 
%   existing tools for DSPs,
%   and showcases the advantages
%   of complete semantics for technology mapping.
% % \lr's implementations
% %   use a similar number of FGPA resources compared
% %   to those generated by
% %   Xilinx's Vivado, Lattice's Diamond,
% %   and the open-source Yosys,
% %   and take roughly the same amount of time
% %   to generate.
% Finally,
%   \lr's implementations
%   are \textit{correct by construction,}
%   as they are derived directly from the semantics
%   imported from the vendor-supplied HDL.\tighten
% % Finally,
% %   we show the broader applicability
% %   of \lr
% %   by integrating it into
% %   Yosys,
% %   significantly expanding Yosys's
% %   support for
% %   DSPs on Lattice ECP5
% %   and Xilinx UltraScale+,
% %   and requiring orders of magnitude less code
% %   to support more DSPs in the future.
