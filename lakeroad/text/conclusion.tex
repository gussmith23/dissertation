\label{sec:lakeroad-conclusion}

\Cref{part:lakeroad} presented \lr,
  a novel approach to \gls{fpga} \gls{technology-mapping}.
\lr utilizes both more adaptable \cref{thesis:algorithms}---%
  sketch-guided \gls{program-synthesis}---%
  and more explicit \cref{thesis:models}---%
  vendor-supplied simulation models---%
  to provide greater \cref{thesis:correctness},
  completeness (i.e.~\cref{thesis:optimizations}),
  and extensibility (i.e.~reduced \cref{thesis:devtime})
  over state-of-the-art tools.
Because program synthesis tools
  can efficiently explore large search spaces, 
  \lr 
  can find mappings
  of hardware designs
  to FPGA DSPs
  in more cases
  than state-of-the-art tools,
  often finding more efficient implementations
  in the process.
With our techniques
  of semantics extraction
  from HDL
  and architecture-independent sketch templates,
  users must expend little manual effort 
  to apply \lr to
  a given FPGA architecture
  and extend it to handle further primitives.
% \lr showcases 
%   the incompleteness of 
%   existing tools for DSPs,
%   and showcases the advantages
%   of complete semantics for technology mapping.
Moreover, our formalization of \lr
  fosters greater confidence
  in its correctness.
\lr hence enables the 
  extensible, efficient, and correct 
  lowering of hardware designs to FPGAs,
  highlighting the effectiveness
  of program synthesis
  for FPGA technology mapping.


\lr cleanly and completely realizes
  my thesis set out at the start of this dissertation:
  utilize explicit, vendor-supplied \cref{thesis:models},
  apply state-of-the-art automated reasoning 
  \cref{thesis:algorithms},
  and you will produce a 
  powerful compiler backend when measured along the axes of
  \cref{thesis:optimizations},
  \cref{thesis:correctness}, and
  \cref{thesis:devtime}.
  