\section{Formalization}
\label{sec:formalization}


%%%%%%%%%%%%%%%%%%%% Lakeroad DSL Grammar %%%%%%%%%%%%%%%%%%%%
\begin{minipage}{.35\textwidth}
%%%%%%%%%%%%%%%%%%%%%%%%%%%%%%%%%% 1st Column %%%%%%%%%%%%%%%%%%%%%%%%%%%%%%%%%%
%\footnotesize{}
\begin{tabular}{l@{\hspace{.5em}}l}
$\SynProg$     & $\defn$ $\langle$ $\SynId$, $\seq{\SynId, \SynNode}*\rangle$\\[0.5em]
$\SynNode$       & $\defn \SynBv\ b$  | $\SynVar\ x$ \\
                 & \ \   | $\Op\ op\ \SynId$* \\
                 & \ \   | \IRReg{} $\SynId$ $(\SynBv~b)$ \\
                 & \ \   | \IRPrim{} $\mathsf{binds}$ $\SynProg$ \\
                 & \ \   | $\blacksquare_x$ \\[0.5em]
    
% $e\ \defn$       & $\seq{i, \hat{e}}$ \\[1em]
%
% $\hat{e}\ \defn$ & 
%   $b \vbar x \vbar$ 
%   \lstinline[language=thelang,mathescape]!let $x$ := $e$ in $e$! $\vbar$ 
% $o\ e^{*} \vbar$ \\
%   & \lstinline[language=thelang,mathescape]!Reg $e$ $e$ $b$! $\vbar$ 
%     \lstinline[language=thelang,mathescape]!Prim $x^{*}$ $e^{*}$ $e$! $\vbar$ 
% $\blacksquare_x$
\end{tabular}
\end{minipage}%
%%%%%%%%%%%%%%%%%%%%%%%%%%%%%%%%%% 2nd Column %%%%%%%%%%%%%%%%%%%%%%%%%%%%%%%%%%
\begin{minipage}{.25\textwidth}
\footnotesize{}
\begin{tabular}{l l}
$\SynId$  & $id  \in \mathbb{N}$ \\  
Bitvectors  & $b  \in \mathbb{BV}$ \\
Variables & $x  \in \text{LegalVarNames}$\\[0.5em]
Operators     & $op \in$ $\OpBv  \cup  \OpWire$\\[0.5em]
$\mathsf{binds}$ & $bs \in (\text{Variables} \rightharpoonup \SynId)$
\end{tabular}
\end{minipage}
%%%%%%%%%%%%%%%%%%%%%%%%%%%%%%%%%% 3rd Column %%%%%%%%%%%%%%%%%%%%%%%%%%%%%%%%%%
\begin{minipage}{.4\textwidth}
\footnotesize{}
\begin{tabular}{l l}
\;\;\; Wire op & $\OpWire =$ \lstinline[language=thelang,mathescape]!$\{$concat$,$extract$, \ldots \}$! \\
\;\;\; Non-wire op & $\OpBv =$ \lstinline[language=thelang,mathescape]!$\{+, -, \times, \ldots \}$! \\
\end{tabular}
\end{minipage}


%%%%%%%%%%%%%%%%%%%% END LR DSL GRAMMAR %%%%%%%%%%%%%%%%%%%%


In this section we formalize
  the syntax and semantics of \lr,
  and then state and prove
  soundness and completeness properties (assuming the correctness of the underlying program synthesis tool and HDL compiler).
%We begin by stating the soundness and
  %and completeness properties, and then
  %build up the necessary machinery to prove
  %them bottom up.

%\hl{TODO: State soundness and completeness theorems}

%\hl{Note: This is a rough narrative pass/overview of formalization. It should be compiled town to paper-worth latex.
%}

%To talk about correctness and completeness we need to construct an abstract mathematical model of Anaxi. So, the outline of this writeup is:

% \begin{enumerate}
% \item Syntax and Semantics
% \item Mathematical Model of Anaxi
% \item Correctness of Anaxi
% \item Completeness of Anaxi
% \end{enumerate}

\subsection{Syntax and Semantics}

%
%%%%%%%%%%%%%%%%%%%% Lakeroad DSL Grammar %%%%%%%%%%%%%%%%%%%%
\begin{figure}
% This line is causing a font error.
%\small
\setlength{\grammarparsep}{3pt plus 1pt minus 1pt} % increase separation between rules
\setlength{\grammarindent}{5em} % increase separation between LHS/RHS 
\small{}
\begin{grammar}

<\lowlrir> ::= `prim' name <list> <\lowlrir>
       \alt signal-const | signal-var  | <list> 
       \alt <signal-op> | <list-op>

<list>  ::= [<\lowlrir>\ldots]

<signal-op> ::= `extract' int int <\lowlrir>
         \alt `concat' <list>
%         \alt `zero-extend' int <lrir>

<list-op> ::= `list-ref' int <\lowlrir>
         % \alt `take' int <\lowlrir>
         % \alt `drop' int <\lowlrir>


% <routing> ::= `bitwise' | `bitwise-reverse'

% This isn't used
% <fpga-fn> ::= `ecp5-lut4' \ |\ \ldots

<sketch> ::= <\lowlrir> | <hole>

<hole> ::= `(' `bitvec' width `)' 
    \alt `(' `choose' <sketch> \ldots `)'
         
\end{grammar}

\caption{Syntax for \lrir.}
\label{fig:lakeroad-dsl-grammar}
\end{figure}
%%%%%%%%%%%%%%%%%%%% END LR DSL GRAMMAR %%%%%%%%%%%%%%%%%%%%

%\begin{figure*}

\newcommand{\lrirSemanticsVspace}{5mm}

% \inference{\textrm{bv} ~ v}{v}[\lrir-bv]
% 
% \vspace{\lrirSemanticsVspace}

% \begin{subfigure}{0.32\textwidth}
%   \centering
  
%%%%%%%%%%%%%%%%%%%%%%%%%%%%%%%%%%
% LRIR-list
\mbox{
\inference{e_0 \Downarrow v_0 
   ~ \cdots ~ e_n \Downarrow v_n
}
          {     \left[ e_0,\, \ldots,\, e_n\right] 
                \Downarrow 
                \left[ v_0,\, \ldots,\, v_n\right] }
          [{\bf list}]}%
\quad
\mbox{\inference{l \Downarrow \lbrack v_0 \ldots v_i \ldots v_n\rbrack}
          {\texttt{list-ref}\ l\ i 
           \Downarrow 
           v_i}
          [{\bf list-ref}]
}
%%%%%%%%%%%%%%%%%%%%%%%%%%%%%%%%%%
% LRIR-prim
\quad
\mbox{
\inference%
  {e \Downarrow v 
  }
  {\texttt{prim}\ \rho\ \kappa\ e \Downarrow \Gamma(\rho)(\kappa)(v)}
  [{\bf prim}]
}

  \vspace{\lrirSemanticsVspace}

%%%%%%%%%%%%%%%%%%%%%%%%%%%%%%%%%%
% LRIR-extract
\mbox{
\inference{e \Downarrow \left(\langle b_0 \ldots b_i \ldots b_j \ldots b_n\rangle, s \right)}
          {\texttt{extract}\ i\ j\ e \Downarrow \left( \langle b_i \ldots b_j \rangle, s \right)}
          [{\bf extract}]
}%
\quad
%%%%%%%%%%%%%%%%%%%%%%%%%%%%%%%%%%
% LRIR-concat
\mbox{
\inference{   e_0 \Downarrow \left(\langle b^0_0\ldots b^0_{n_1}\rangle, s_0 \right)
              ~ \cdots ~
              e_k \Downarrow \left( \langle b^k_0\ldots b^k_{n_k} \rangle, s_n\right)
& s = \bigcup^i s_i
              }
          {\texttt{concat}\ e_0,\,\ldots,\, e_k
           \Downarrow 
           \left(\langle b^0_0 \ldots b^0_{n_1} 
           \ldots
           b^k_0 \ldots b^k_{n_k} \rangle, 
           s\right)}
          [{\bf concat}]
}


% \end{subfigure}%
% \hspace{5mm}%
% \begin{subfigure}{0.25\textwidth}


%%%%%%%%%%%%%%%%%%%%%%%%%%%%%%%%%%
% LRIR-take
% \inference{l \Downarrow \lbrack v_0 \ldots  
%                                 v_n\rbrack
%             & 0 \leq i \leq n+1}
%           {\texttt{take}\ l\ i 
%            \Downarrow 
%            \lbrack v_0,\, \ldots,\, v_{i-1} \rbrack}
%           [{\bf take}]
% \vspace{\lrirSemanticsVspace}

% \end{subfigure}\hspace{5mm}%
% \begin{subfigure}{0.32\textwidth}
%   \centering

          
%%%%%%%%%%%%%%%%%%%%%%%%%%%%%%%%%%
% LRIR-drop
% \inference{l \Downarrow \lbrack v_0 \ldots v_n\rbrack
%            & 0 \leq i \leq n+1 }
%           {\texttt{drop}\ l\ i 
%            \Downarrow 
%            \lbrack v_i,\, \ldots,\, v_{n} \rbrack}
%           [{\bf drop}]
% \vspace{\lrirSemanticsVspace}


          
% \end{subfigure}%


% \begin{subfigure}{.5\textwidth}
% \end{subfigure}%
% \hspace{5mm}%
% \begin{subfigure}{0.4\textwidth}
% \centering
%%%%%%%%%%%%%%%%%%%%%%%%%%%%%%%%%%
% LRIR-bitwise-phys-to-log
% \inference%
%   {l \Downarrow 
%          \lbrack 
%           \langle bv^0_0,\, \ldots,\,  bv^0_k\rangle,\,
%           \cdots,\,
%           \langle bv^n_0,\, \ldots,\,  bv^n_k\rangle
%          \rbrack}
%   {\texttt{wire}\ \ \texttt{bitwise}\ \ l
%    \Downarrow
%       \lbrack 
%        \langle bv^0_0,\, \ldots,\,  bv^n_0\rangle,\,
%        \cdots,\,
%        \langle bv^0_k,\, \ldots,\,  bv^n_k\rangle
%       \rbrack}
%   [{\bf bw-wire}]
% \vspace{\lrirSemanticsVspace}

% \end{subfigure}

\hspace{30mm}\begin{subfigure}{0.9\textwidth}

%%%%%%%%%%%%%%%%%%%%%%%%%%%%%%%%%%
% LRIR-bwrev-phys-to-log
% \inference%
%   {l \Downarrow 
%          \lbrack 
%           \langle bv^0_0,\, \ldots,\,  bv^0_k\rangle,\,
%           \cdots,\,
%           \langle bv^n_0,\, \ldots,\,  bv^n_k\rangle
%          \rbrack}
%   {\texttt{wire}\ \ \texttt{bitwise-reverse}\ \ l
%    \Downarrow
%       \lbrack 
%        \langle bv_k^0,\, \ldots,\,  bv_k^n\rangle,\,
%        \cdots,\,
%        \langle bv_0^0,\, \ldots,\,  bv_0^n\rangle
%       \rbrack}
%   [{\bf bwrev-wire}]
% \vspace{\lrirSemanticsVspace}

\end{subfigure}
\caption{\lrir semantics. Angled brackets indicate bitvector literals, while square brackets indicate list literals.}
\label{fig:lrir-semantics}
\end{figure*}



% %%%% TRY 2 %%%%%
% \begin{figure*}
% \newcommand{\lrirSemanticsVspace}{5mm}

% \begin{subfigure}{0.3\textwidth}
%   \centering
% %%%%%%%%%%%%%%%%%%%%%%%%%%%%%%%%%%
% % LRIR-list
% \inference{e_0 \Downarrow v_0 & \cdots & e_n \Downarrow v_n}
%           {     \left[ e_0,\, \ldots,\, e_n\right] 
%                 \Downarrow 
%                 \left[ v_0,\, \ldots,\, v_n\right]}
%           [{\bf list}]
% \vspace{\lrirSemanticsVspace}

% %%%%%%%%%%%%%%%%%%%%%%%%%%%%%%%%%%
% % LRIR-extract
% \inference{e \Downarrow \langle b_0 \ldots b_i \ldots b_j \ldots b_n\rangle}
%           {\texttt{extract}\ i\ j\ e \Downarrow \langle b_i \ldots b_j \rangle}
%           [{\bf extract}]
% \vspace{\lrirSemanticsVspace}
% \end{subfigure}
% \begin{subfigure}{0.3\textwidth}
%   \centering
% %%%%%%%%%%%%%%%%%%%%%%%%%%%%%%%%%%

% % LRIR-prim
% \inference%
%   {e \Downarrow v 
%   }
%   {\texttt{prim}\ \rho\ \kappa\ e \Downarrow \Gamma(\rho)(\kappa)(x)}
%   [{\bf prim}]
% \vspace{\lrirSemanticsVspace}

% %%%%%%%%%%%%%%%%%%%%%%%%%%%%%%%%%%
% % LRIR-list-ref

% \inference{l \Downarrow \lbrack v_0 \ldots v_i \ldots v_n\rbrack}
%           {\texttt{list-ref}\ l\ i 
%           \Downarrow 
%           v_i}
%           [{\bf list-ref}]
% \vspace{\lrirSemanticsVspace}
% \end{subfigure}
% \begin{subfigure}{0.3\textwidth}
% %%%%%%%%%%%%%%%%%%%%%%%%%%%%%%%%%%
% % LRIR-take
% \inference{l \Downarrow \lbrack v_0 \ldots  
%                                 v_n\rbrack
%             & 0 \leq i \leq n+1}
%           {\texttt{take}\ l\ i 
%           \Downarrow 
%           \lbrack v_0,\, \ldots,\, v_{i-1} \rbrack}
%           [{\bf take}]
% \vspace{\lrirSemanticsVspace}
          
% %%%%%%%%%%%%%%%%%%%%%%%%%%%%%%%%%%
% % LRIR-drop
% \inference{l \Downarrow \lbrack v_0 \ldots v_n\rbrack
%           & 0 \leq i \leq n+1 }
%           {\texttt{drop}\ l\ i 
%           \Downarrow 
%           \lbrack v_i,\, \ldots,\, v_{n} \rbrack}
%           [{\bf drop}]
% \vspace{\lrirSemanticsVspace}
% \end{subfigure}
% \begin{subfigure}{\textwidth}
% \centering
% %%%%%%%%%%%%%%%%%%%%%%%%%%%%%%%%%%
% % LRIR-concat
% \inference{   e_1 \Downarrow \langle b^1_0\ldots b^1_{n_1}\rangle
%               & \cdots &
%               e_k \Downarrow \langle b^k_0\ldots b^k_{n_k} \rangle}
%           {\texttt{concat}\ e_1,\,\ldots,\, e_k
%           \Downarrow 
%           \langle b^1_0 \ldots b^1_{n_1} 
%           \ldots
%           b^k_0 \ldots b^k_{n_k} \rangle }
%           [{\bf concat}]
% \vspace{\lrirSemanticsVspace}
% \end{subfigure}
% \begin{subfigure}{\textwidth}
% \centering
% \inference%
%   {l \Downarrow 
%          \lbrack 
%           \langle bv^0_0,\, \ldots,\,  bv^0_k\rangle,\,
%           \cdots,\,
%           \langle bv^n_0,\, \ldots,\,  bv^n_k\rangle
%          \rbrack}
%   {\texttt{wire}\ \ \texttt{bitwise}\ \ l
%   \Downarrow
%       \lbrack 
%       \langle bv^0_0,\, \ldots,\,  bv^n_0\rangle,\,
%       \cdots,\,
%       \langle bv^0_k,\, \ldots,\,  bv^n_k\rangle
%       \rbrack}
%   [{\bf bw-wire}]
% \vspace{\lrirSemanticsVspace}
% \end{subfigure}
% %\hspace{30mm}\begin{subfigure}{0.9\textwidth}
% %\centering
% %%%%%%%%%%%%%%%%%%%%%%%%%%%%%%%%%%
% % LRIR-bitwise-phys-to-log
% \begin{subfigure}{0.9\textwidth}
% \centering
% %%%%%%%%%%%%%%%%%%%%%%%%%%%%%%%%%%
% % LRIR-bwrev-phys-to-log
% \inference%
%   {l \Downarrow 
%          \lbrack 
%           \langle bv^0_0,\, \ldots,\,  bv^0_k\rangle,\,
%           \cdots,\,
%           \langle bv^n_0,\, \ldots,\,  bv^n_k\rangle
%          \rbrack}
%   {\texttt{wire}\ \ \texttt{bitwise-reverse}\ \ l
%   \Downarrow
%       \lbrack 
%       \langle bv_k^0,\, \ldots,\,  bv_k^n\rangle,\,
%       \cdots,\,
%       \langle bv_0^0,\, \ldots,\,  bv_0^n\rangle
%       \rbrack}
%   [{\bf bwrev-wire}]
% \vspace{\lrirSemanticsVspace}
% \end{subfigure}
% \caption{\lrir semantics.}
% \label{fig:lrir-semantics}
% \end{figure*}


% \inference{l \in [\textrm{BV}], \textrm{logical-to-physical}~\textrm{bitwise}~l}
% {
% [(\textrm{concat}~(\textrm{extract}~0~0~l[0])~%
% (\textrm{extract}~0~0~l[1])~\dots)~%
% (\textrm{concat}~(\textrm{extract}~1~1~l[0])~%
% (\textrm{extract}~1~1~l[1])~\dots)~\dots ]
% }
% [\lrir-ltop-bitwise]

% \vspace{\lrirSem\lrirSemanticsVspace}

% \inference{l \in [\textrm{BV}], \textrm{logical-to-physical}~\textrm{bitwise-reverse}~l}
% {
% \begin{array}{@{}c@{}}
% [(\textrm{concat}~(\textrm{extract}~n~n~l[0])~%
% (\textrm{extract}~n~n~l[1])~\dots)   \\
% (\textrm{concat}~(\textrm{extract}~n-1~n-1~l[0])~%
% (\textrm{extract}~n-1~n-1~l[1])~\dots)~\dots ]
% \end{array}
% }
% [\lrir-ltop-bitwise-reverse]

% \vspace{\lrirSem\lrirSemanticsVspace}

% \inference
% {\textrm{primitive}~n~e}
% {\textrm{look up primitive semantics for $n$, run on $e$}}

\lr uses the \lr Intermediate Representation (\lrir) to represent
  wire operations and computations on FPGAs, the grammar for which is given in Figure~\ref{fig:lakeroad-dsl-grammar}.
Sketches use the \texttt{hole} construct
  to indicate where
  program synthesis should be used
  to find a replacement AST subtree
  for that hole 
  (either finding a bitvector to substitute or choosing between a list of alternatives).
Otherwise, the grammar consists
  of constructs corresponding to
  what can be implemented on the FPGA.
Wire operations are performed
  by direct
  bitvector manipulations
  like \texttt{concat}
  and \texttt{extract},
  and by high-level manipulations
  via the \texttt{wire} construct;
  these are all architecture-independent.
\lrir supports
  only these simple bitvector operations directly.
More complex manipulations
  are supported through calls 
  to \textit{hardware primitives} (e.g., LUTs and DSPs).
These hardware module invocations
  are captured in \lrir
  by \texttt{prim} terms,
  which refer to primitives
  by name (e.g., ``\texttt{LUT4}'')
  and pass in a configuration.
These constructs can all be lowered
  straightforwardly to HDL code
  that can in turn be compiled
  to an FPGA configuration.

\subsubsection{Combinational Semantics}

The basic operations in \lrir%
  ---all other than \texttt{prim}---%
  have stateless (combinational) semantics.
Namely, they are defined over
  bitvectors and lists,
  taking bitvectors or lists of bitvectors as arguments
  and returning a new bitvector or list of bitvectors,
  without any other changes
  to the program state.
The semantics for these constructs 
  is given in Figure~\ref{fig:lrir-semantics}.


% At its core,
%   \lrir is
%   an IR for manipulating
%   bitvectors and lists, where \lrir expressions should convert directly to low-level, FPGA-specific Verilog
%   code.
% % A primary design goal of \lrir
% %   is that expressions
% %   should convert directly
% %   to low-level, FPGA-specific
% %   Verilog code.
% This means that
%   \lrir directly supports only a limited subset
%   of possible bitvector operations
%   (e.g., bit extraction
%   or bitvector concatenation).
% More complex manipulations,
%   including computations
%   over bitvectors,
%   are supported only through calls to
%   \textit{hardware primitives} (e.g., LUTs and DSPs)
%   by using a \texttt{prim} expression;
%   these \texttt{prim} expression
%   can be directly converted
%   to Verilog module instantiations.
% Its leaf expressions
%   are bitvector constants
%   and variables,
%   where variables
%   represent the inputs
%   to the program.

\hl{The semantics define an interpreter $\mathcal{I}$}

We need to define the semantics for:

1. \textit{Combinational evaluation (w/out state...state is elided)}
2. \textit{Sequential evaluation (w/ state)}


\subsubsection{Combinational Semantics}
\hl{Update and reference figure 7 in paper; discuss}

\subsubsection{Sequential Semantics}
\hl{Update and reference figure 5 in paper; discuss}

\subsection{Mathematical Model of Anaxi}

In this section formalize Anaxi by modelling it a function 

$$f_A : Sketch \times Design \times Sems \to HDL$$
$$f_A := f_{HDL} \circ f_{Synth}$$

where
$$f_{Synth} : Sketch \times Design \times Sems \to IR$$
and
$$f_{HDL} : IR \to HDL.$$

We do not include sketch generation
  as part of our formalization, and treat
  it as a preprocessing step.


\subsubsection{Anaxi Synthesis: $f_{Synth}$}

The synthesis step takes a sketch $s$, a design $d$, and some interpreter $\mathcal{I}$, and queries Rosette to find an implementation $p$ of sketch $s$ such that for all inputs $i$, $p(i) = d(i)$. Here our design $d$ is being used as a specification.

We need to be able to
  synthesize \textit{combinational}
  and \textit{sequential} programs. 

\begin{itemize}
\item \textbf{Synthesizing a Combinational Program:}
  To synthesize a combinational
    program we pass the sketch
    $s$ to Rosette,
    along with an interpreter
    defined by the combinational
    semantics described in the
    \hl{Combinational Semantics}
    section.
  No state (e.g., clock cycles,
    etc) are used here.

  We then issue a query to Rosette:
  
  $$\forall i\ \in Inputs\ \exists p \in AxIR\ s.t. p(i) = d(i)$$

  Rosette performs exhaustive
    search to find
    a $p$ that satisfies
    these constraints
    if one exists.
    
\item \textbf{Synthesizing Sequential Programs:}
  To synthesize a sequential
    program we need to
    implement the semantics of
    multiple clock cycles.
  We do this by treating
    a program $p$ as a
    function from $k$ input
    bits and an input
    state to $\ell$ output
    bits and an output
    state $\Sigma$:

  $$p : \mathbb B^k \times \Sigma \to \mathbb B^\ell \times \Sigma.$$

  We represent $n$ 
    sequential applications
    of program $p$ with
    the function
    
  \[ p^{*}: \left(\mathbb{B}^k\right)^n \times \Sigma \to \mathbb{B}^\ell \times \Sigma,\]

  where

  \[p^{*}( \langle b_1, b_2, \ldots, b_n\rangle,\ \sigma_0)\]
  
  is defined to be
  
  $$p(b_1,\ \sigma_0)$$
  
  if $n = 1$, and
  \[p^{*}(\langle b_2, \ldots, b_n\rangle,\ \pi_\Sigma(s(b_1, \sigma_0)))\]
  
  if $n > 1$.



  Each $b_i \in \mathbb{B}^k$ 
    is the input bits 
    being wired to the
    circuit at time step $i$,
    and $\pi_\Sigma: \mathbb{B}^\ell \times \Sigma \to \Sigma$
    is the projection 
    function that drops
    the output bits and
    returns only the
    resulting state.

\hl{NOTE: we can model this in two ways: as a vectorized notation, or as an inductive definition. Both are usable; the above is inductive.}

  We then issue a query to Rosette:

  $$\forall \hat{b} \ \in \left(\mathbb{B}^k\right)^n,\ \exists p \in AxIR\ s.t.\ \pi_{Val} (p(\hat{b}, \sigma_0)) = \pi_{Val}(d(\hat{b}, \sigma_0)),$$

  where $\hat{b} = \langle b_1, \ldots, b_n\rangle$.

\end{itemize}

\subsubsection{Anaxi Compilation: $f_{HDL}$}
\hl{TODO: Discuss the process of compilation (Gus will need to do this!)}

\subsection{Program Synthesis}

The definition of program synthesis is

$$\exists p\,\, \forall \hat{i} \quad Spec(\hat{i}) = p(\hat{i})$$


\subsection{Correctness of Anaxi}

Correctness of Anaxi is decomposed into correctness of synthesis and correctness of compilation.

\hl{TODO: add commutative diagram!}

\subsubsection{Correctness of Compilation}

Correctness of compilation states that whenever $f_{HDL}$ maps an IR program $p$ to an HDL program $v$, this transformation preserves the semantics of the IR program: $\llbracket p\rrbracket = \llbracket v \rrbracket$.

We claim that compilation is correct. While we do not have a formal proof (e.g., from a proof assistant), compilation is a simple process and we have tested it extensively.

\subsubsection{Correctness of Synthesis}

\begin{itemize}
\item \textbf{Our TCB}
  \begin{itemize}
	\item Rosette/Solvers
	\item Interpreter $\mathcal{I}$ , which in turn depends on
     \begin{itemize}
		\item Vendor-supplied semantics
		\item Translation of semantics into an interpreter
        \end{itemize}
    \end{itemize}
\end{itemize}

Correctness of synthesis states that whenever $f_{Synth}$ maps a sketch $s$, a design $d$, and an interpreter $\mathcal{I}$ to an IR program $p$, then the semantics of the original design are preserved: $[|d|] = [|p|]$.

We get correctness of synthesis from Rosette: by the definition of program synthesis, as long as our query that we have constructed is correct then the resulting. Correctness of synthesis relies on the correctness of Rosette. We construct a query

\subsection{Completeness of Anaxi}

Anaxi cannot claim full completeness, but it can claim \textit{sketch completeness:} if there is a sketch that can be instantiated to a program that implements a design $d$, then up to timeouts Anaxi will find it. We break sketch completeness down into \textit{Sketch Completeness of Synthesis} and \textit{Completeness of Compilation}.

\subsubsection{ Sketch-Completeness of Synthesis}

We argue that $f_{Synth}$
  is sketch-complete: that is,
  if there exists a program $p'$
  that can be formed by
  implementing a sketch 
  $s$ such that
  $\llbracket p'\rrbracket = \llbracket d \rrbracket$
  then $f_{Synth}$ will
  produce some program 
  $p$ (where $p \equiv p'$).

Stated another way,
  sketch-completeness of
  synthesis states that $f_{Synth}$
  is total on the restricted 
  domain of 
  $Sketch \times Design \times Sems$.
This follows from the
  definition of program
  synthesis \hl{(todo: elaborate)}.

\subsubsection{Completeness of Compilation}

Completeness of compilation says that $f_{HDL}$ is a total function (i.e., compilation never fails).

