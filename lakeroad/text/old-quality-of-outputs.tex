\subsection{Quality of Output}
To evaluate the quality of \lr's output,
  we compare \lr against four tools---%
  Yosys,
  Vivado,
  Diamond,
  Quartus---%
  on the task of mapping
  a set of benchmark designs
  to DSPs
  on three FPGA architectures---%
  Xilinx UltraScale+,
  Lattice ECP5,
  \hl{todo intel something?}.
Note that we do not include
  SOFA,
  as the SOFA architecture
  does not include DSPs.

Our set of DSP benchmarks
  is assembled from GitHub.
Using GitHub's advanced search tool,
  we searched through Verilog projects
  on GitHub
  which had submodules
  intended to be compiled onto
  the DSPs of various FPGA platforms.
These submodules were identified
  in a number of ways:
  for example, by searching for
  the Xilinx-specific
  \texttt{(* use\_dsp = "yes" *)}
  annotation.
We attempt to compile \textit{every}
  benchmark
  on \textit{every}
  architecture,
  although most benchmarks
  were written
  for a specific FPGA architecture.

\cref{tab:dsp-benchmarks}
  shows our results.
  
\hl{We are comparable in quality to
  existing, proprietary tools.}

\hl{We enable broader support
  for targeting DSPs
  in open-source tools like Yosys.}




  
Furthermore, to highlight \lr's
  utility to existing open-source tools,
  we directly integrated 
  integrated \lr 
  into Yosys
  as the 
  \texttt{\lowercaselr}
  subcommand,
  using only 75 lines
  of C++.
Thus, when we mention
  \lr in this section,
  we actually mean
  the
  \texttt{\lowercaselr}

\hl{The below section is  text from the old
``support for complex primitives'' section.
Tasks: (1) determine whether to keep
the stuff about yosys integration,
or where else we might put it
(2) fold the ``synthetic'' benchmarks 
into our benchmark suite, if we don't find
existing benchmarks that do those things
}


In this section,
  we evaluate \lr's ability
  to utilize complex FPGA primitives.
To do so,
  we integrated \lr 
  into Yosys
  as the 
  \texttt{\lowercaselr}
  subcommand,
  using 75 lines
  of C++.
We then compare 
  the output of the 
  \texttt{\lowercaselr}
  subcommand
  against the output of
  Yosys's existing
  \texttt{synth\_xilinx}
  and
  \texttt{synth\_ecp5}
  subcommands,
  which target Xilinx UltraScale+
  and Lattice ECP5
  respectively.
These results highlight how \lr's approach
  enables providing high-quality technology mapping
  in Yosys with little manual implementation effort,
  suggesting ease of adding support for new
  future FPGA architectures.

To evaluate \lr's ability
  to map to complex primitives
  on Xilinx UltraScale+,
  we use the
  \texttt{\lowercaselr}
  and 
  \texttt{synth\_xilinx}
  subcommands
  to attempt to map
  a set of Verilog modules
  to an UltraScale+ DSP.
Specifically,
  our set of modules
  implement the set of computations
  listed in tables 2-1 and 2-2
  of Xilinx's
  UltraScale Architecture DSP Slice User Guide~\cite{userguide}.
These tables list
  the functions implemented
  by the UltraScale+ DSP;
  thus, we know that all of
  the modules in our set
  should map one-to-one
  to UltraScale+ DSPs.
  
The results for UltraScale+
  are listed in
  \cref{tab:yosys-xilinx}.
As you can see,
  the \texttt{\lowercaselr}
  subcommand of Yosys
  is able to map all expressions
  to a single DSP
  by automatically determining
  the correct configuration
  of the DSP's 20+ ports
  and 40+ parameters.
Meanwhile,
  Yosys's
  \texttt{synth\_xilinx}
  can only map to DSPs
  via manually written
  patterns,
  and currently, 
  Yosys only implements patterns
  for mapping multiplication.
Thus,
  \texttt{synth\_xilinx}
  is able to capture the multiplication
  portion of each target module,
  but it maps the rest of the computation
  (e.g., additions, subtractions,
  or logic functions)
  to LUTs and carry chains.
Note that the runtimes
  of the two passes are comparable,
  except in the case of the $(a-b)^2-c$
  expression,
  when the \texttt{\lowercaselr}
  subcommand
  is an order of magnitude slower;
  we speculate this is due to unpredictable
  behavior in the underlying SAT solver.
  
To evaluate \lr's ability
  to map to complex primitives
  on Lattice ECP5,
  we use the
  \texttt{\lowercaselr}
  and 
  \texttt{synth\_ecp5}
  subcommands
  to attempt to map
  a set of Verilog modules
  to a Lattice ECP5 DSP.
The Lattice ECP5 DSP
  is split into multiple primitives;
  we evaluate \lr's ability to map
  to the multiplier (MULT18X18D)
  and ALU (ALU24B) primitives.
The results are shown
  in \cref{tab:yosys-lattice}.
Similar to the Xilinx UltraScale+ results,
  we demonstrate that 
  Yosys's \texttt{synth\_ecp5}
  subcommand
  fails to map all expressions
  to the ECP5 DSP,
  with some of the computation
  being implemented onto CCU2Cs.
Note that
  \texttt{\lowercaselr}
  uses 1 multiplier and 1 ALU
  for the expression $a \times b$,
  while \texttt{synth\_ecp5}
  correctly uses just a single multiplier.
  this is because \lr's DSP sketch
  always instantiates both
  a multiplier and an ALU.
Detecting when the ALU is not necessary
  can be implemented using optimization features
  of modern SAT and SMT solvers;
  we leave this development
  for future work.
  
\begin{table}
    \centering
    \begin{tabular}{c|c}
    \end{tabular}
    \caption{Completeness of \lr against other tools.}
    \label{tab:mylabel}
\end{table}

We also demonstrate \lr's robustness across workloads
  compared to SOTA synthesis tools on three architectures. 
On a number of benchmarks, \lr succeeds in mapping 
  to DSPs compared to Yosys, and the architecture's corresponding proprietary synthesis tool.

%\subsection{RQ5: Usability by Real Frameworks}
%\label{sec:rq5}
%
%To demonstrate the ease of integrating \lr output with existing tools we
%replace the ISA implementations of the Calyx compiler with \lr's output.
%Calyx~\cite{nigam2021compiler}
%  is a language
%  and compiler
%  for hardware designs.
%  it supports multiple frontend languages,
%  including their own Calyx
%  language,
%  Dahlia~\cite{nigam2020predictable},
%  and TVM~\cite{chen2018tvm}.
%Calyx
%  compiles designs
%  into synthesizable,
%  architecture-independent
%  behavioral Verilog,
%  relying on hardware synthesis tools
%  to finish compilation
%  to the target FPGA
%  (currently, they focus on
%    Xilinx boards).
%
%To test correctness,
%  we run the modified Calyx's entire 
%  test suite
%  and ensure all tests pass.
%\hl{Do any of the tests
%  actually test the resulting Verilog?
%  Yes, I'm fairly certain a number of them
%  Verilate the code and simulate
%  and check the results.
%  So we need to make sure we say that,
%  in case a reviewer thinks
%  that the tests will pass no matter
%  what.}
%
%To test performance,
%  we compile smaller sample Calyx programs
%  found within Calyx's test suite,
%  and larger hardware designs
%  found in Calyx's evaluation repository. 
%We then run Calyx's
%  of Calyx's test suite
%  and evaluation,
%  allowing us to compile 
%  
%
%The purpose of our end-to-end
%  evaluation
%  is not to demonstrate
%  improved performance.
%To truly compete
%  with legacy logic synthesis
%  and optimization,
%  we would need an entire
%  FPGA compiler pipeline
%  designed to target an ISA
%  and with built-in 
%  optimization passes.
%Calyx includes optimization
%  passes \hl{(is that true?)},
%  but still relies on
%  hardware synthesis tools
%  like Vivado
%  for final optimization.
%Instead, we aim to match
%  previous performance,
%  while demonstrating
%  the ease of integration of an ISA
%  with a toolchain,
%  implying that future FPGA compilers
%  can gain the benefits of \lr-generated
%  ISAs
%  (quick support of new architectures,
%    multiple implementations of instructions,
%    etc.)
%    
%Note that our
%  ISA does not cover
%  the entire Calyx ISA.
%Notably,
%  we do not cover
%  their register-/memory-related
%  instructions.
%We also do not
%  support 
%  instructions
%  outside of their
%  ``core'' ISA;
%  i.e.~we do not
%  cover their
%  math instruction set.
%
%\subsection{Pareto Frontier Exploration}
%
%\hl{
%This might not need to be 
%  its own subsection,
%  but we need to highlight
%  the ability
%  to generate multiple implementations
%  per instruction.}


\hl{
The below text is from the 
``proof of concept''
section added during response.
(1) fold the table into the new DSP benchmark 
table.
(2) determine what text below we should keep.
}

\begin{table}[]
\hl{Fold this into DSP benchmark suite}
    \normalsize
    \centering
    \caption{
Synthesizing sequential designs
  onto a Xilinx DSP48E2
  using our approach.
All designs are 16 bit.}
    \label{tab:sequential}
    \begin{tabular}{c|c}
\hline
design & synthesis time (s) \\
\hline
2-stage $a\times b + c$ &  3.3 \\
3-stage $(a + b) \times c + d$ &  3.4 \\
3-stage $(a + b) \times c \oplus d$ &  3.3 
    \end{tabular}
\end{table}
  
The techniques presented
  can be applied to sequential logic
  as well.
Our method for extracting semantics 
  from Verilog
  (\cref{sec:implementation:importing-semantics})
  converts Verilog to the btor format~\cite{btor},
  which natively supports
  sequential logic
  with its \texttt{state} and \texttt{next}
  expressions.
This, in combination with 
  modifications to \lrir,
  enable support for synthesizing
  sequential logic.
A full handling
  of large sequential designs
  requires techniques
  in addition to just program synthesis:
for example,
  using richer 
  hardware design languages
  to convey timing information, 
and
  applying efficient term rewriting
  techniques
  such as equality saturation~\cite{willsey2021egg}
  to automatically decompose large designs.
We plan to present these techniques
  in future work.
Here, though, we present
  proof-of-concept evidence
  that program synthesis
  can be applied not only
  to combinational designs,
  but to sequential designs as well.

To demonstrate
  our technique's ability to handle
  sequential logic,
  we run synthesis tasks
  similar to those presented
  in \cref{tab:xilinx-resource-utilization}.
\cref{tab:sequential}
  presents the results.
Each line represents a sequential
  design which is synthesized into
  an instantiation of a Xilinx DSP48E2.
As our results show,
  we are able to map
  interesting sequential circuits,
  such as pipelined multiply--accumulators,
  onto DSPs,
  in timescales similar to those shown in
  \cref{tab:xilinx-resource-utilization,tab:lattice-resource-utilization,tab:sofa-resource-utilization,tab:yosys-lattice,tab:yosys-xilinx}.


