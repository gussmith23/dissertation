\section{Evaluation}
\label{sec:evaluation}

We now evaluate
  \lr in terms of completeness
  and extensibility.
In the following experiments,
  we target four FPGA architectures:
  \textbf{Xilinx UltraScale+},
  commonly used
  for large, high-performance workloads;
  \textbf{Lattice ECP5},
  commonly used 
  in low-power, low-cost scenarios;
  \textbf{Intel Cyclone 10 LP},
  an FPGA designed for low-cost,
  high-volume use cases,
  and 
  \textbf{SOFA}~\cite{sofa},
  a recent, open-source
  FPGA developed by the
  research community.
We compare \lr to existing
  technology mappers.
For
  Xilinx Ultrascale+, Lattice ECP5, and
  Intel Cyclone 10 LP,
  we compare \lr against both
  the open source toolchain Yosys~\cite{wolf2013yosys}
  and the
  state-of-the-art,
  proprietary, closed source
  toolchains
  for each architecture.\footnote{
    Again, licensing restrictions 
    prevent our naming the specific 
    proprietary tools, but
    they are familiar, standard packages 
    used by many hardware designers.}\tighten
% Though SOFA is nominally
%   supported by OpenFPGA~\cite{tang2019openfpga},
%   we failed to get the toolchain working despite
%   corresponding with the authors.
The experiments were conducted 
  on a system running Ubuntu 20.04.3 
  with an AMD EPYC 7702P 64-Core CPU.
The resident set size of a single \lr process
  did not exceed
  300MB while running our evaluation.
We use the software versions listed in \cref{sec:implementation}.
% Racket version 8.9 and
%   Rosette version 4.1 with
%   cvc5 version 1.0.1 and
%   Bitwuzla version 1.0
%   were used.

\subsection{\lr Completeness}
\label{sec:completeness}
% \begin{table}[]
\caption{Synthesis time and resource usage when synthesizing various expressions for Xilinx UltraScale+ using our \texttt{\lowercaselr} command and using the existing \texttt{synth\_xilinx} command in Yosys.}
\label{tab:yosys-xilinx}
\hl{Replace this figure with more expansive DSP tables}
\setlength{\tabcolsep}{2pt}
\centering
\begin{tabular}{l|ll|llll}
\hline
instruction                      & \multicolumn{2}{l|}{\texttt{\lowercaselr}} & \multicolumn{4}{l}{\texttt{synth\_xilinx}} \\
(all 16 bit)                     & time (s)           & DSP          & time (s)  & DSP  & LUT2  & CARRY4 \\ \hline
$a\times b$                      & 2.49               & 1            & 4.03      & 1    &       &        \\
$((a+b) \times c)+d$             & 2.44               & 1            & 4.18      & 1    & 32    & 8      \\
$(a\times b) + c$                & 3.02               & 1            & 4.02      & 1    & 16    & 4      \\
$(a+b)\times c$                  & 1.75               & 1            & 3.83      & 1    & 16    & 4      \\
$(a+b)^2$                        & 1.84               & 1            & 3.83      & 1    & 16    & 4      \\
$(a-b)^2$                        & 2.29               & 1            & 3.82      & 1    & 16    & 4      \\
$(a-b)^2 - c$                    & 21.7               & 1            & 3.93      & 1    & 32    & 8      \\
$(a+b)^2 +c$                     & 2.17               & 1            & 3.92      & 1    & 32    & 8      \\
$((a+b)\times c) \oplus d$       & 1.87               & 1            & 3.92      & 1    & 32    & 4      \\
$\sim(((a+b)\times c) \oplus d)$ & 2.43               & 1            & 4.01      & 1    & 32    & 4     
\end{tabular}
\end{table}
% \begin{table}[]
\caption{Synthesis time and resource usage when synthesizing various expressions for Lattice ECP5 using our \texttt{\lowercaselr} command and using the existing \texttt{synth\_ecp5} command in Yosys.}
\label{tab:yosys-lattice}
\hl{Replace this figure with more expansive DSP tables}
\setlength{\tabcolsep}{2pt}
\centering
\begin{tabular}{l|lll|lll}
\hline
instruction       & \multicolumn{3}{l|}{\texttt{\lowercaselr}} & \multicolumn{3}{l}{\texttt{synth\_ecp5}} \\
(all 16 bit)      & time (s)    & MULT         & ALU         & time (s)  & MULT        & CCU2C        \\ \hline
$a\times b$       & 1.42         & 1        & 1       & 0.65        & 1       &         \\
$(a\times b) + c$ & 1.87         & 1        & 1       & 0.74        & 1       & 8       \\
$(a\times b) - c$ & 1.61         & 1        & 1       & 0.61        & 1       & 8      
\end{tabular}
\end{table}
The reliance of many technology mappers,
  including state-of-the-art tools,
  on hand-written patterns
  %for mapping to DSPs
  leads them to fail when attempting to map many workloads
  that \textit{should} be mapped to a single DSP.
In particular, 
  the process of partial
  design mapping 
  (illustrated in \cref{sec:overview}) 
  becomes
  a laborious endeavor because
  of this incompleteness:
  hardware designers
  hand-instantiate DSPs rather
  than rely on substandard
  automated tooling, repeating
  the work each time they identify a potential
  opportunity to use a DSP.
\lr's greater mapping completeness 
  significantly
  reduces the burden on hardware
  designers during partial
  design mapping and marks 
  the first step in automated
  mapping for full designs.
  We next evaluate how \lr's program synthesis approach
  enables it to achieve greater completeness
  for these program fragments.
  
\paragraph{\textnormal{\textit{\textbf{Evaluation Setup.}}}}

We highlight three particularly complex
  DSPs for the Xilinx Ultrascale+,
  Lattice ECP5,
  and Intel Cyclone 10 LP architectures: 
  the Xilinx DSP48E2,
  Lattice ALU54A/MULT18X18C 
  (a single DSP composed of 
  two primitives),
  and Intel cyclone10lp\_mac\_mult.
SOFA provides no DSP, and is not included
  in this part of the evaluation.
For each architecture's DSP,
  we enumerate a large subset 
  of the designs
  theoretically mappable
  to a single DSP 
  according to its
  configuration manual.
This microbenchmark set
  aims to capture
  the real-world designs
  which hardware designers
  would attempt to map
  to a platform's DSP.
For each architecture, we compare \lr
  to both the corresponding
  state-of-the-art toolchain for
  the architecture 
  as well as to Yosys.
For Xilinx Ultrascale+, 
  the DSP48E2 configuration manual details the
  structure of designs mappable
  to the primitive.
Our designs for Xilinx include 
  all permutations of the design form
  $((a \pm b) * c) \odot d$,
  where $\odot \in \{\&, |, \pm, \oplus \}$,
  as well as designs of the forms $(a * b)$ and $((a * b) \pm c)$.
We pipeline each of these
  workloads from zero to three stages 
  and use bitwidths from 8 to 18 bits.
For the DSP on Lattice, we similarly enumerate
  all designs of the form $(a * b) \odot c$, 
  where $\odot \in \{\&, |, \oplus, \pm\}$, and of the form $(a * b)$.
For each of these designs, 
  we use zero to two stages
  and bitwidths from 8 to 18 bits.
This results in
    1320 microbenchmarks for Xilinx UltraScale+,
    396 for Lattice ECP5,
    and 66 for Intel Cyclone 10 LP.
Though \lr's output is correct by construction,
  we further validate its output
  by simulating each \lr-compiled design
  over thousands of consecutive cycles
  using Verilator.\tighten
  
\paragraph{\textnormal{\textit{\textbf{Comparison to Existing Toolchains.}}}}
As demonstrated in Figure~\ref{fig:xilinx-completeness} (top),
  \lr maps $44\times$ more
  designs than Yosys
  and $2.1\times$ more designs
  than the proprietary,
  state-of-the-art toolchain on Xilinx Ultrascale+.
On Lattice ECP5,
  \lr maps $6.0\times$ more
  designs than Yosys 
  and $3.6\times$ more designs
  than the proprietary,
  state-of-the-art toolchain.
On Intel Cyclone 10 LP,
  \lr successfully maps all designs:
  $3\times$ more designs
  than the proprietary,
  state-of-the-art toolchain for Intel.
Yosys fails to map a single design
  on Intel.
State-of-the-art toolchains
  for all architectures fail
  to map more than half
  of the queried designs.
\lr times out on 
less than 20\% of designs.%
\footnote{We restricted Rosette
synthesis time to 
120 seconds, 40 seconds, and 20 seconds for
Xilinx, Lattice, and Intel
respectively, and
marked failure past that (though bitvector synthesis problems
are decidable).}
Note that \lr
  returns ``UNSAT'' on 
  approximately 260
  designs on UltraScale+, i.e., 
  \lr claims there is
  \textit{no} possible mapping
  to a DSP48E2 for the
  requested workload.
In all of these cases,
  both Xilinx SOTA
  and Yosys
  agree with \lr 
  and do not map the designs
  to a single DSP.
We conclude that 
  the set of designs we presented in 
  \textit{Evaluation Setup}
  must be overly broad;
  though the documentation implies
  that all of these designs are mappable
  to a single DSP,
  all three Xilinx synthesis tools 
  surveyed indicate that they are
  indeed not mappable.

For timing, we compared the mapping time for each
  of the tools
  and report the results
  in Figure~\ref{fig:xilinx-completeness} (bottom).
The wide ranges for \lr
  show that solver time for different
  program synthesis queries
  is highly variable.
This is explored more deeply in
  \cref{fig:lakeroad-synth-time},
  which shows that most synthesis queries
  terminate quickly,
  with a long tail of slower queries.
Note that the state-of-the-art technology
mapper for Ultrascale+ has a slow running time
  due to its long start-up process.\tighten

Regarding which solvers
  in the portfolio were
  most useful,
  of all terminating
  (success or UNSAT)
  \lr experiments,
  Bitwuzla was the first
  to complete
  for 671 of them,
  STP for 519,
  Yices2 for 464,
  and cvc5 for 64.

\lr's greater completeness
  directly translates into resource reduction.
On average, for each microbenchmark,
  \lr uses 3.9 fewer LEs
  (logic elements: LUTs, muxes, or carry chains)
  and 7.5 fewer registers than the Xilinx SOTA,
  7.2 fewer LEs/11.9 fewer registers than the Lattice SOTA,
  8.2 fewer LEs/14.3 fewer registers than the Intel SOTA,
  and 33.3 fewer LEs/11.4 fewer registers than Yosys. 
In the real world, the small modules
  captured by our microbenchmarks
  may be reused dozens
  if not hundreds of times
  across a large design.
Thus, the sizable resource
  reduction \lr provides on a single
  microbenchmark
  will be multiplied significantly
  for an entire design.\tighten


\paragraph{\textnormal{\textit{\textbf{Discussion.}}}}
Compared to Yosys,
  it is clear that
  \lr provides more complete support
  for
  programmable DSPs.
However, \lr's greater completeness
  over Yosys
  is perhaps not surprising since 
  Yosys is an open-source tool
  still under active development.
Part of the appeal
  of the Yosys toolchain
  is the diversity of backends
  it can target;
  these results show that, if incorporated
  into Yosys, \lr
  would further increase
  Yosys's flexibility and generality.
Perhaps most surprising
  is that \lr is more complete
  than
  specialized proprietary toolchains. 
Even the UNSAT results \lr produces 
  can be useful to designers 
  since they indicate
  potential flaws
  in the documentation or vendor-provided semantics.
% Furthermore, 
%   the UNSAT results 
%   that \lr produces
%   %are demonstrative
%   reflect
%   of the complexity
%   of specialized primitives:
% Even the documentation,
%   which would ideally serve as
%   a single source of truth,
%   may not accurately
%   reflect the behavior of the primitive.
In the context of a larger
  synthesis tool, \lr
  would provide stronger
  guarantees for mapping
  modules of larger designs.






\begin{figure}
    \centering
\includegraphics[width=0.44\textwidth]{assets/succeeded\_vs\_failed\_all.png}

\vspace{1em}
\footnotesize

    \begin{tabular}{|l|S[table-format=3.2]|S[table-format=3.2]|S[table-format=3.2]|}
    \hline
     Tool & {Median Time (s)} & \multicolumn{2}{c|}{Min / Max Time (s)} \\ \hline
    \multicolumn{4}{|c|}{\textbf{Xilinx}} \\ 
    \hline
         \lr & 14.99 & 2.99 & 127.70 \\  \hline
         SOTA Xilinx & 261.61 & 227.82 & 598.67 \\ \hline
         Yosys & 14.97 & 6.66 & 21.10 \\ \hline
         %
    \multicolumn{4}{|c|}{\textbf{Lattice}} \\
    \hline
         \lr & 9.49 & 6.70 & 55.23 \\ \hline
         SOTA Lattice & 2.32 & 0.95 & 4.52 \\ \hline
         Yosys & 2.31 & 0.90 & 4.01 \\ \hline
    \multicolumn{4}{|c|}{\textbf{Intel}} \\
    \hline
         \lr & 2.92 & 2.12 & 4.13 \\ \hline
         SOTA Intel & 38.73 & 19.11 & 43.49 \\ \hline
         Yosys & 0.96 & 0.48 & 1.88 \\ \hline
    \end{tabular}

    % \begin{tabular}{|l|l|l::l|l|l|}
    % \hline
    % \multicolumn{3}{|c::}{\textbf{Xilinx}} & \multicolumn{3}{c|}{\textbf{Lattice}} \\
    % \hline
    %      Tool & Median Time (s) & Range (s) & Tool & Median Time (s) & Range (s)\\ \hline
    %      \lr & 6.02 & 1.55--182 & \lr & 5.45 & 4.39--186\\ \hline
    %      SOTA (Xil.) & 134 & 121--152 & SOTA (Lat.) & 0.777 & 0.593--1.04 \\ \hline
    %      Yosys & 3.57 & 2.95--4.45 & Yosys & 0.592 & 0.442--0.798 \\ \hline
    % \end{tabular}

    \caption{
Results of 
  the completeness experiments
  described in \cref{sec:completeness},
  measuring the completeness 
  of technology mapping tools for DSPs on Xilinx UltraScale+ and Lattice ECP5,
  plus timing information.
A single bar in the bar chart
  communicates,
  for a given FPGA architecture
  and technology mapper,
  the proportion of the microbenchmarks
  that the given technology mapper could map to a single DSP.
In \lr's case, experiments can 
  either succeed 
  (\lr maps the microbenchmark
    to a single DSP),
  timeout,
  or return UNSAT.
For the other tools,
  experiments can either
  succeed
  or fail
  (i.e., the tool returns a mapping,
    but the mapping uses more than
    a single DSP).
There are a total of 
  1320 experiments/microbenchmarks for Xilinx,
  396 for Lattice,
  and 66 for Intel.
    }
    \label{fig:xilinx-completeness}
\end{figure}
\begin{figure}
    \centering
    \includegraphics[width=0.43\textwidth]{assets/time/lakeroad\_time\_xilinx.png}
    \includegraphics[width=0.43\textwidth]{assets/time/lakeroad\_time\_lattice.png}
    \includegraphics[width=0.43\textwidth]{assets/time/lakeroad\_time\_intel.png}
\caption{
Histograms of \lr program synthesis runtime
  for all terminating
  (success or UNSAT) \lr experiments
  described in \cref{sec:completeness}, with timeout thresholds indicated with a vertical
  dotted red line.
% Note that these histograms capture only
%   the runtime of the underlying solvers;
%   however, in most cases, solver runtime
%   makes up the majority of \lr's runtime.
}
    \label{fig:lakeroad-synth-time}
\end{figure}

% \begin{table}[]
% \caption{This table summarizes the results
%   of portfolio solving, listing
%   for each solver the number
%   of terminating (success or UNSAT)
%   \lr experiments
%   that each solver was first to complete.}    
%   \label{tab:solver-counts}
%     \begin{tabular}{|c|c|}
% \hline
% \textbf{solver}&\textbf{\# experiments}\\
% \hline
% Bitwuzla & 671 \\
% STP&519 \\
% Yices2&464\\
% cvc5&64\\
% \hline
%     \end{tabular}
% \end{table}

\CatchFileDef{\LatticeResourceTable}{generated/lattice-resource-table.tex}{}
\CatchFileDef{\XilinxResourceTable}{generated/xilinx-resource-table.tex}{}
\CatchFileDef{\SofaResourceTable}{generated/sofa-resource-table.tex}{}
\CatchFileDef{\BigTable}{generated/ThE-BiG-TaBlE-WiTh-EvErYtHiNg-In-It.tex}{}
\CatchFileDef{\LatticeTable}{generated/LaTtIcE.tex}{}
\CatchFileDef{\XilinxTable}{generated/xilinx-non-auto.tex}{}
\CatchFileDef{\LatticeTable}{generated/lattice-non-auto.tex}{}
\CatchFileDef{\SofaTable}{generated/SoFa.tex}{}
\CatchFileDef{\SofaTable}{generated/sofa-non-auto.tex}{}
\CatchFileDef{\FinalTable}{generated/expressive-select.tex}{}
\newcommand{\titletilt}{90}


% \begin{table*}[t]
% \scriptsize{}
% \caption{Selected benchmark runtime and resource
%   utilization across Xilinx, Lattice, and SOFA
%   backends. 
%   All instructions are for 32-bit 
%   inputs and outputs; we ran these instructions 
%  %\sandy{"and similar"?} 
%   and similar
%   for 8- and 16-bit inputs and outputs.}
% \label{tab:expressiveness-table}
% \newcolumntype{?}{!{\vrule width 1pt}}
% \begin{center}
% \resizebox{\textwidth}{!}{%
% \setlength{\extrarowheight}{2pt}
% \setlength{\tabcolsep}{2.6pt}
% \begin{tabular}{ll||rrr|rrrrr|rrr|c||rrr|rrrr|rrr|c||cc|c|c}
% \toprule
% \multicolumn{2}{c}{} & \multicolumn{12}{c}{{\bf Xilinx Ultrascale+}} & \multicolumn{11}{c}{{\bf Lattice ECP5}} & \multicolumn{4}{c}{{\bf SOFA}} \\ 
% \cmidrule(r){3-14} \cmidrule(r){15-25} \cmidrule(r){26-29}
% && \multicolumn{3}{c}{\lr}
%  & \multicolumn{5}{c}{Yosys}
%  & \multicolumn{3}{c}{SOTA Xil.} 
%  & \multicolumn{1}{c}{SOTA Lat.}
%  & \multicolumn{3}{c}{\lr}
%  & \multicolumn{4}{c}{Yosys}
%  & \multicolumn{3}{c}{SOTA Lat.} 
%  & \multicolumn{1}{c}{SOTA Xil.}
%  & \multicolumn{2}{c}{\lr}
%  & \multicolumn{1}{c}{SOTA Xil.} 
%  & \multicolumn{1}{c}{SOTA Lat.}\\
%  \midrule
 
% \FinalTable{}\\

% \bottomrule
% \end{tabular}}
% \end{center}
% \end{table*}

\subsection{\lr Extensibility and Expressiveness}

In addition to being
  correct by construction (\cref{sec:formalization}) and
  more complete 
  than existing FPGA technology mappers (\cref{sec:completeness}),
  \lr can also easily extend to new FPGA architectures.
Furthermore, automatic primitive semantics extraction 
  from vendor-provided HDL simulation models
  enables \lr to support diverse, highly configurable
  FPGA primitives.


The architecture descriptions
    vary in length from 20 to 240
    source lines of code (SLoC).
% \cref{table:loc} shows that the
%   architecture description for each
%   target FPGA architecture varies from
%   20 to 240 lines of code.
%
SOFA (20 SLoC) is the simplest, shown in full
  in \cref{fig:sofa-architecture-description}.
The descriptions for Xilinx (185 SLoC), 
  Lattice (240 SLoC), and Intel (178 SLoC)
  are longer since those
  FPGA architectures provide a
  wider range of configurable primitives.
% Though a user study is beyond the scope of this paper,
%   anecdotally, we found these
%   architecture descriptions straightforward to develop:
%   for each primitive,
%   the user includes a series of fields 
%   %in a YAML file
%   that describe
%   the ``type'' of its interface
%   (ports and parameters) as well as 
%   the path to a (vendor-provided) behavioral HDL model
%   that specifies its behavior.

As a point of comparison,
  the open-source Yosys toolchain,
  which has roughly 200 contributors on GitHub,
  provides technology mapping
  for Xilinx UltraScale+
  across over a dozen complex
  Verilog, C++, and Python files (about 1300
  lines of code).
We cannot provide similar numbers
  for state-of-the-art proprietary tools,
  but a developer
  of one such technology mapper
  shared that extending their tool to
  support new FPGA architectures
  was extremely difficult since it 
  ``spans millions of lines of low-level C.''
This is not surprising; Yosys aims to
  target a variety of vendor architectures, 
  while proprietary tools have teams of
  engineers to extract better mapping 
  (evident by Yosys' limitations
  in \cref{sec:completeness}).
By contrast,
  \lr supports
  diverse architectures and
  is easy to extend.
Even if a user
  wants to target a completely
  new architecture that
  \lr does not support,
  architecture-independent
  sketch templates allow reuse
  of previously implemented mapping
  strategies, and the user is
  only required to provide
  a few lines of
  high-level configuration
  for each primitive in 
  the architecture description.
  
\cref{table:imported-primitves} further
  highlights \lr's expressiveness,
  i.e., its ability to support a diverse
  range of configurable primitives
  by automatically extracting semantics from
  vendor-provided HDL simulation models.
\lr can import the semantics
  of large configurable primitives, 
  such as the UltraScale+ DSP (896 lines of Verilog)
  or Lattice ECP5's ALU and multiplier units (1642 and
  795 lines of Verilog, respectively).
It is difficult and error-prone
  to manually formalize the full semantics for these primitives;
  partial support by ad hoc search procedures
  that rely on syntactic pattern matching
  leads to missing many mapping opportunities,
  as shown in \cref{sec:completeness}.

% \paragraph{Extending to LUTs.}

% Though \lr is primarily designed to
%   support programmable primitives like DSPs,
%   our approach can also support more basic elements like
%   LUTs and carry chains.
% \cref{tab:expressiveness-table} shows
%   the results of using \lr to
%   map basic arithmetic and comparison operators
%   to these elements on our target FPGAs.
% Baselines include
%   Yosys and state-of-the-art tools
%   for Lattice and Xilinx FPGAs.
% Even with basic sketches and modest effort,
%   \lr runs quickly and often produces results
%   comparable to highly optimized tools.


%%%%%%%%%%%%%%%%%%%% LoC Table %%%%%%%%%%%%%%%%%%%%
% \begin{table}
% \caption{Number of source lines of code (excluding comments and empty lines) for each architecture description.}
% \centering
% \footnotesize{}
% \label{table:loc}
% \begin{tabular}{l r}
%  {\bf FPGA}   & {\bf SLoC} \\\hline
%  Xilinx Ultrascale+     &  185       \\
%  Lattice ECP5           &  240       \\
%  Intel Cyclone 10 LP    &  178       \\
%  SOFA                   &  20        \\
% \end{tabular}
% \end{table}

\begin{table}
\caption{FPGA primitives imported 
  automatically by \lr from vendor-provided 
  Verilog models, with number of source lines of code 
  (excluding comments and empty lines) of 
  the original Verilog models.} 
\centering
\footnotesize
\label{table:imported-primitves}
\begin{tabular}{lrr}
 {\bf FPGA}   & \textbf{Primitive} & {\bf Verilog SLoC} \\\hline
 Xilinx Ultrascale+  & LUT6 & 88       \\
                     % & LUT6_2 & 123 \\ 
                     & CARRY8 & 23       \\
                     & DSP48E2 & 896       \\
                     % & \\
                     \hline
 Lattice ECP5 & LUT2 & 5       \\
              & LUT4 & 7       \\
              & CCU2C & 60       \\
              % & ALU24B & 672       \\
              % & ALU54B & 1942 \\
              & ALU54A & 1642 \\ 
              % & MULT18X18D & 992       \\
              & MULT18X18C & 795 \\
              % & \\
              \hline
 Intel Cyclone 10 LP  & cyclone10lp\_mac\_mult   &  319       \\ 
              % & \\
              \hline
 SOFA      & frac\_lut4   &  69       \\ 
\end{tabular}
\end{table}










%%%%%%%%%%%%%%%%%%%%%%%%%%%%%%%%%%%% 
% BEGIN COMMENTING OUT OF BIG TABLE
%%%%%%%%%%%%%%%%%%%%%%%%%%%%%%%%%%%% 
\iffalse
%%%%%%%%%%%%%%%%%%%%%%%%%%
% All Resources Table
\begin{table*}[t]
\caption{Tool runtime
  and FPGA resource utilization
  for \lr, Vivado, Diamond, and Yosys,
  when generating implementations for our set of instructions to  
  Xilinx UltraScale+, Lattice ECP5, and SOFA.
The ``Template'' column
  displays which template
  was used for the \lr implementation:
  BW is bitwise, Carry is bitwise with carry, 
  Mult is multiplication,
  Comp is comparison,
  and DSP is the DSP template.
A check in the
  ``\textit{DSP?}'' columns
  indicates that,
  in addition to finding an implementation
  with the listed template,
  \lr was also able to implement the instruction
  with the DSP template.
  Note that neither of the proprietary
  Vivado nor Diamond tools are portable --
  they only narrowly work for their own
  respective FPGA architecture.
  Furthermore, Yosys support for SOFA
  is unusable.
  Only Lakeroad was able to complete
  instruction synthesis across all three
  architectures.
  }
\label{tab:resource-utilization}

\newcolumntype{?}{!{\vrule width 1pt}}
\begin{center}
\scriptsize
\begin{tabular}{ll||rrr:c|rrrrrr|rrrr|c||rrrr|rrrrr|rrrr|c||rr}
\toprule
\multicolumn{2}{c}{} & \multicolumn{15}{c}{{\bf Xilinx Ultrascale+}} 
 & \multicolumn{14}{c}{{\bf Lattice ECP5}} & \multicolumn{2}{c}{{\bf SOFA}}\\
\cmidrule(r){3-17}
\cmidrule(lr){18-31}
\cmidrule(l){32-33}
&& \multicolumn{4}{c}{Lakeroad}
 & \multicolumn{6}{c}{Yosys}
 & \multicolumn{4}{c}{Vivado} 
 & \multicolumn{1}{c}{Diam.}
 & \multicolumn{4}{c}{Lakeroad} 
 & \multicolumn{5}{c}{Yosys}
 & \multicolumn{4}{c}{Diamond} 
 & Viv.
 & \multicolumn{2}{c}{Lakeroad}\\
 \midrule
%%%%%%%%%%
% \LatticeResourceTable{}
\BigTable{}\\

\bottomrule
\end{tabular}
\end{center}
\normalsize
\end{table*}
\fi
%%%%%%%%%%%%%%%%%%%%%%%%%%%%%%%%%%%% 
% END COMMENTING OUT OF BIG TABLE
%%%%%%%%%%%%%%%%%%%%%%%%%%%%%%%%%%%% 