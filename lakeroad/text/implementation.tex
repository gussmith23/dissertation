\section{Implementation}
\label{sec:implementation}

% \Cref{sec:overview,sec:formalization} 
%   cover the core structure of \lr;
%   it is our intent that the tool
%   should be re-implementable
%   from just these sections.
% In this section,
%   we describe implementation details
%   that may be useful
%   when reimplementing \lr.

\lr is composed of
  approximately 13K lines of Racket
  plus approximately 58K lines
  of Racket
  automatically generated from 
  vendor-supplied Verilog.
Vendor-supplied Verilog
  was obtained from Lattice Diamond,
  Intel Quartus,
  and Xilinx Vivado
  sources.
%   as well as
% \lr uses the 
%   Rosette synthesis 
%   tool~\cite{torlak2014lightweight}.
We used Vivado version v2023.1,
  Quartus 22.1std.1 Build 917 02/14/2023 SC Lite Edition,
  Diamond version 3.12,
  Yosys version 0.36+42 (commit \texttt{70d3531}),
  the cvc5~\cite{barbosa22cvc5} and Yices2~\cite{dutertre2006yices,dutertre2014yices}
  solvers
  included in the 2023-08-06 release of \texttt{oss-cad-suite} from YosysHQ,
  the Bitwuzla solver at commit \texttt{b655bc0}~\cite{bitwuzla},
  the STP solver at commit \texttt{0510509a},
  Racket version 8.9~\cite{racket,racket:ref},
  and Rosette version 4.1~\cite{rosette4}.

\subsection{Primitive Interfaces}
\label{sec:impl-primitive-interfaces}

As described in \cref{sec:overview},
\textit{primitive interfaces}
  describe abstract versions
  of common FPGA primitives,
  which allow sketch templates
  to be architecture-independent.
To date, \lr declares primitive interfaces for
  $n$-input LUTs, $w$-width carry chains, 
  $n$-input muxes,
  and DSPs with up to four data inputs and one clock input.
The next section includes a concrete example
  of \lr's LUT4 primitive interface.\tighten

% Sketch templates are nearly valid
%   \lrir expressions, but with
%   architecture-independent
%   primitive interfaces
%   in place of architecture-specific
%   primitive interfaces.
% When \lr specializes a sketch template
%   into a sketch,
%   it simply converts each
%   instance of a
%   primitive interface
%   into the corresponding architecture-specific
%   counterpart.
% The information required to do this conversion
%   is exactly what is specified
%   in the architecture description.


\subsection{Architecture Descriptions}
\label{sec:impl-arch-descs}

\begin{figure}
\begin{minted}[fontsize=\footnotesize]{YAML}
implementations:
  - interface: { name: LUT, num_inputs: 4 }
    internal_data: { sram: 16 }
    modules:
      - module_name: frac_lut4
        filepath: SOFA/frac_lut4.v
        ports:
          - { name: in, direction: in, width: 4, 
              value: (concat I3 I2 I1 I0) }
          - { name: mode, direction: in,
              width: 1, value: (bv 0 1) }
          - { name: lut4_out, direction: out,
              width: 1 }
        parameters: [{ name: sram, value: sram }]
    outputs: { O: lut4_out }
    \end{minted}
    \caption{
SOFA architecture description.
% The SOFA architecture
%   description uses
%   the SOFA \texttt{frac\_lut4}
%   primitive to implement
%   \lr's 4-input LUT 
%   primitive interface. 
% Inputs 
%   and outputs of the primitive
%   interface are mapped to the 
%   ports of the \texttt{frac\_lut4}. 
% The LUT's configurable inputs
%   are also specified: \texttt{frac\_lut4}
%   allows programming 
%   the LUT function by defining
%   its memory input \texttt{sram}.
  }
    \label{fig:sofa-architecture-description}
\end{figure}


\noindent As described in \cref{sec:overview},
  \textit{architecture descriptions}
  convey the information
  required to convert
  each instance of a primitive interface
  into the corresponding
  architecture-specific module,
  which occurs while converting
  sketch templates
  into sketches.
The architecture description is
  the only additional input that
  may be required from a user
  to support a new architecture;
  it is a one-time effort 
  that is reusable
  for any designs in an architecture.
% Each architecture description is
%   a 20--250-line YAML file, and is
%   an easy one-time development specification
%   that is reusable for any designs on an architecture, and
%   many are already provided in \lr's prototype.
Architecture descriptions
  are simply lists
  (provided as YAML files)
  of the primitive interfaces
  that an architecture implements,
  but, crucially,
  also include 
  architecture-specific
  port and parameter values
  in a map called \texttt{internal\_data}.
Values in this map
  become symbolic values
  solvable by the SMT solver.
%\lr also allows users to specify additional constraints
Additional constraints can also be specified
  in the architecture description 
  to rule out invalid configurations
  and minimize the solver's search space.\tighten

As an example,
  \cref{fig:sofa-architecture-description}
  shows the architecture description
  for the SOFA~\cite{sofa}
  FPGA architecture.
The description contains
  a single primitive interface implementation, i.e., 
  LUT4.
\lr's LUT4 primitive interface
  standardizes the names of a LUT4's inputs and outputs,
  naming the inputs
  \texttt{I0} through \texttt{I3}
  and the output
  \texttt{O}.
The SOFA implementation 
  of the LUT4 primitive interface
  uses
  the SOFA-specific
  \texttt{frac\_lut4}
  primitive.
Primitive interface inputs \texttt{I0} through \texttt{I3}
  are mapped to
  the actual
  input port of the \texttt{frac\_lut4},
  named \texttt{in}.
Likewise, the \texttt{frac\_lut4} output
  \texttt{lut4\_out}
  is mapped to the primitive interface output \texttt{O}.
The \texttt{internal_data} field
  declares \texttt{sram},
  the LUT's 16-bit internal memory,
  as an architecture-specific detail
  to be solved during synthesis.
 
If a sketch template uses a primitive interface
  not included in the architecture description
  (e.g., SOFA does not implement carries),
  \lr may still be able to implement the primitive interface
  based on primitive interfaces
  the architecture \textit{does} implement.
To date, \lr can implement any mux
  with LUTs, 
  a larger LUT from  smaller LUTs,
  a smaller LUT from a larger LUT,
  a carry from LUTs,
  and a smaller DSP from a larger DSP;
  it handles these conversions during sketch generation.


\subsection{Sketch Templates, Sketches, and Sketch Generation}

\label{sec:impl-sketch-templates}

As described in \cref{sec:overview},
  \lr 
  captures common FPGA implementation patterns
  in reusable, architecture-independent
  \textit{sketch templates.}
Thus far, we have described only 
  the relatively simple
  \texttt{dsp} sketch template,
  which instantiates a DSP.
As a more complex example
  of capturing common FPGA implementation patterns,
  consider
  the \texttt{bitwise-with-carry}
  sketch template, which
  uses $n$ LUTs and a carry chain
  to implement designs such as 
  addition or subtraction.
As of the paper's publication date, 
  \lr provides 5 sketch templates: 
  \texttt{dsp},
  \texttt{bitwise}, \texttt{bitwise-with-carry},
  \texttt{comparison} (LUT- and carry-based arithmetic comparison), 
  and
  \texttt{multiplication} (LUT-based multiplication).
 
The process of converting sketch templates
  to sketches
  is implemented as described in
  \cref{sec:overview}
  and \cref{sec:impl-arch-descs}.
\lr iterates over every
  primitive interface instance
  in the sketch
  and replaces it with
  the concrete primitive
  in accordance with
  the architecture's 
  architecture description.
If the architecture description
  does not implement the requested
  primitive interface,
  \lr checks whether it can implement
  the primitive interface with other
  implemented interfaces
  (e.g., implementing a smaller LUT with
  a larger LUT)
  and raises an error otherwise.

Sketch templates and sketches alike 
  are written in a domain-specific language (DSL)
  embedded into Rosette,
  whose implementation closely mirrors the syntax
  and semantics of \UberLang. 
The only significant difference is that
  the interpreter implementation
  does not use bitvector streams natively.
Instead, each invocation of the interpreter
  represents a single timestep,
  and all intermediate values from the previous timestep
  are taken as input.
Streams are then built up
  using multiple invocations of the interpreter.

  
\subsection{Importing Semantics from Verilog Modules}
\label{sec:implementation:importing-semantics}

\lr uses 
  Yosys~\cite{wolf2013yosys}
  to convert Verilog modules
  into the \texttt{btor2} format~\cite{btor} % Btor2 , BtorMC and Boolector 3.0
  and then converts the resulting \texttt{btor2}
  to Rosette/Racket code.

Due to the semantics of the Verilog language
  and the internal implementation of Yosys,
  extracting semantics from Verilog modules
  may require the following manual modifications
  to accommodate semantics extraction and synthesis:
  % some manual modifications generally
  % need to be made
  % to Verilog modules
  % for them to be importable by \lr,
  % and for their semantics to be useful:
  
\begin{itemize}[leftmargin=*]
\item As Yosys converts
  parameters from variables
  to constant values
  immediately upon module import,
  module parameters should be converted to
  ports
  to ensure they remain variables
  (and thus solvable by the SMT solver).
Note that not all parameters 
  can always be converted to ports,
  meaning some parameters cannot be solved for.
\item Strings should be converted to bitvectors.
% \item Multiplication expressions
%   may need to be coerced
%   to generate bitvector expressions
%   that do not cause pathological SMT solving
%   behavior.
\item All registers should be initialized.
\item All instances of \texttt{x} and \texttt{z} values should be  
  converted to 2-state logic (0 or 1).
\end{itemize}
Note that these caveats
  apply only 
  to our prototype implementation,
  not the general technique
  of semantics extraction from HDL.
Once these manual modifications are made,
  the following series of Yosys passes
  can be used to convert the Verilog
  into suitable \texttt{btor2}:
\texttt{prep; flatten; pmuxtree; opt_muxtree; clk2fflogic; prep; write_btor}.

We implement
  the translation from \texttt{btor2}
  to Rosette bitvector expressions
  as a 1:1 translation 
  since both languages
  are simply operations over bitvectors.


% \subsection{Sketch Generation}
% \label{sec:impl:sketch-generation}

% Sketch generation
%   converts architecture-independent
%   \textit{sketch templates}
%   into architecture-specific
%   \textit{sketches} on which
%   we can perform
%   program synthesis.
  
% \lrir sketches are 
%   \lrir programs with
%   holes.
% A \textit{hole} is an
%   unspecified part of
%   a program that will be
%   filled in by program synthesis.
% Rosette provides
%   utilities like
%   \texttt{define-symbolic}
%   for creating symbolic variables
%   for the solver to solve for
%   and \texttt{choose}
%   for allowing the solver to choose
%   between multiple potential expressions.

% Sketches use \lrir's \texttt{primitive} construct
%   which represents a architecture-specific
%   FPGA primitive.
% These primitives often need to
%   be configured (e.g., a LUT's memory
%   or a DSPs parameters),
%   and rather than requiring the user to
%   configure these manually we rely on Rosette's
%   program synthesis to automatically
%   discover configuration.

% While sketches are architecture-specific,
%   their general shape is often shared
%   between architectures.
% To prevent users
%   from having to write
%   new sketches
%   for every new architecture,
%   \lr
%   provides \textit{sketch templates,}
%   which capture high-level design patterns
%   common across many 
%   FPGA architectures.

% \subsubsection{Sketch Templates}
% \label{sec:templates}

% Sketch templates abstract away concrete
%   computational units with primitive interfaces
%   that describe a class of computation
%   (e.g., a carry chain, or a LUT, or a DSP)
%   at a high level.
% Users of \lr need to specify how their
%   FPGA's hardware components map to these
%   primitive interfaces in an architecture
%   description.

% \lr provides a number of architecture-independent
%   sketch templates,
%   which capture common implementation
%   patterns
%   useful across FPGA architectures.
% Sketches are \lrir programs
%   with instances of 
%   architecture-specific
%   \texttt{primitive} expressions
%   replaced with 
%   architecture-independent
%   \textit{primitive interface}
%   expressions.
% When a template is instantiated,
%   \lr replaces
%   each primitive interface instance
%   with an architecture-specific
%   primitive interface---%
%   a process described in detail
%   in \cref{sec:interfaces}.
% Once all primitive interfaces
%   are instantiated
%   to architecture-specific primitives,
%   the template has been converted to an
%   architecture-specific \lrir
%   sketch.
% The sketch can then be interpreted
%   with the
%   \lrir interpreter
%   and run through synthesis
%   with Rosette.

% We will describe \lr's implemented templates
%   in \cref{implementation:implemented-templates}.
% However, before we do that,
%   we will explain primitive interfaces
%   in detail,
%   as they lie at the core of
%   sketch templates.


\subsection{Program Synthesis and Compilation to Verilog}
\label{sec:implementation:program-synthesis}

We implement
  the synthesis procedure
  defined in \cref{subsec:lr-correctness-and-completeness}
  with Rosette.
Multiple clock cycle guarantees,
  as described in \cref{subsec:lrfn-bmc},
  are implemented simply by making $c+1$
  total assertions,
  asserting the output of the input design
  and the sketch are equal
  after each of the $c+1$ timesteps.
We use a portfolio solving
  method, running
  Bitwuzla~\cite{niemetz2020bitwuzla},
  cvc5~\cite{barbosa22cvc5},
  Yices2~\cite{dutertre2006yices,dutertre2014yices},
  and STP~\cite{stp}
  in parallel
  and using results from the first
  solver to terminate.
To produce Verilog,
  \lr compiles the program from its internal DSL
  to the JSON format defined
  by Yosys using a straightforward translation 
  and then uses Yosys to output Verilog.
% Compilation from \lrir
%   to Yosys's JSON format
%   is a straightforward translation.


\subsection{Integration with Other Tools}

This paper describes \lr
  as a standalone tool,
  but the core \lr implementation
  could be integrated directly into
  existing tools.
Though out of scope for this paper,
  we have early,
  encouraging results
  integrating \lr
  as a Yosys pass
  that lets users tag modules
  with annotations similar to 
  (and much richer than) Xilinx's
  \texttt{use\_dsp} annotation.
We then map annotated modules 
  to primitives using \lr,
  which let us easily apply \lr to
  many fragments within a larger design.
We plan to more fully
  integrate \lr into Yosys in future work,
  which should radically improve the completeness
  of Yosys's DSP mapping ability,
  as shown in \cref{fig:xilinx-completeness}.\looseness-1
