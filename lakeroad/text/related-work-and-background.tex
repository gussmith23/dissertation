\section{Related Work}
\label{sec:background-and-related-work}

To the best of our knowledge,
  \lr is the first work
  to apply the technique of program synthesis
  to FPGA technology mapping.
Indeed, as noted by Sisco \textit{et al.}~\cite{sisco2022synthesis},
  program synthesis has seldom been applied
  in the domain of hardware design although
  its underlying formal methods techniques
  are frequently used for
  the \textit{formal verification}
  of hardware designs rather than compilation,
  % \footnote{Program synthesis and verification are inherently related, since synthesis entails a verification step: The design ultimately produced must be checked against the specification.}
  as in Bluespec SystemVerilog~\cite{nikhil2004bluespec},
  \koika~\cite{bourgeat2020essence},
  and Kami~\cite{choi2017kami}.
Sisco \textit{et al.} cite two examples
  of works that use program synthesis for hardware design,
  Verisketch~\cite{ardeshiricham19verisketch} and Sketchilog~\cite{becker14sketchilog},
  both of which apply program synthesis to produce HDL implementations from high-level designs.
Other works use program synthesis
  to generate \textit{software}
  that runs on low-powered hardware,
  like Chlorophyll~\cite{phothilimthana2014chlorophyll},
  which targets extremely memory-constrained
  power-efficient processors,
  Chipmunk~\cite{gao2019chipmunk},
  which targets programmable network switches,
  and Diospyros~\cite{vanhattum2021vectorization},%
  \footnote{Diospyros uses symbolic evaluation, which is related to program synthesis, to lift imperative programs for digital signal processors into a high-level mathematical representation that can then be used with the technique of equality saturation~\cite{tate2011equality} to generate optimized code for the target devices. This is also distinct from the program synthesis techniques referenced elsewhere in this paper.}
  which generates vectorized programs for standalone digital signal processors (more powerful and general-purpose devices than the DSP units in FPGAs).
These works demonstrate the utility of program synthesis 
  for generating code that handles
  specific wrinkles in hardware designs,
  as does the use of program synthesis in \lr
  to harness the programmability of FPGA DSPs.

\lr is also related to past work in FPGA compilation and techmapping,
  much of which does not 
  %\sandy{Are you using 'entreaty' to mean 'attempt'? Can you simply use the latter?}
  entreaty to support
  programmable DSPs with as much generality.
ODIN~\cite{jamieson2005verilog} and ODIN-II~\cite{jamieson2010odin}
  are used in \textit{hard-block synthesis}
  for FPGAs,
  which is the task of mapping portions
  of hardware designs to specialized units (\textit{hard blocks}) like multipliers.
They operate purely over syntax (e.g., mapping \texttt{*} to a multiplier)
  and so are greatly limited in their ability
  to handle programmable DSPs.
The ABC~\cite{brayton2010abc} logic synthesis tool
  is used to lower hardware designs 
  into LUT and carry-chain configurations; it 
  is related to \lr in that it also uses constraint solvers
  to find configurations,
  though it is not general enough to handle
  a wide variety of programmable DSPs,
  unlike the program synthesis techniques used in \lr.
Note also that the use of
  configuration files in \lr to
  abstract away details of the FPGA architecture
  was inspired by past work in FPGA compilation,
  including OpenFPGA~\cite{tang2019openfpga}
  and the Verilog-to-Routing
  project (VTR)~\cite{rose2012vtr},
  both of which use abstract architecture descriptions
  to facilitate portability across designs,
  though these projects are limited in their support for DSPs.
Library-Parameterized Models~\cite{1993lpm, lpmaltera}
  define generic interfaces for common primitives and are also similar to \lr's primitive interfaces,
  though they are limited in their ability to represent configurable units like DSPs.\tighten


  


% While much work exists that applies formal methods to hardware,
%   most of this work
%   has concerned the
%   \textit{formal verification}
%   of hardware designs,
%   viz., proving that an implementation of a design
%   fulfills a specification for it.
% Examples of such work include
%   the HOL, FORTE, and VOSS proof assistants~\cite{seger2005industrially,joyce2005hol,githubGitHubTeamVossVossII},
%   used by Intel, among others.

% It should be noted, nevertheless,
%   that while program synthesis
%   has had limited use in the hardware domain,
%   the formal methods techniques
%   that underlie program synthesis
%   have frequently been used
%   for \textit{formally verifying}
%   hardware designs against machine-readable specifications.
% For example, the HOL/FORTE/VOSS/VOSS II~\cite{seger2005industrially,joyce2005hol,githubGitHubTeamVossVossII} proof assistants
%   have been widely used
%   at Intel and other companies
%   for verifying properties
%   of hardware designs.
% Synthesis and verification are inherently related,
%   since synthesis includes a verification step
%   (it must be determined that the proposed design fulfills the specification).
% Verification techniques
%   check that an existing design fulfills a specification;
%   synthesis entails filling the holes in a proposed design
%   in order for the specification to be fulfilled (thus requiring the generated design to also be checked against the specification).

% Much work exists
%   applying formal methods
%   for \textit{formal verification}
%   of hardware;
%   see, for example,
%   work on HOL/FORTE/VOSS/VOSS II ~\cite{seger2005industrially,joyce2005hol,githubGitHubTeamVossVossII}
%   used by Intel,
%   among others.
% Formal verification and program synthesis
%   are two sides of the same coin;
%   formal verification checks whether,
%   given a fully specified design,
%   the design implements a given
%   specification,
%   while \lr's application of program
%   synthesis
%   checks whether, given a design with
%   ``holes,''
%   the holes can be filled in
%   to generate a design which implements
%   the given specification.

% comparable to Lakeroad because it takes an architecture description and gives you a compiler
% However, doesn't support complex constructs like DSPs, mainly just LUTs
% Might be conceivable to explore combining OpenFPGA's generality with Lakeroad's support for DSPs
% OpenFPGA~\cite{tang2019openfpga}
%   is a framework
%   for the rapid prototyping of FPGA fabrics.
% OpenFPGA includes a method of describing architectures
%   as architecture description files
%   in an XML format,
%   and enables compilation of designs
%   (including technology mapping)
%   to any architecture represented in this format.
% This architecture description
%   and compilation flow
%   is implemented with and originally proposed by
%   the Verilog-to-Routing
%   project (VTR)~\cite{rose2012vtr}.
  
% VTR is a large project
%   with the general goal 
%   of automating the generation of
%   FPGA synthesis tools
%   given just a description
%   of the FPGA architecture.
% VTR is powered by three main tools:
%   ABC~\cite{brayton2010abc}
%   for logic synthesis,
%   ODIN-II~\cite{jamieson2010odin}
%   for hard-block synthesis,
%   and VPR~\cite{betz1997vpr}
%   for physical synthesis.
% Both ABC and ODIN-II are relevant to this work. 

  
% ODIN-II~\cite{jamieson2010odin}
%   and its predecessor ODIN~\cite{jamieson2005verilog}
%   are tools
%   for \textit{hard block synthesis,}
%   which is the process of mapping a hardware design
%   to specialized units,
%   or \textit{hard blocks,}
%   such as multipliers.
% ODIN and ODIN-II
%   operate purely over \textit{syntax:}
%   e.g. searching for a literal ``\texttt{*}''
%   symbol
%   in Verilog code and mapping it
%   to a multiplier.
% This may work
%   for an FPGA architecture
%   with a small number
%   of non-programmable units
%   (e.g., fixed multipliers, fixed adders),
%   but is not suitable for programmable units
%   such as DSPs, which
%   can implement an ever-expanding
%   range of computations;
%   the number of patterns a user would need to write
%   would be immense.
% \lr, on the other hand,
%   operates over \textit{semantics:}
%   the low-level meaning
%   of a hardware design.
% It is thus automatically able to determine
%   the functioning of a DSP,
%   and able to find opportunities
%   ODIN would miss.
  
% Library Parameterized Models~\cite{1993lpm, lpmaltera}
%   are similar to \lr's
%   primitive interfaces:
%   they define generic interfaces for 
%   common primitives in FPGAs,
%   which can then be mapped to FPGA-specific
%   implementations.
% They do not, however,
%   have the ability to represent configurable
%   units such as LUTs and DSPs.

Virtual FPGA overlays~\cite{lysecky2005firm,brant2012zuma, landgraf2021compiler}
  are another approach to
  improving the mapping of hardware designs
  to hardware.
Overlays present a ``virtual''
  FPGA architecture;
  each actual architecture
  must then define a mapping
  from virtual to actual primitives.
This required translation is similar to
  \lr's requirement on users
  to implement primitive interfaces
  in an architecture description,
  though it requires more user effort.
The translation from virtual to actual architecture
  often comes with
  a steep resource
  and performance overhead.
  % not dissimilar to the overheads
  % incurred by \lr, e.g.,
  % in the imperfect alignment
  % of Lattice's CCU2C primitive
  % with \lr's carry-chain primitive interface.

% Faster and
%   more flexible
%   hardware compilation
%   is so desirable
%   that 
%   a number of approaches 
%   go so far as to
%   propose special 
%   hardware for which
%   compilation is easier%
%   ~\cite{xiao2022pld,lysecky2004configurable,sekanina2000r,lysecky2004dynamic,stitt2003dynamic}.

% There is a large body of work
%   focusing on the semantics of hardware designs;
%   see
%   Bluespec SystemVerilog~\cite{nikhil2004bluespec},
%   \koika~\cite{bourgeat2020essence},
%   and Kami~\cite{choi2017kami}.
% Unlike \lr, these works generally focus 
%   on 
%   \textit{verification} of existing hardware designs,
%   however,
%   rather than \textit{compilation.}
  


%\subsection{Hardware Compilation}
%\label{sec:background-hardware-compilation}
%
%Xilinx's
%  Vivado Design Suite~\cite{vivado}
%  is related.
%So are a number of open source tools~\cite{rose2012vtr, lavin2018rapidwright, wolf2013yosys, eldridge2021mlir,urbach2022hls}.
%CIRCT is part of MLIR~\cite{lattner2021mlir}.
%
%In addition to focusing
%  on \textit{usability,}
%  there is a focus on
%  \textit{correctness}
%  in new hardware compilers.
%\cite{herklotz2021empirical} finds that
%  a number of commercial HLS compilers
%  produce incorrect hardware
%  in 2.5\% of cases;
%  as a result,
%  \cite{herklotz2021formal} presents
%  the first formally-verified
%  HLS compiler from C
%  by building a Verilog backend
%  for the formally verified CompCert
%  C compiler~\cite{leroy2016compcert}.
%
%  
%
%New approaches to compilation
%  often take issue
%  with a fundamental step
%  in traditional hardware compilation,
%  in which the input design---%
%  represented in a high-level,
%  semantically rich language
%  like behavioral SystemVerilog---%
%  is compiled all the way down
%  to a low-level representation
%  like and--inverter graphs~\cite{callahan1998fast}.
%This makes the process of
%  technology mapping
%  for FPGAs challenging,
%  as the low-level representation
%  must be lifted up
%  to the 
%  higher level of abstraction
%  of FPGA primitives,
%  such as look-up tables (LUTs)
%  or even digital signal processors (DSPs).
%Reticle~\cite{vega2021reticle},
%  which our work builds directly off of,
%  introduces a new representation
%  which is compiled directly to
%  FPGA primitives
%  via instruction selection.
%
%\subsection{Formal Hardware Models}
%\label{sec:background-formal-hardware-models}
%
%VHDL~\cite{ieee20191076}
%  and Verilog/SystemVerilog~\cite{design2005ieee}.
%Also \cite{choi2017kami}.
%
%Bluespec SystemVerilog~\cite{nikhil2004bluespec}
%  raises the abstraction level
%  of SystemVerilog.
%\koika~\cite{bourgeat2020essence} introduces
%  more control over scheduling
%  which depends on dynamic analyses
%  which might incur hardware overhead
%  but can be simplified away
%  potentially by static analyses.
%Kami~\cite{choi2017kami}
%  is a recent formalization
%  of a subset of Bluespec
%  in the theorem prover Coq.
%Being embedded in Coq,
%  Kami allows for 
%  verification---%
%  in the software sense of the word,
%  that is, formal verification
%  of mathematical properties---%
%  of hardware designs
%  by stating and proving theorems
%  about them.
%Kami can also be compiled
%  to FPGAs.
%
%A number of recent representations
%  enrich
%   hardware design representations
%  with new semantics.
%Calyx~\cite{nigam2021compiler}
%  explicitly separates
%  datapath specification
%  from control logic.
%Dahlia~\cite{nigam2020predictable}
%  uses affine types
%  to more richly describe
%  the semantics of
%  reading to 
%  and writing from
%  memories
%  in FPGAs.
%Aetherling~\cite{durst2020type}
%  presents a
%  language
%  which captures
%  the spatial/temporal tradeoff
%  of custom hardware,
%  and uses the language
%  to build streaming accelerators.
%Reticle~\cite{vega2021reticle}
%  adds placement information
%  directly into the language
%  representation,
%  conveying useful information
%  to the underlying tools.
%
%\textit{Satisfiability modulo theories} (SMT)
%  is a generalization of the SAT problem
%  allowing for variables and constraints
%  from domains
%  (or \textit{theories})
%  other than boolean,
%  such as the bitvectors
%  and uninterpreted functions.
%Rosette~\cite{torlak2013growing,torlak2014lightweight}
%  embeds SMT solving capabilities
%  within the programming language
%  Racket,
%  enabling users to easily formulate
%  constraint problems,
%  or even use solvers
%  to write programs themselves,
%  also known as 
%  \textit{solver-aided programming}
%  or \textit{synthesis.}
%Under the hood,
%  Rosette is powered by
%  Z3~\cite{de2008z3},
%  a popular SMT solver.
%Rosette and Z3 also support
%  optimization:
%  Rosette via Z3 can produce
%  a solution which satisfies
%  the constraints
%  and which is optimal
%  relative to a given
%  cost function.
%
%Chlorophyll~\cite{phothilimthana2014chlorophyll}
%  is a prime example
%  of how solvers can be used
%  in compilers.
%Chlorophyll
%  uses Z3
%  to compile programs
%  for a unique
%  architecture:
%  an array of power-efficient
%  processors.
%It does so by
%  expressingt he particular,
%  architecture-specific concerns
%  of code compilation
%  as constraints
%  to Z3.