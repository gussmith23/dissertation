\begin{figure} 
\begin{minted}[fontsize=\footnotesize]{python}
def merge(state0, ...):
  new_s = {}
  for s in [state0, ...], key in s:
    (new_val, new_vers) = s.get(key)
    (_, old_vers) = new_s.get_or_default(key, (_, -1))
    if new_vers > old_vers:
      new_s[key] = (new_val, new_vers)
  return new_s

def update(state, key, val):
  new_state = state.copy()
  (_, old_vers) = state.get_or_default(key, (_, -1))
  new_state[key] = (val, old_vers + 1)
  return new_state

def register(clk, d):
  # look up previous values
  s = merge(clk.state, d.state)
  (old_clk_val, _) = s.get_or_default("clk", (0, _))
  (old_d_val, _) = s.get_or_default("d", (0, _))
  (old_q_val, _) = s.get_or_default("q", (0, _))

  # compute the output
  clock_ticked = old_clk_val == 0 and clk.value == 1
  new_q_val = old_d_val if clock_ticked else old_q_value
  new_state = s.update("clk", clk.value)
               .update("d", d.value)
               .update("q", new_q_value)
  
  return signal(new_q_val, new_state)

>>> v0 = register(signal(0), signal(0xa)); v0
signal(0, {"clk": (0, 0), "d": (0xa, 0), "q": (0, 0) })
>>> register(signal(1, v0.state), signal(0xa, v0.state))
signal(0xa, {"clk": (1, 1), "d": (0xa, 1), "q": (0xa, 1) })
\end{minted}
    \caption{
Pseudocode implementation 
  of the core \signal operations
  (\texttt{merge} and \texttt{update}),
  and a demonstration of how \signal{}s
  can be used to implement the semantics
  of a simple register. 
The register takes two \texttt{signal}s as input:
  \texttt{clk} and \texttt{d}.
The states from the two inputs are first merged
  into a single map,
  which is used to look up old values of
  \texttt{clk}, \texttt{d},
  and the output \texttt{q}. 
Version numbers in the state maps
  help resolve potential merge conflicts.
Finally, we compute
  the new value of \texttt{q}
  based on whether or not
  the clock ticked. 
Note that this is just an example;
  \lrir does not include
  a register primitive.
Rather, registers are contained within
  the Verilog models
  we import, and their semantics
  are captured in 
  \cref{fig:verilog-semantics}
  (see \textbf{dff-hold} and \textbf{dff-tick}).
}
    \label{fig:signal-complete}
\end{figure}