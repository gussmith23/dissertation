\begin{figure}
\begin{minted}[fontsize=\footnotesize]{python}
def register(clk, d):
  # look up previous values
  s = merge(clk.state, d.state)
  (old_clk_val, _) = s.get_or_default("clk", (0, _))
  (old_d_val, _) = s.get_or_default("d", (0, _))
  (old_q_val, _) = s.get_or_default("q", (0, _))

  # compute the output
  clock_ticked = old_clk_val == 0 and clk.value == 1
  new_q_val = old_d_val if clock_ticked else old_q_value
  new_state = s.update("clk", clk.value)
               .update("d", d.value)
               .update("q", new_q_value)
  
  return signal(new_q_val, new_state)

>>> v0 = register(signal(0), signal(0xa)); v0
signal(0, {"clk": (0, 0), "d": (0xa, 0), "q": (0, 0) })
>>> register(signal(1, v0.state), signal(0xa, v0.state))
signal(0xa, {"clk": (1, 1), "d": (0xa, 1), "q": (0xa, 1) })
\end{minted}
    \caption{
Pseudocode implementation 
  of a simple register 
  using the \texttt{signal} data structure.
The register takes two \texttt{signal}s as input:
  \texttt{clk} and \texttt{d}.
The states from the two inputs are first merged
  into a single map,
  which is used to look up old values of
  \texttt{clk}, \texttt{d},
  and the output \texttt{q}.
Finally, we compute
  the new value of \texttt{q}
  based on whether or not
  the clock ticked.
}
    \label{fig:signal-example}
\hl{explain: register is a very simple examplke, and we dont' actually have a register primitive, but registers are contained within many of our complex primitives}
\end{figure}