\begin{figure}
\begin{minted}[fontsize=\footnotesize]{python}
def merge(state0, ...):
  new_s = {}
  for s in [state0, ...], key in s:
    (new_val, new_vers) = s.get(key)
    (_, old_vers) = new_s.get_or_default(key, (_, -1))
    if new_vers > old_vers:
      new_s[key] = (new_val, new_vers)
  return new_s

def update(state, key, val):
  new_state = state.copy()
  (_, old_vers) = state.get_or_default(key, (_, -1))
  new_state[key] = (val, old_vers + 1)
  return new_state
\end{minted}
% In case we want to add examples
% >>> v0 = register(signal(0), signal(0xa)); v0
% signal(0, {"clk": 0, "d": 0xa, "q": 0 })
% >>> register(signal(1, v0.state), signal(0xa, v0.state))
% signal(0xa, {"clk": 1, "d": 0xa, "q": 0xa })
    \caption{
% Pseudocode implementation 
%   of a simple register 
%   using the \texttt{signal} data structure.
% The register takes two \texttt{signal}s as input:
%   \texttt{clk} and \texttt{d}.
% The states from the two inputs are first merged
%   into a single map,
%   which is used to look up old values of
%   \texttt{clk}, \texttt{d},
%   and the output \texttt{q}.
% Finally, we compute
%   the new value of \texttt{q}
%   based on whether or not
%   the clock ticked.
Pseudocode implementation
  of merging two state
  maps.
Versioning guarantees
  resolution in the case
  of state merge conflict.
}

    \label{fig:state-merge-and-update}
\end{figure}