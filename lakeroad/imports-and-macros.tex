% Commented out for ASPLOS because causes compile errors (bib generates runaway
% argument errors). I think this is because the `acmart` package defines the
% \cite{} command and there was a conflict

% \usepackage{cite}
% Commented out for ASPLOS because causes compile error (something involving \Bbbk)
%
\DeclareMathAlphabet{\altmathcal}{OMS}{cmsy}{m}{n}
\let\Bbbk\relax
\usepackage{amsmath,amsfonts,amssymb}
% Gus swapping algorithmic for algorithmicx/algpseudocode
%\usepackage{algorithmic}
\usepackage{algpseudocode}
\usepackage{algorithm}
\usepackage{graphicx}
\usepackage{textcomp}
\usepackage{xcolor}
\usepackage{adjustbox}
\usepackage[htt]{hyphenat} % to break \texttt with hyphens
\usepackage{semantic} % for semantics
\usepackage{soul}
% \usepackage{todonotes}
\usepackage{xspace}
% https://github.com/gpoore/minted/issues/113#issuecomment-223451550
%\usepackage[frozencache,cachedir=.]{minted}
\usepackage[cachedir=.]{minted}
\usepackage{pgfplots}
\usepackage{pgfplotstable}
\pgfplotsset{compat=1.17}
\usepackage{catchfile}
\usepackage{ragged2e}
\usepackage{caption}
% Causing error
\usepackage{subcaption} 
\usepackage[utf8]{inputenc}
%\usepackage{syntax} % for grammar
\usepackage{tabularx}
\usepackage{listings}
\usepackage{amsmath}
\usepackage{xspace}
%\usepackage{courier}
\usepackage{float}
\usepackage{microtype}
\usepackage[T1]{fontenc}
%\usepackage{inconsolata}
\usepackage{booktabs}
\usepackage{color}
\usepackage{enumerate}
%\usepackage[inline]{enumitem}
\usepackage{multirow}
\usepackage{syntax}
\usepackage[referable]{threeparttablex}
\usepackage[title]{appendix}
% \usepackage{geometry}
\usepackage{stmaryrd} % for denotation brackets
\usepackage{siunitx} % to align decimal places
\usepackage{enumitem}
%% Added packages
\usepackage{caption}
\usepackage{arydshln}
\usepackage{dashrule}

% Commutative Diagram
\usepackage{tikz-cd}
\usepackage{tikz}

% For ::=
\usepackage{mathtools}
% For inference rules
\usepackage{mathpartir}

\usepackage[capitalize,noabbrev]{cleveref}

%%%% UNCOMMENT TO TURN OFF HIGHLIGHTING
%\renewcommand\hl[1]{}

\newcommand{\sm}[1]{\textbf{\color{blue}[SM: #1]}}
\newcommand{\koika}{Kôika\xspace}
\newcommand{\g}{Glenside\xspace}
%\newcommand{\g}{Lakeroad\xspace}
\newcommand{\hwsw}{hardware--software}
\newcommand{\accesspatternshape}[2]{$($$\left( #1 \right)$, $\left( #2 \right)$$)$}
%\newcommand{\accesspatternshape}[2]{$( ( #1 ) , ( #2 ) )$}
\newcommand{\itc}{\texttt{im2col}\xspace}
\newcommand{\ctd}{\texttt{conv2d}\xspace}
\newcommand{\egr}{egraph\xspace}
\newcommand{\egrs}{\egr{}s\xspace}
\newcommand{\Egr}{Egraph\xspace}
\newcommand{\Egrs}{\Egr{}s\xspace}

% "codeword" ie im2col and conv2d
\newcommand{\tcd}[1]{\texttt{#1}}
\newcommand{\mcd}[1]{\mathrm{\tcd{#1}}}

% The following macros capture high-level ideas
\newcommand{\ThePlatonicApproach}{Solver-Aided FPGA ISA Implementation\xspace}
\newcommand{\theplatonicapproach}{solver-aided FPGA ISA implementation\xspace}
\newcommand{\Theplatonicapproach}{Solver-aided FPGA ISA implementation\xspace}
\newcommand{\note}[1]{{\color{red}{\bf Note:} \textit{#1}}}
% Make a Suggestion
\newcommand{\sug}[2]{{\color{red}\st{#1}}\xspace{} {\color{teal}#2}\xspace{}}
\newcommand{\lookupt}{look-up table\xspace}
\newcommand{\Lookupt}{Look-up table\xspace}
\newcommand{\lut}{LUT\xspace}

% For denotational Semantics
\newcommand{\denote}[1]{\llbracket {#1} \rrbracket}
% The following commands represent the three lakeroad components
\newcommand{\SketchGeneration}{Sketch Generation\xspace}
\newcommand{\HDLCompilation}{HDL Compilation\xspace}


% \newcommand{\lr}{Lakeroad\xspace}
% \newcommand{\lowercaselr}{lakeroad\xspace}
% \newcommand{\lrfn}{\text{$f_{\text{LR}}$}\xspace}
% \newcommand{\lrir}{\textsc{LRIR}\xspace}
% \newcommand{\LRSynthesis}{Lakeroad Synthesis\xspace}
% \newcommand{\LRSynth}{\textsc{LrSynth}\xspace}
% \newcommand{\lowlrir}{lrir}

\newcommand{\lr}{Lakeroad\xspace}
\newcommand{\lowercaselr}{lakeroad\xspace}
\newcommand{\lrfn}{\text{$f_{\textsc{lr}}$}\xspace}   % Lakeroad as a function
\newcommand{\lrfnbmc}{\text{$f_{\textsc{lr}}^{*}$}\xspace}   % Lakeroad as a function
\newcommand{\lrir}{\textsc{LRIR}\xspace}
\newcommand{\LRSynthesis}{Lakeroad Synthesis\xspace}
\newcommand{\LRSynth}{\textsc{LrSynth}\xspace}
\newcommand{\lowlrir}{lrir}


%%%%%%%%%%%%%%%%%%%%%%%%%%%%%%%%%%%%%%%%%%%%%%%%%%%%%%%%%%%%%%%%%%%%%%%%%%%%%
%%%%%%%%%%%%%%%%%%%%%%%% Syntax and Semantics Macros %%%%%%%%%%%%%%%%%%%%%%%%
%%%%%%%%%%%%%%%%%%%%%%%%%%%%%%%%%%%%%%%%%%%%%%%%%%%%%%%%%%%%%%%%%%%%%%%%%%%%%

%%% Declaring some syntax variables.
\newcommand{\SynProg}{\mathsf{Prog}\xspace}
\newcommand{\SynId}{\mathsf{Id}\xspace}
\newcommand{\SynBv}{\mathsf{BV}\xspace}
\newcommand{\SynVar}{\mathsf{Var}\xspace}
\newcommand{\SynNode}{\mathsf{Node}\xspace}
\newcommand{\SynPrim}{\mathsf{Prim}\xspace}
\newcommand{\Op}{\mathsf{OP}\xspace}
\newcommand{\OpBv}{\Op_{bv}}
\newcommand{\OpWire}{\Op_{w}}
\newcommand{\IRReg}{\lstinline[language=thelang]{Reg}\xspace}
\newcommand{\IRPrim}{\lstinline[language=thelang]{Prim}\xspace}

\newcommand{\Time}{\textsf{Time}\xspace}


\newcommand{\Spec}{\textsc{Desg}\xspace}
\newcommand{\VendorSemantics}{\textsc{Sems}\xspace}
\newcommand{\Impl}{\textsc{Impl}\xspace}
\newcommand{\HDL}{\textsc{HDL}\xspace}
\newcommand{\SketchTemplate}{\textsc{Template}\xspace}
\newcommand{\Sketch}{\textsc{Sketch}\xspace}
\newcommand{\ArchDescr}{\textsc{ArchDescr}\xspace}
\newcommand{\HDLCompile}{\textsc{HDLComp}\xspace}
\newcommand{\Design}{\textsc{Desg}\xspace}
\newcommand{\Semantics}{\textsc{Sems}\xspace}

% 3LA placeholder
\newcommand{\TLA}{3LA\xspace}
\newcommand{\AppNum}{six\xspace}
\newcommand{\egg}{\texttt{egg}\xspace}
% instruction name
\newcommand{\instrInText}[1]{\texttt{\small #1}}
% variable name
\newcommand{\varInText}[1]{$\texttt{#1}$}
% markers
\newcommand{\cmark}{{\color{deepgreen}$\boldsymbol\vee$}}
\newcommand{\xmark}{{\color{deepred}$\boldsymbol\times$}}
\newcommand{\diy}{{\color{deepblue}$\boldsymbol\thicksim$}}
\definecolor{navy}{HTML}{0f1566}
\newcommand{\asplos}[1]{{\color{navy}#1}}
%% inline enumerating style
% \newlist{inlinelist}{enumerate*}{1}
% \setlist*[inlinelist,1]{%
%   label=(\arabic*),
% }
%% paragraph no indent
\newcommand{\para}[1]{\paragraph{\hspace{-1em}#1}}

%% norm expression in math equation
\newcommand{\norm}[1]{\left\lVert#1\right\rVert}

\newcommand{\gs}[1]{\todo[color=gray!60]{\sf #1}}

%%%%%%%%%%% Theorem Macro %%%%%%%%%%%%
% \newtheorem{theorem}{Theorem}
\newtheorem{property}{Property}
\definecolor{mygreen}{rgb}{0,0.6,0}
\lstdefinestyle{lispstyle}{
  backgroundcolor=\color{white},
  basicstyle=\ttfamily\footnotesize,
  breakatwhitespace=false,
  breaklines=true,
  captionpos=b,
  commentstyle=\color{mygreen},
  % escapeinside={\%*}{*)},          % if you want to add LaTeX within your code
  extendedchars=true,
  keepspaces=true,
  keywordstyle=\color{black},
  language=Lisp,
  morekeywords={*,...},
  numbers=none,
  numbersep=5pt,
  numberstyle=\tiny\color{mygray},
  rulecolor=\color{black},
  showspaces=false,
  showstringspaces=false,
  showtabs=false,
  stringstyle=\color{black},
  tabsize=2,
  title=\lstname
}
\lstdefinestyle{pystyle}{
  backgroundcolor=\color{white},
  basicstyle=\ttfamily\footnotesize,
  breakatwhitespace=false,
  breaklines=true,
  captionpos=b,
  commentstyle=\color{mygreen},
  % escapeinside={\%*}{*)},          % if you want to add LaTeX within your code
  extendedchars=true,
  keepspaces=true,
  keywordstyle=\color{blue},
  language=Python,
  morekeywords={*,...},
  numbers=none,
  numbersep=5pt,
  numberstyle=\tiny\color{mygray},
  rulecolor=\color{black},
  showspaces=false,
  showstringspaces=false,
  showtabs=false,
  stringstyle=\color{black},
  tabsize=2,
  title=\lstname
}

% Our default style will be lispstyle
\lstset{style=lispstyle}

%%%%%%%%%%%%%%%%%%%%%%%%%%%%%%%%%%%%%%%%%%%%%%%%%%%%%%
%%%%%%%%%%% YAML syntax highlighting %%%%%%%%%%%%%%%%%

% http://tex.stackexchange.com/questions/152829/how-can-i-highlight-yaml-code-in-a-pretty-way-with-listings

% here is a macro expanding to the name of the language
% (handy if you decide to change it further down the road)
\newcommand\YAMLcolonstyle{\color{black}\mdseries}
\newcommand\YAMLkeystyle{\color{black}\bfseries}
\newcommand\YAMLvaluestyle{\color{black}\mdseries}

\makeatletter

\newcommand\language@yaml{yaml}

\expandafter\expandafter\expandafter\lstdefinelanguage
\expandafter{\language@yaml}
{
  keywords={true,false,null,y,n},
  keywordstyle=\color{darkgray}\bfseries,
  basicstyle=\color{black}\linespread{0.8}\ttfamily\footnotesize,
  comment=[l]{\#},
  morecomment=[s]{/*}{*/},
  commentstyle=\color{purple}\ttfamily,
  stringstyle=\YAMLvaluestyle\ttfamily,
  moredelim=[l][\color{orange}]{\&},
  moredelim=[l][\color{magenta}]{*},
  moredelim=**[il][\YAMLcolonstyle{:}\YAMLvaluestyle]{:},   % switch to value style at :
  morestring=[b]',
  morestring=[b]",
  literate =    {---}{{\ProcessThreeDashes}}3
                {>}{{\textcolor{red}\textgreater}}1     
                {|}{{\textcolor{red}\textbar}}1 
                {\ -\ }{{\mdseries\ -\ }}3,
}

% switch to key style at EOL
\lst@AddToHook{EveryLine}{\ifx\lst@language\language@yaml\YAMLkeystyle\fi}
\makeatother

\newcommand\ProcessThreeDashes{\llap{\color{cyan}\mdseries-{-}-}}

%%%%%%%%%%% YAML syntax highlighting %%%%%%%%%%%%%%%%%
%%%%%%%%%%%%%%%%%%%%%%%%%%%%%%%%%%%%%%%%%%%%%%%%%%%%%%

% \newcommand{\qed}{\blacksquare}

% Concatenation
\newcommand\mdoubleplus{\mathbin{+\mkern-10mu+}}

% Signal
\newcommand{\signal}{\texttt{signal}\xspace}

%%%%%%%%%%%%%%%%%%%%%%%%%%%%%%%%%%%%%%%%%%%%%%%%%%%%%%

\DeclareMathOperator{\defn}{\Coloneqq}
\DeclareMathOperator{\vbar}{\, |\, }
\lstdefinelanguage{thelang}{
  basicstyle=\ttfamily,
  keywordstyle=\color{black}\bfseries,
  morekeywords=[1]{let,in,:=,Reg,Prim,Op},
  morekeywords=[2]{},
  morekeywords=[3]{},
  alsoletter={:=},
  morestring=[b]",
  morecomment=[l]{\#},
  morecomment=[s]{(*}{*)},
  moredelim=**[is][\color{white}]{(&}{&)},
}
\newcommand{\nterm}[1]{\langle#1\rangle}
\newcommand{\seq}[1]{\langle#1\rangle}
\newcommand{\set}[1]{\{#1\}}

\newcommand{\smallStepSymbol}{\to}
\newcommand{\bigStepSymbol}{\downarrow}
\newcommand{\smallStep}[5]{\ensuremath{#1,#2,#3,#4 \smallStepSymbol #5}}
\newcommand{\bigStep}[4]{\ensuremath{#1, #2, #3 \bigStepSymbol #4}}
\newcommand{\evalsTo}[4]{\ensuremath{#1, #2, #3 \to^{+} #4}}
\newcommand{\multiEvalsTo}[5]{\ensuremath{#1, #2, #3 \to^{+} #4, #5}}


\DeclareMathOperator{\defaultop}{\ \triangleright\ }

\newcommand{\UberLang}{\ensuremath{\altmathcal{L}_\textsc{lr}}\xspace}
\newcommand{\SpecLang}{\ensuremath{\altmathcal{L}_\textsc{beh}}\xspace}
\newcommand{\ImplLang}{\ensuremath{\altmathcal{L}_\textsc{struct}}\xspace}
\newcommand{\SketchLang}{\ensuremath{\altmathcal{L}_\textsc{sketch}}\xspace}

\newcommand{\separatingline}{{\color{gray}\hrule}}

\newcommand{\tighten}{\looseness=-1}
