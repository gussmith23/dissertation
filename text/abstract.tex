Compilers convert
  between representations---%
  usually, from 
  higher-level, human writable code
  to lower-level,
  machine-readable code.
A compiler backend is
  the portion of the 
  compiler containing
  optimizations
  and code generation routines
  for a specific hardware target.
% Compiler backends are constructed
%   using
%   a number of core
%   algorithms;
%   inbuilt into these algorithms
%   are 
%   models of the target hardware.
%
In this dissertation,
  I advocate for a specific way of
  building compiler backends:
  namely, by automatically generating them
  from explicit, formal models of hardware
  using automated reasoning algorithms.
I describe how automatically generating compilers
  from formal models of hardware
  leads to increased optimization ability,
  stronger correctness guarantees,
  and reduced development time
  for compiler backends.
%
As evidence, I present two
  case studies:
  first, \g,
  which uses
  equality saturation
  to increase the \TLA compiler's
  ability to offload operations
  to machine learning accelerators,
  and second,
  \lr,
  a technology mapper for FPGAs
  which uses program synthesis
  and semantics extracted from Verilog
  to map hardware designs
  to complex, programmable hardware primitives.
% Glenside and Lakeroad
%   both demonstrate
%   how,
%   by automatically generating
%   portions of compiler backends
%   using more adaptable algorithms
%   and more explicit models of hardware,
%   we we improve their
%   optimization ability and
%   correctness, while
%   easing development effort.
