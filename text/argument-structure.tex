\section{Argument Structure}

This section outlines the hierarchical
  argument
  of this dissertation.
At the top is the 
  thesis of the dissertation;
  everything else present
  in this dissertation
  is meant to prove this thesis.
The next level of bullets
  represent the arguments
  which directly support this thesis.
Following from this,
  each subsequent level of bullets
  represent the arguments
  which support the previous level of
  arguments.
We use the symbol ``$\Leftarrow$''
  to capture the fact that
  each sublevel of arguments
  taken together
  should imply their parent claim.
  
  

\textbf{Thesis:}
  Automatically generating
  compiler backends
  from formal models of hardware
  improves correctness,
  leads to emergent optimization,
  and increases extensibility.
\begin{itemize}[label=$\Leftarrow$]
 \item Generating the backend
      for a compiler for deep learning
      accelerators
      by modeling hardware
      as rewrite rules
      leads to
      emergent optimizations
      and more mapping opportunities.
 \begin{itemize}[label=$\Leftarrow$]
  \item blah
 \end{itemize}
 \item Generating an FPGA technology
   mapper
   using semantics
   automatically extracted
   from Verilog simulation models
   leads to greater correctness,
   completeness,
   and extensibility.
    \begin{itemize}
        \item The models already exist in the form of Verilog simulation models.
    \end{itemize}
\end{itemize}