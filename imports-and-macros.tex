% General thesis imports
\usepackage{mdframed}
\usepackage{glossaries}

% Bold the first occurrence of a glossary word
\DeclareRobustCommand{\glossfirstformat}[1]{\textit{#1}}
\renewcommand*{\glsdisplayfirst}[4]{\glossfirstformat{#1#4}}
\usepackage[normalem]{ulem}
\renewcommand*{\glstextformat}[1]{\color{lightgray}{\dotuline{\color{black} #1}}}


% \makeglossaries
\makenoidxglossaries

% NOTE: use \glstext and not \gls in this file.
% This prevents the glossary from thinking that uses of these words in the glossary are the first occurrence in text. We'd rather those first occurrences happen e.g. in the intro.

\newglossaryentry{compiler}
{
    name=compiler,
    description={
A tool which converts between representations.
    }
}


\newglossaryentry{target}
{
    name=target,
    description={
The platform which a \glstext{compiler} generates code for.
    }
}

\newglossaryentry{compilerbackend}
{
    name=compiler backend,
    description={
The portion of a \glstext{compiler} that concerns the \glstext{target},
  e.g., target-specific optimizations and code generation.
    }
}

\newglossaryentry{validation}
{
    name=validation,
    description={
The process of sanity-checking a program or hardware design using a
  limited, finite set
  of inputs, and judging the correctness 
  of the outputs.
Also referred to as \textit{testing.}
Validation should specifically be distinguished from
  \glstext{verification}.
Confusingly, 
  in the world of hardware design,
  validation is often referred to as
  verification,
  while verification (by our definition)
  is referred to as \textit{formal} verification.
    }
}

\newglossaryentry{verification}
{
    name=verification,
    description={
The process of mathematically proving correctness
  about a program or hardware design.
While the proofs of correctness themselves
  can have limitations,
  verification is generally more thorough
  than \glstext{validation} or \textit{testing.}
In the world of hardware design,
  ``verification''
  generally refers to what we call validation or testing,
  while ``formal verification''
  refers to our verification.
    }
}

\newglossaryentry{hardwaresynthesis}
{
    name={hardware synthesis},
    description={
Another term for hardware compilation.
The process of converting a hardware design
  captured in a high-level language
  to a low-level target implementation,
  for example,
  a netlist which can be programmed onto an FPGA
  or a geometry file to be made into an ASIC.
  \hl{define the terms in here}
    }
}

% Note: this is a nice hack from
% https://tex.stackexchange.com/questions/8946/how-to-combine-acronym-and-glossary
% which gives acronym-like behavior but
% still allows for a definition.
\newglossaryentry{dsl}
{
    name={DSL},
    first={domain-specific language (DSL)},
    description={
A language designed for a custom purpose \hl{fill out}
    }
}

\newglossaryentry{mlkernel}
{
    name={machine learning kernel},
    description={
A core subroutine used within many machine learning workloads---for example,
  matrix multiplication.
Machine learning kernels are often highly optimized,
  often by building custom hardware to implement
  them efficiently.
    }
}

% Glenside imports
\usepackage{soul}
\usepackage{ragged2e}
\usepackage{caption}
\usepackage{subcaption}
\usepackage[utf8]{inputenc}
%\usepackage{syntax} % for grammar
\usepackage{tabularx}
\usepackage{listings}
\usepackage{amsmath}
\usepackage{xspace}
%\usepackage{courier}
\usepackage{float}
\usepackage{enumitem}
\usepackage{microtype}
% TODO(@gussmith23): Does this break anything? I used it to remove a warning.
\microtypesetup{nopatch={footnote}}
\usepackage[T1]{fontenc}
%\usepackage{inconsolata}
\usepackage{graphicx}
\usepackage{hyperref}
\usepackage[table]{xcolor}
%\usepackage{amssymb}
\usepackage{wrapfig} % For \begin{wrapfigure}
\usepackage{cleveref}
\usepackage{bibentry}
\usepackage{minted}
\usepackage{tikz}
\usetikzlibrary{positioning, 
                quotes}


\lstset{columns=fullflexible}
\lstset{basicstyle=\small\ttfamily,breaklines=true,keepspaces=true,}


\newcommand{\g}{Glenside\xspace}
%\newcommand{\g}{Lakeroad\xspace}
\newcommand{\hwsw}{hardware--software}
\newcommand{\accesspatternshape}[2]{$($$\left( #1 \right)$, $\left( #2 \right)$$)$}
%\newcommand{\accesspatternshape}[2]{$( ( #1 ) , ( #2 ) )$}
\newcommand{\itc}{\texttt{im2col}\xspace}
\newcommand{\ctd}{\texttt{conv2d}\xspace}
\newcommand{\egr}{egraph\xspace}
\newcommand{\egrs}{\egr{}s\xspace}
\newcommand{\Egr}{Egraph\xspace}
\newcommand{\Egrs}{\Egr{}s\xspace}

% "codeword" ie im2col and conv2d
\newcommand{\tcd}[1]{\texttt{#1}}
\newcommand{\mcd}[1]{\mathrm{\tcd{#1}}}

\newcommand\cse{Computer Science \& Engineering}
\newcommand\pgas{Paul G.\ Allen School of \cse}


% 3LA macros
% 3LA placeholder
\newcommand{\TLA}{3LA\xspace}
\newcommand{\AppNum}{six\xspace}
\newcommand{\egg}{\texttt{egg}\xspace}
% instruction name
\newcommand{\instrInText}[1]{\texttt{\small #1}}
% variable name
\newcommand{\varInText}[1]{$\texttt{#1}$}
% markers
\newcommand{\cmark}{{\color{deepgreen}$\boldsymbol\vee$}}
\newcommand{\xmark}{{\color{deepred}$\boldsymbol\times$}}
\newcommand{\diy}{{\color{deepblue}$\boldsymbol\thicksim$}}
\newcommand{\mapping}{IR-to-accelerator mapping\xspace}

% Lakeroad imports
\usepackage{amsfonts}
\usepackage{algorithm}
\usepackage{textcomp}
\usepackage{adjustbox}
\usepackage[htt]{hyphenat} % to break \texttt with hyphens
\usepackage{semantic} % for semantics
% https://github.com/gpoore/minted/issues/113#issuecomment-223451550
%\usepackage[frozencache,cachedir=.]{minted}
% \usepackage[cachedir=.]{minted}
\usepackage{pgfplots}
\usepackage{pgfplotstable}
\pgfplotsset{compat=1.17}
\usepackage{catchfile}
\usepackage{ragged2e}
\usepackage{subcaption} 
\usepackage[utf8]{inputenc}
\usepackage{tabularx}
\usepackage{listings}
\usepackage{booktabs}
\usepackage{color}
\usepackage{multirow}
\usepackage{syntax}
\usepackage[referable]{threeparttablex}
\usepackage[title]{appendix}
\usepackage{stmaryrd} % for denotation brackets
\usepackage{siunitx} % to align decimal places
\usepackage{arydshln}
\usepackage{dashrule}
\usepackage{tikz-cd}
\usepackage{mathtools}
\usepackage{mathpartir}

% Lakeroad macros
\newcommand{\sm}[1]{\textbf{\color{blue}[SM: #1]}}
\newcommand{\koika}{Kôika\xspace}

% The following macros capture high-level ideas
\newcommand{\ThePlatonicApproach}{Solver-Aided FPGA ISA Implementation\xspace}
\newcommand{\theplatonicapproach}{solver-aided FPGA ISA implementation\xspace}
\newcommand{\Theplatonicapproach}{Solver-aided FPGA ISA implementation\xspace}
\newcommand{\note}[1]{{\color{red}{\bf Note:} \textit{#1}}}
% Make a Suggestion
\newcommand{\sug}[2]{{\color{red}\st{#1}}\xspace{} {\color{teal}#2}\xspace{}}
\newcommand{\lookupt}{look-up table\xspace}
\newcommand{\Lookupt}{Look-up table\xspace}
\newcommand{\lut}{LUT\xspace}

% For denotational Semantics
\newcommand{\denote}[1]{\llbracket {#1} \rrbracket}
% The following commands represent the three lakeroad components
\newcommand{\SketchGeneration}{Sketch Generation\xspace}
\newcommand{\HDLCompilation}{HDL Compilation\xspace}


% \newcommand{\lr}{Lakeroad\xspace}
% \newcommand{\lowercaselr}{lakeroad\xspace}
% \newcommand{\lrfn}{\text{$f_{\text{LR}}$}\xspace}
% \newcommand{\lrir}{\textsc{LRIR}\xspace}
% \newcommand{\LRSynthesis}{Lakeroad Synthesis\xspace}
% \newcommand{\LRSynth}{\textsc{LrSynth}\xspace}
% \newcommand{\lowlrir}{lrir}

\newcommand{\lr}{Lakeroad\xspace}
\newcommand{\lowercaselr}{lakeroad\xspace}
\newcommand{\lrfn}{\text{$f_{\textsc{lr}}$}\xspace}   % Lakeroad as a function
\newcommand{\lrfnbmc}{\text{$f_{\textsc{lr}}^{*}$}\xspace}   % Lakeroad as a function
\newcommand{\lrir}{\textsc{LRIR}\xspace}
\newcommand{\LRSynthesis}{Lakeroad Synthesis\xspace}
\newcommand{\LRSynth}{\textsc{LrSynth}\xspace}
\newcommand{\lowlrir}{lrir}


%%%%%%%%%%%%%%%%%%%%%%%%%%%%%%%%%%%%%%%%%%%%%%%%%%%%%%%%%%%%%%%%%%%%%%%%%%%%%
%%%%%%%%%%%%%%%%%%%%%%%% Syntax and Semantics Macros %%%%%%%%%%%%%%%%%%%%%%%%
%%%%%%%%%%%%%%%%%%%%%%%%%%%%%%%%%%%%%%%%%%%%%%%%%%%%%%%%%%%%%%%%%%%%%%%%%%%%%

%%% Declaring some syntax variables.
\newcommand{\SynProg}{\mathsf{Prog}\xspace}
\newcommand{\SynId}{\mathsf{Id}\xspace}
\newcommand{\SynBv}{\mathsf{BV}\xspace}
\newcommand{\SynVar}{\mathsf{Var}\xspace}
\newcommand{\SynNode}{\mathsf{Node}\xspace}
\newcommand{\SynPrim}{\mathsf{Prim}\xspace}
\newcommand{\Op}{\mathsf{OP}\xspace}
\newcommand{\OpBv}{\Op_{bv}}
\newcommand{\OpWire}{\Op_{w}}
\newcommand{\IRReg}{\lstinline[language=thelang]{Reg}\xspace}
\newcommand{\IRPrim}{\lstinline[language=thelang]{Prim}\xspace}

\newcommand{\Time}{\textsf{Time}\xspace}


\newcommand{\Spec}{\textsc{Desg}\xspace}
\newcommand{\VendorSemantics}{\textsc{Sems}\xspace}
\newcommand{\Impl}{\textsc{Impl}\xspace}
\newcommand{\HDL}{\textsc{HDL}\xspace}
\newcommand{\SketchTemplate}{\textsc{Template}\xspace}
\newcommand{\Sketch}{\textsc{Sketch}\xspace}
\newcommand{\ArchDescr}{\textsc{ArchDescr}\xspace}
\newcommand{\HDLCompile}{\textsc{HDLComp}\xspace}
\newcommand{\Design}{\textsc{Desg}\xspace}
\newcommand{\Semantics}{\textsc{Sems}\xspace}

\definecolor{navy}{HTML}{0f1566}
\newcommand{\asplos}[1]{{\color{navy}#1}}
%% inline enumerating style
% \newlist{inlinelist}{enumerate*}{1}
% \setlist*[inlinelist,1]{%
%   label=(\arabic*),
% }
%% paragraph no indent
\newcommand{\para}[1]{\paragraph{\hspace{-1em}#1}}

%% norm expression in math equation
\newcommand{\norm}[1]{\left\lVert#1\right\rVert}

\newcommand{\gs}[1]{\todo[color=gray!60]{\sf #1}}

%%%%%%%%%%% Theorem Macro %%%%%%%%%%%%
% \newtheorem{theorem}{Theorem}
\newtheorem{property}{Property}
\definecolor{mygreen}{rgb}{0,0.6,0}
\lstdefinestyle{lispstyle}{
  backgroundcolor=\color{white},
  basicstyle=\ttfamily\footnotesize,
  breakatwhitespace=false,
  breaklines=true,
  captionpos=b,
  commentstyle=\color{mygreen},
  % escapeinside={\%*}{*)},          % if you want to add LaTeX within your code
  extendedchars=true,
  keepspaces=true,
  keywordstyle=\color{black},
  language=Lisp,
  morekeywords={*,...},
  numbers=none,
  numbersep=5pt,
  numberstyle=\tiny\color{mygray},
  rulecolor=\color{black},
  showspaces=false,
  showstringspaces=false,
  showtabs=false,
  stringstyle=\color{black},
  tabsize=2,
  title=\lstname
}
\lstdefinestyle{pystyle}{
  backgroundcolor=\color{white},
  basicstyle=\ttfamily\footnotesize,
  breakatwhitespace=false,
  breaklines=true,
  captionpos=b,
  commentstyle=\color{mygreen},
  % escapeinside={\%*}{*)},          % if you want to add LaTeX within your code
  extendedchars=true,
  keepspaces=true,
  keywordstyle=\color{blue},
  language=Python,
  morekeywords={*,...},
  numbers=none,
  numbersep=5pt,
  numberstyle=\tiny\color{mygray},
  rulecolor=\color{black},
  showspaces=false,
  showstringspaces=false,
  showtabs=false,
  stringstyle=\color{black},
  tabsize=2,
  title=\lstname
}

% Our default style will be lispstyle
\lstset{style=lispstyle}

%%%%%%%%%%%%%%%%%%%%%%%%%%%%%%%%%%%%%%%%%%%%%%%%%%%%%%
%%%%%%%%%%% YAML syntax highlighting %%%%%%%%%%%%%%%%%

% http://tex.stackexchange.com/questions/152829/how-can-i-highlight-yaml-code-in-a-pretty-way-with-listings

% here is a macro expanding to the name of the language
% (handy if you decide to change it further down the road)
\newcommand\YAMLcolonstyle{\color{black}\mdseries}
\newcommand\YAMLkeystyle{\color{black}\bfseries}
\newcommand\YAMLvaluestyle{\color{black}\mdseries}

\makeatletter

\newcommand\language@yaml{yaml}

\expandafter\expandafter\expandafter\lstdefinelanguage
\expandafter{\language@yaml}
{
  keywords={true,false,null,y,n},
  keywordstyle=\color{darkgray}\bfseries,
  basicstyle=\color{black}\linespread{0.8}\ttfamily\footnotesize,
  comment=[l]{\#},
  morecomment=[s]{/*}{*/},
  commentstyle=\color{purple}\ttfamily,
  stringstyle=\YAMLvaluestyle\ttfamily,
  moredelim=[l][\color{orange}]{\&},
  moredelim=[l][\color{magenta}]{*},
  moredelim=**[il][\YAMLcolonstyle{:}\YAMLvaluestyle]{:},   % switch to value style at :
  morestring=[b]',
  morestring=[b]",
  literate =    {---}{{\ProcessThreeDashes}}3
                {>}{{\textcolor{red}\textgreater}}1     
                {|}{{\textcolor{red}\textbar}}1 
                {\ -\ }{{\mdseries\ -\ }}3,
}

% switch to key style at EOL
\lst@AddToHook{EveryLine}{\ifx\lst@language\language@yaml\YAMLkeystyle\fi}
\makeatother

\newcommand\ProcessThreeDashes{\llap{\color{cyan}\mdseries-{-}-}}

%%%%%%%%%%% YAML syntax highlighting %%%%%%%%%%%%%%%%%
%%%%%%%%%%%%%%%%%%%%%%%%%%%%%%%%%%%%%%%%%%%%%%%%%%%%%%

% \newcommand{\qed}{\blacksquare}

% Concatenation
\newcommand\mdoubleplus{\mathbin{+\mkern-10mu+}}

% Signal
\newcommand{\signal}{\texttt{signal}\xspace}

%%%%%%%%%%%%%%%%%%%%%%%%%%%%%%%%%%%%%%%%%%%%%%%%%%%%%%

\DeclareMathOperator{\defn}{\Coloneqq}
\DeclareMathOperator{\vbar}{\, |\, }
\lstdefinelanguage{thelang}{
  basicstyle=\ttfamily,
  keywordstyle=\color{black}\bfseries,
  morekeywords=[1]{let,in,:=,Reg,Prim,Op},
  morekeywords=[2]{},
  morekeywords=[3]{},
  alsoletter={:=},
  morestring=[b]",
  morecomment=[l]{\#},
  morecomment=[s]{(*}{*)},
  moredelim=**[is][\color{white}]{(&}{&)},
}
\newcommand{\nterm}[1]{\langle#1\rangle}
\newcommand{\seq}[1]{\langle#1\rangle}
\newcommand{\set}[1]{\{#1\}}

\newcommand{\smallStepSymbol}{\to}
\newcommand{\bigStepSymbol}{\downarrow}
\newcommand{\smallStep}[5]{\ensuremath{#1,#2,#3,#4 \smallStepSymbol #5}}
\newcommand{\bigStep}[4]{\ensuremath{#1, #2, #3 \bigStepSymbol #4}}
\newcommand{\evalsTo}[4]{\ensuremath{#1, #2, #3 \to^{+} #4}}
\newcommand{\multiEvalsTo}[5]{\ensuremath{#1, #2, #3 \to^{+} #4, #5}}


\DeclareMathOperator{\defaultop}{\ \triangleright\ }

\newcommand{\UberLang}{\ensuremath{\altmathcal{L}_\textsc{lr}}\xspace}
\newcommand{\SpecLang}{\ensuremath{\altmathcal{L}_\textsc{beh}}\xspace}
\newcommand{\ImplLang}{\ensuremath{\altmathcal{L}_\textsc{struct}}\xspace}
\newcommand{\SketchLang}{\ensuremath{\altmathcal{L}_\textsc{sketch}}\xspace}

\newcommand{\separatingline}{{\color{gray}\hrule}}

\newcommand{\tighten}{\looseness=-1}

\DeclareMathAlphabet{\altmathcal}{OMS}{cmsy}{m}{n}

\newtheorem{lemma}[theorem]{Lemma}


%%%%%%%%%%%%%%%%%%% THESIS STATEMENT MACROS %%%%%%%%%

% Ubiquity claim
%
% The \crefformat allows us to define a \label for the claim
% and then later \cref that label to produce a nice one-word
% cross reference to our thesis point.
\newcommand{\thesisubiquitylabel}{\textbf{\textsc{Ubiquity}}}
\crefformat{thesis:ubiquity}{#2\thesisubiquitylabel#3}
\newcommand{\thesisclaimubiquity}{
    Formal models of hardware
      are already ubiquitous,
      but underutilized.
}
% Optimizations claim
%
\newcommand{\thesisoptimizationslabel}{\textbf{\textsc{Optimizations}}}
\crefformat{thesis:optimizations}{#2\thesisoptimizationslabel#3}
\newcommand{\thesisclaimoptimizations}{
    Automatically generating compiler backends
      gives rise to emergent optimizations.
}
%
% Dev time claim
\newcommand{\thesisdevtimelabel}{\textbf{\textsc{Devtime}}}
\crefformat{thesis:devtime}{#2\thesisdevtimelabel#3}
\newcommand{\thesisclaimdevtime}{
    Automatically generating compiler backends
      reduces development time.
}
%
% Correctness claim
\newcommand{\thesiscorrectnesslabel}{\textbf{\textsc{Correctness}}}
\crefformat{thesis:correctness}{#2\thesiscorrectnesslabel#3}
\newcommand{\thesisclaimcorrectness}{
    Automatically generating compiler backends
      enables verification.
}
\newcommand{\thesisalgorithmslabel}{\textbf{\textsc{Algorithms}}}
\crefformat{thesis:algorithms}{#2\thesisalgorithmslabel#3}
\newcommand{\thesismodelslabel}{\textbf{\textsc{Models}}}
\crefformat{thesis:models}{#2\thesismodelslabel#3}
\newcommand{\mythesis}{
Compiler backends
  should be generated
  (\thesisalgorithmslabel)
  from explicit formal models
  (\thesismodelslabel)
  of the hardware they target,
  for the following reasons:
  \label[thesis:algorithms]{thesis:algorithms}
  \label[thesis:models]{thesis:models}
  \hl{rephrase}

  \hl{HYPERLINK LOCATIONS ARE STILL WRONG!!!}

% Still not sure on itemindent/leftmargin.
% Do I want the hanging indent?
\begin{itemize}[%label=\textbf{Claim \arabic*},
                itemindent=0pt,leftmargin=90pt]
                
  % \item[\thesisubiquitylabel] \label[thesis:ubiquity]{thesis:ubiquity}
  %   \thesisclaimubiquity
    
  \item[\thesisoptimizationslabel] 
    \label[thesis:optimizations]{thesis:optimizations} 
    \thesisclaimoptimizations
  
  \item[\thesisdevtimelabel] 
    \label[thesis:devtime]{thesis:devtime} \thesisclaimdevtime
    
  \item[\thesiscorrectnesslabel]
    \label[thesis:correctness]{thesis:correctness}
    \thesisclaimcorrectness
\end{itemize}
}

%%%%%%%%%%%%%%%%%%% end THESIS STATEMENT MACROS %%%%%

% From 3LA
%% inline enumerating style
\newlist{inlinelist}{enumerate*}{1}
\setlist*[inlinelist,1]{%
  label=(\arabic*)
}